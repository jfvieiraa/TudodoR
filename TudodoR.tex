% Options for packages loaded elsewhere
\PassOptionsToPackage{unicode}{hyperref}
\PassOptionsToPackage{hyphens}{url}
%
\documentclass[
]{book}
\usepackage{lmodern}
\usepackage{amssymb,amsmath}
\usepackage{ifxetex,ifluatex}
\ifnum 0\ifxetex 1\fi\ifluatex 1\fi=0 % if pdftex
  \usepackage[T1]{fontenc}
  \usepackage[utf8]{inputenc}
  \usepackage{textcomp} % provide euro and other symbols
\else % if luatex or xetex
  \usepackage{unicode-math}
  \defaultfontfeatures{Scale=MatchLowercase}
  \defaultfontfeatures[\rmfamily]{Ligatures=TeX,Scale=1}
\fi
% Use upquote if available, for straight quotes in verbatim environments
\IfFileExists{upquote.sty}{\usepackage{upquote}}{}
\IfFileExists{microtype.sty}{% use microtype if available
  \usepackage[]{microtype}
  \UseMicrotypeSet[protrusion]{basicmath} % disable protrusion for tt fonts
}{}
\makeatletter
\@ifundefined{KOMAClassName}{% if non-KOMA class
  \IfFileExists{parskip.sty}{%
    \usepackage{parskip}
  }{% else
    \setlength{\parindent}{0pt}
    \setlength{\parskip}{6pt plus 2pt minus 1pt}}
}{% if KOMA class
  \KOMAoptions{parskip=half}}
\makeatother
\usepackage{xcolor}
\IfFileExists{xurl.sty}{\usepackage{xurl}}{} % add URL line breaks if available
\IfFileExists{bookmark.sty}{\usepackage{bookmark}}{\usepackage{hyperref}}
\hypersetup{
  pdftitle={Uso do sistema R para análise de dados},
  hidelinks,
  pdfcreator={LaTeX via pandoc}}
\urlstyle{same} % disable monospaced font for URLs
\usepackage{color}
\usepackage{fancyvrb}
\newcommand{\VerbBar}{|}
\newcommand{\VERB}{\Verb[commandchars=\\\{\}]}
\DefineVerbatimEnvironment{Highlighting}{Verbatim}{commandchars=\\\{\}}
% Add ',fontsize=\small' for more characters per line
\usepackage{framed}
\definecolor{shadecolor}{RGB}{248,248,248}
\newenvironment{Shaded}{\begin{snugshade}}{\end{snugshade}}
\newcommand{\AlertTok}[1]{\textcolor[rgb]{0.94,0.16,0.16}{#1}}
\newcommand{\AnnotationTok}[1]{\textcolor[rgb]{0.56,0.35,0.01}{\textbf{\textit{#1}}}}
\newcommand{\AttributeTok}[1]{\textcolor[rgb]{0.77,0.63,0.00}{#1}}
\newcommand{\BaseNTok}[1]{\textcolor[rgb]{0.00,0.00,0.81}{#1}}
\newcommand{\BuiltInTok}[1]{#1}
\newcommand{\CharTok}[1]{\textcolor[rgb]{0.31,0.60,0.02}{#1}}
\newcommand{\CommentTok}[1]{\textcolor[rgb]{0.56,0.35,0.01}{\textit{#1}}}
\newcommand{\CommentVarTok}[1]{\textcolor[rgb]{0.56,0.35,0.01}{\textbf{\textit{#1}}}}
\newcommand{\ConstantTok}[1]{\textcolor[rgb]{0.00,0.00,0.00}{#1}}
\newcommand{\ControlFlowTok}[1]{\textcolor[rgb]{0.13,0.29,0.53}{\textbf{#1}}}
\newcommand{\DataTypeTok}[1]{\textcolor[rgb]{0.13,0.29,0.53}{#1}}
\newcommand{\DecValTok}[1]{\textcolor[rgb]{0.00,0.00,0.81}{#1}}
\newcommand{\DocumentationTok}[1]{\textcolor[rgb]{0.56,0.35,0.01}{\textbf{\textit{#1}}}}
\newcommand{\ErrorTok}[1]{\textcolor[rgb]{0.64,0.00,0.00}{\textbf{#1}}}
\newcommand{\ExtensionTok}[1]{#1}
\newcommand{\FloatTok}[1]{\textcolor[rgb]{0.00,0.00,0.81}{#1}}
\newcommand{\FunctionTok}[1]{\textcolor[rgb]{0.00,0.00,0.00}{#1}}
\newcommand{\ImportTok}[1]{#1}
\newcommand{\InformationTok}[1]{\textcolor[rgb]{0.56,0.35,0.01}{\textbf{\textit{#1}}}}
\newcommand{\KeywordTok}[1]{\textcolor[rgb]{0.13,0.29,0.53}{\textbf{#1}}}
\newcommand{\NormalTok}[1]{#1}
\newcommand{\OperatorTok}[1]{\textcolor[rgb]{0.81,0.36,0.00}{\textbf{#1}}}
\newcommand{\OtherTok}[1]{\textcolor[rgb]{0.56,0.35,0.01}{#1}}
\newcommand{\PreprocessorTok}[1]{\textcolor[rgb]{0.56,0.35,0.01}{\textit{#1}}}
\newcommand{\RegionMarkerTok}[1]{#1}
\newcommand{\SpecialCharTok}[1]{\textcolor[rgb]{0.00,0.00,0.00}{#1}}
\newcommand{\SpecialStringTok}[1]{\textcolor[rgb]{0.31,0.60,0.02}{#1}}
\newcommand{\StringTok}[1]{\textcolor[rgb]{0.31,0.60,0.02}{#1}}
\newcommand{\VariableTok}[1]{\textcolor[rgb]{0.00,0.00,0.00}{#1}}
\newcommand{\VerbatimStringTok}[1]{\textcolor[rgb]{0.31,0.60,0.02}{#1}}
\newcommand{\WarningTok}[1]{\textcolor[rgb]{0.56,0.35,0.01}{\textbf{\textit{#1}}}}
\usepackage{longtable,booktabs}
% Correct order of tables after \paragraph or \subparagraph
\usepackage{etoolbox}
\makeatletter
\patchcmd\longtable{\par}{\if@noskipsec\mbox{}\fi\par}{}{}
\makeatother
% Allow footnotes in longtable head/foot
\IfFileExists{footnotehyper.sty}{\usepackage{footnotehyper}}{\usepackage{footnote}}
\makesavenoteenv{longtable}
\usepackage{graphicx,grffile}
\makeatletter
\def\maxwidth{\ifdim\Gin@nat@width>\linewidth\linewidth\else\Gin@nat@width\fi}
\def\maxheight{\ifdim\Gin@nat@height>\textheight\textheight\else\Gin@nat@height\fi}
\makeatother
% Scale images if necessary, so that they will not overflow the page
% margins by default, and it is still possible to overwrite the defaults
% using explicit options in \includegraphics[width, height, ...]{}
\setkeys{Gin}{width=\maxwidth,height=\maxheight,keepaspectratio}
% Set default figure placement to htbp
\makeatletter
\def\fps@figure{htbp}
\makeatother
\setlength{\emergencystretch}{3em} % prevent overfull lines
\providecommand{\tightlist}{%
  \setlength{\itemsep}{0pt}\setlength{\parskip}{0pt}}
\setcounter{secnumdepth}{5}
\usepackage{booktabs}
\usepackage[]{natbib}
\bibliographystyle{apalike}

\title{Uso do sistema R para análise de dados}
\author{}
\date{\vspace{-2.5em}2020-04-11}

\begin{document}
\maketitle

{
\setcounter{tocdepth}{1}
\tableofcontents
}
\hypertarget{pruxe9-requisitos}{%
\chapter{Pré requisitos}\label{pruxe9-requisitos}}

Material em construção.

Este material, em forma de notas de aula, foi escrito para a disciplina do Mestrado em Engenharia Agrícola, intitulado Uso do sistema R para análise de dados, no primeiro semestre de 2018.
Estas notas de aulas é uma coletânea de apostilas, livros, sites, forum e cursos voltando ao sistema R. Foi utilizado desses materiais sua estrutura didática e rotinas que foram adaptados para o perfil da disciplina.
O material consultado encontra-se referenciado no final de cada capitulo.

\hypertarget{intro}{%
\chapter{R Básico}\label{intro}}

Este primeio capitulo foi baseado no curso on-line \emph{Code School Try R} e \emph{Datacamp}, modificações foram realizadas utilizando outros materiais que se encontram referenciado no final do Capitulo.

Iremos abordar as expressões básicas do R.
Começaremos simples, com \textbf{números}, \textbf{strings} e valores \textbf{true/false}. Em seguida, mostraremos como armazenar esses valores em variáveis e como transmiti-los as funções. Como obter ajuda sobre as funções e no final vamos carregar um arquivo

\hypertarget{expressuxf5es}{%
\section{Expressões}\label{expressuxf5es}}

Vamos tentar matemática simples. Digite o comando abaixo e aperte enter

\begin{Shaded}
\begin{Highlighting}[]
\DecValTok{2}\OperatorTok{+}\DecValTok{8}
\end{Highlighting}
\end{Shaded}

\begin{verbatim}
## [1] 10
\end{verbatim}

Note que é impresso o resultado, 10.

Digite a frase ``Engenharia Agrícola''

\begin{Shaded}
\begin{Highlighting}[]
\StringTok{"Engenharia Agrícola"}
\end{Highlighting}
\end{Shaded}

\begin{verbatim}
## [1] "Engenharia Agrícola"
\end{verbatim}

Agora tente multiplicar 6 vezes 5 (* é o operador de multiplicação).

\begin{Shaded}
\begin{Highlighting}[]
\DecValTok{6}\OperatorTok{*}\DecValTok{5}
\end{Highlighting}
\end{Shaded}

\begin{verbatim}
## [1] 30
\end{verbatim}

\hypertarget{valores-booleanos}{%
\section{Valores Booleanos}\label{valores-booleanos}}

Algumas expressões retornam um ``valor lógico'': TRUE ou FALSE e/ou ``booleanos''.
Vamos tentar digitar uma expressões que nos dá um valor lógico:

\begin{Shaded}
\begin{Highlighting}[]
\DecValTok{7}\OperatorTok{<}\DecValTok{12}
\end{Highlighting}
\end{Shaded}

\begin{verbatim}
## [1] TRUE
\end{verbatim}

E outro valor lógico (sinal duplo de igualdade)

\begin{Shaded}
\begin{Highlighting}[]
\DecValTok{6}\OperatorTok{+}\DecValTok{5}\OperatorTok{==}\DecValTok{10}
\end{Highlighting}
\end{Shaded}

\begin{verbatim}
## [1] FALSE
\end{verbatim}

\textbf{T} e \textbf{F} são taquigrafia para TRUE e FALSE. Tente isso:

\begin{Shaded}
\begin{Highlighting}[]
\NormalTok{F}\OperatorTok{==}\OtherTok{FALSE}
\end{Highlighting}
\end{Shaded}

\begin{verbatim}
## [1] TRUE
\end{verbatim}

\hypertarget{variuxe1veis}{%
\section{Variáveis}\label{variuxe1veis}}

Você pode armazenar valores em uma variável para usar mais tarde.
Digite \textbf{x \textless- 28} para armazenar um valor em \textbf{x}.

\begin{Shaded}
\begin{Highlighting}[]
\NormalTok{x<-}\DecValTok{28}
\end{Highlighting}
\end{Shaded}

Tende dividr \textbf{x} por \textbf{4}( \textbf{/} é o operador da divisão).

\begin{Shaded}
\begin{Highlighting}[]
\NormalTok{x}\OperatorTok{/}\DecValTok{4}
\end{Highlighting}
\end{Shaded}

\begin{verbatim}
## [1] 7
\end{verbatim}

Você pode retribuir qualquer valor a uma variável a qualquer momento.
Tente atribuir ``Engenharia Agrícola''em x.

\begin{Shaded}
\begin{Highlighting}[]
\NormalTok{x <-}\StringTok{ "Engenharia Agrícola"}
\end{Highlighting}
\end{Shaded}

Tente imprimir o valor atual de x.

\begin{Shaded}
\begin{Highlighting}[]
\NormalTok{x}
\end{Highlighting}
\end{Shaded}

\begin{verbatim}
## [1] "Engenharia Agrícola"
\end{verbatim}

\hypertarget{funuxe7uxf5es}{%
\section{Funções}\label{funuxe7uxf5es}}

Você pode chamar uma \textbf{função} digitando seu nome, seguido de um ou mais argumentos para essa função entre parênteses.

Vamos tentar usar a função \texttt{sum()}, para adicionar alguns números. Entrar:

\begin{Shaded}
\begin{Highlighting}[]
\KeywordTok{sum}\NormalTok{ (}\DecValTok{2}\NormalTok{, }\DecValTok{4}\NormalTok{, }\DecValTok{6}\NormalTok{)}
\end{Highlighting}
\end{Shaded}

\begin{verbatim}
## [1] 12
\end{verbatim}

Alguns argumentos têm nomes. Por exemplo, para repetir um valor 3 vezes, você chamaria a função \texttt{rep} e forneceria seu argumento \textbf{times}:

\begin{Shaded}
\begin{Highlighting}[]
\KeywordTok{rep}\NormalTok{(}\StringTok{"Engenharia Agrícola"}\NormalTok{, }\DataTypeTok{times=}\DecValTok{3}\NormalTok{)}
\end{Highlighting}
\end{Shaded}

\begin{verbatim}
## [1] "Engenharia Agrícola" "Engenharia Agrícola" "Engenharia Agrícola"
\end{verbatim}

Tente chamar a função \texttt{sqrt} para obter a raiz quadrada 16.

\begin{Shaded}
\begin{Highlighting}[]
\KeywordTok{sqrt}\NormalTok{(}\DecValTok{16}\NormalTok{)}
\end{Highlighting}
\end{Shaded}

\begin{verbatim}
## [1] 4
\end{verbatim}

\hypertarget{ajuda}{%
\section{Ajuda}\label{ajuda}}

A função \texttt{help\ ()} traz ajuda para a função desejada. Tente exibir ajuda para a função \texttt{mean}:

\begin{Shaded}
\begin{Highlighting}[]
\KeywordTok{help}\NormalTok{ (mean)}
\end{Highlighting}
\end{Shaded}

A função \texttt{example\ ()} traz exemplos de usos. Tente exibir exemplos para a função \texttt{min}:

\begin{Shaded}
\begin{Highlighting}[]
\KeywordTok{example}\NormalTok{(min)}
\end{Highlighting}
\end{Shaded}

\begin{verbatim}
## 
## min> require(stats); require(graphics)
## 
## min>  min(5:1, pi) #-> one number
## [1] 1
## 
## min> pmin(5:1, pi) #->  5  numbers
## [1] 3.141593 3.141593 3.000000 2.000000 1.000000
## 
## min> x <- sort(rnorm(100));  cH <- 1.35
## 
## min> pmin(cH, quantile(x)) # no names
## [1] -2.1499421 -0.4059580  0.2592357  0.6704064  1.3500000
## 
## min> pmin(quantile(x), cH) # has names
##         0%        25%        50%        75%       100% 
## -2.1499421 -0.4059580  0.2592357  0.6704064  1.3500000 
## 
## min> plot(x, pmin(cH, pmax(-cH, x)), type = "b", main =  "Huber's function")
\end{verbatim}

\includegraphics{TudodoR_files/figure-latex/unnamed-chunk-14-1.pdf}

\begin{verbatim}
## 
## min> cut01 <- function(x) pmax(pmin(x, 1), 0)
## 
## min> curve(      x^2 - 1/4, -1.4, 1.5, col = 2)
\end{verbatim}

\includegraphics{TudodoR_files/figure-latex/unnamed-chunk-14-2.pdf}

\begin{verbatim}
## 
## min> curve(cut01(x^2 - 1/4), col = "blue", add = TRUE, n = 500)
## 
## min> ## pmax(), pmin() preserve attributes of *first* argument
## min> D <- diag(x = (3:1)/4) ; n0 <- numeric()
## 
## min> stopifnot(identical(D,  cut01(D) ),
## min+           identical(n0, cut01(n0)),
## min+           identical(n0, cut01(NULL)),
## min+           identical(n0, pmax(3:1, n0, 2)),
## min+           identical(n0, pmax(n0, 4)))
\end{verbatim}

\hypertarget{referuxeancia}{%
\section{Referência}\label{referuxeancia}}

MELO, M. P.; PETERNELI, L. A. \textbf{Conhecendo o R: Um visão mais que estatística}. Viçosa, MG: UFV, 2013. 222p.

\textbf{Prof.~Paulo Justiniando Ribeiro} \textgreater{}\url{http://www.leg.ufpr.br/~paulojus/}\textless{}

\textbf{Prof.~Adriano Azevedo Filho} \textgreater{}\url{http://rpubs.com/adriano/esalq2012inicial}\textless{}

\textbf{Prof.~Fernando de Pol Mayer} \textgreater{}\url{https://fernandomayer.github.io/ce083-2016-2/}\textless{}

\textbf{Site Interativo Datacamp} \textgreater{}\url{https://www.datacamp.com/}\textless{}

\hypertarget{estruturas-de-dados}{%
\chapter{Estruturas de Dados}\label{estruturas-de-dados}}

Este segundo Capitulo foi baseado no curso on-line \emph{Code School Try R} e no livro \href{https://www.editoraufv.com.br/produto/conhecendo-o-r-uma-visao-mais-que-estatistica/1109294}{\textbf{Conhecendo o R: Um visão mais que estatística}}, modificações foram realizadas utilizando outros materiais que se encontram referenciado no final do Capitulo.

\hypertarget{vetor}{%
\section{Vetor}\label{vetor}}

Um vetor é simplesmente uma lista de valores.
A maneira mais simples de usar um vetor é usando o comando \texttt{c()}, que concatena elementos num mesmo objeto.
Exemplo

\begin{Shaded}
\begin{Highlighting}[]
\NormalTok{x<-}\StringTok{ }\KeywordTok{c}\NormalTok{(}\DecValTok{2}\NormalTok{,}\DecValTok{3}\NormalTok{,}\DecValTok{5}\NormalTok{,}\DecValTok{7}\NormalTok{,}\DecValTok{11}\NormalTok{) }
\NormalTok{x}
\end{Highlighting}
\end{Shaded}

\begin{verbatim}
## [1]  2  3  5  7 11
\end{verbatim}

Os argumentos de \texttt{c()} podem ser tanto elementos únicos quanto outros objetos. Adicione três números no \textbf{vetor x}

\begin{Shaded}
\begin{Highlighting}[]
\NormalTok{y<-}\StringTok{ }\KeywordTok{c}\NormalTok{(x,}\DecValTok{13}\NormalTok{,}\DecValTok{17}\NormalTok{,}\DecValTok{19}\NormalTok{)}
\NormalTok{y}
\end{Highlighting}
\end{Shaded}

\begin{verbatim}
## [1]  2  3  5  7 11 13 17 19
\end{verbatim}

\hypertarget{vetores-de-sequuxeancia}{%
\subsection{Vetores de Sequência}\label{vetores-de-sequuxeancia}}

Se você precisa de um vetor com uma sequência de números, você pode cria-lo com a notação \emph{start:end}. Vamos fazer um vetor com valores de 1 a 7:

\begin{Shaded}
\begin{Highlighting}[]
\DecValTok{1}\OperatorTok{:}\DecValTok{7}
\end{Highlighting}
\end{Shaded}

\begin{verbatim}
## [1] 1 2 3 4 5 6 7
\end{verbatim}

Uma maneira mais versátil de fazer sequências é chamar a função \texttt{seq}. Vamos fazer o mesmo com \texttt{seq\ ()} :

\begin{Shaded}
\begin{Highlighting}[]
\KeywordTok{seq}\NormalTok{(}\DecValTok{1}\OperatorTok{:}\DecValTok{7}\NormalTok{)}
\end{Highlighting}
\end{Shaded}

\begin{verbatim}
## [1] 1 2 3 4 5 6 7
\end{verbatim}

A função \texttt{seq} também permite que você use incrementos diferentes de 1. Experimente com etapas de 0.5.

\begin{Shaded}
\begin{Highlighting}[]
\KeywordTok{seq}\NormalTok{(}\DecValTok{1}\NormalTok{,}\DecValTok{7}\NormalTok{,}\FloatTok{0.5}\NormalTok{)}
\end{Highlighting}
\end{Shaded}

\begin{verbatim}
##  [1] 1.0 1.5 2.0 2.5 3.0 3.5 4.0 4.5 5.0 5.5 6.0 6.5 7.0
\end{verbatim}

\begin{Shaded}
\begin{Highlighting}[]
\KeywordTok{seq}\NormalTok{(}\DecValTok{7}\NormalTok{,}\DecValTok{1}\NormalTok{,}\OperatorTok{-}\FloatTok{0.5}\NormalTok{) }
\end{Highlighting}
\end{Shaded}

\begin{verbatim}
##  [1] 7.0 6.5 6.0 5.5 5.0 4.5 4.0 3.5 3.0 2.5 2.0 1.5 1.0
\end{verbatim}

Todo objeto possui atributos intrínsecos: tipo e tamanho. Com relação ao tipo ele pode ser: \textbf{numérico}, \textbf{caractere}, \textbf{complexo} e \textbf{lógico}. Existem outros tipos, como por exemplo, funções ou expressões, porém esses não representam dados.
As funções \texttt{mode()} e \texttt{length()} mostram o tipo e tamanho de um objeto, respectivamente.

\begin{Shaded}
\begin{Highlighting}[]
\NormalTok{z<-}\KeywordTok{c}\NormalTok{(}\DecValTok{1}\NormalTok{,}\DecValTok{3}\NormalTok{,}\DecValTok{5}\NormalTok{,}\DecValTok{7}\NormalTok{,}\DecValTok{11}\NormalTok{) }
\KeywordTok{mode}\NormalTok{ (z)}
\end{Highlighting}
\end{Shaded}

\begin{verbatim}
## [1] "numeric"
\end{verbatim}

\begin{Shaded}
\begin{Highlighting}[]
\KeywordTok{length}\NormalTok{(z)}
\end{Highlighting}
\end{Shaded}

\begin{verbatim}
## [1] 5
\end{verbatim}

\begin{Shaded}
\begin{Highlighting}[]
\NormalTok{a <-}\StringTok{ "Angela"}
\NormalTok{b<-}\OtherTok{TRUE}\NormalTok{; }
\NormalTok{c<-8i }\CommentTok{#objetos com tipos diferentes}
\KeywordTok{mode}\NormalTok{(a); }
\end{Highlighting}
\end{Shaded}

\begin{verbatim}
## [1] "character"
\end{verbatim}

\begin{Shaded}
\begin{Highlighting}[]
\KeywordTok{mode}\NormalTok{(b); }
\end{Highlighting}
\end{Shaded}

\begin{verbatim}
## [1] "logical"
\end{verbatim}

\begin{Shaded}
\begin{Highlighting}[]
\KeywordTok{mode}\NormalTok{(c) }\CommentTok{#exibe os atributos "tipo" dos objetos }
\end{Highlighting}
\end{Shaded}

\begin{verbatim}
## [1] "complex"
\end{verbatim}

Se o vetor é muito longo e não ``cabe'' em uma linha o R vai usar as linhas seguintes para continuar imprimindo o vetor.

\begin{Shaded}
\begin{Highlighting}[]
\NormalTok{longo<-}\DecValTok{100}\OperatorTok{:}\DecValTok{50} \CommentTok{#sequência decrescente de 100 a 50}
\NormalTok{longo }\CommentTok{#exibe o conteúdo do objeto }
\end{Highlighting}
\end{Shaded}

\begin{verbatim}
##  [1] 100  99  98  97  96  95  94  93  92  91  90  89  88  87  86  85  84  83  82
## [20]  81  80  79  78  77  76  75  74  73  72  71  70  69  68  67  66  65  64  63
## [39]  62  61  60  59  58  57  56  55  54  53  52  51  50
\end{verbatim}

Os números entre colchetes não fazem parte do objeto e indica a posição do vetor naquele ponto. Pode-se ver que {[}1{]} indica que o primeiro elemento do vetor estão naquela linha, {[}17{]} indica que a linha seguinte começa pelo décimo setimo elemento do vetor e
assim por diante.

Você pode recuperar um valor individual dentro de um vetor fornecendo seu índice numérico entre colchetes. Tente obter o valor 18:

\begin{Shaded}
\begin{Highlighting}[]
\NormalTok{longo[}\DecValTok{18}\NormalTok{]}
\end{Highlighting}
\end{Shaded}

\begin{verbatim}
## [1] 83
\end{verbatim}

Muitas línguagem de programação iniciam índices de matriz em 0, mas os índices vetoriais de R começam em 1. Obtenha o primeiro valor digitando:

\begin{Shaded}
\begin{Highlighting}[]
\NormalTok{longo[}\DecValTok{1}\NormalTok{]}
\end{Highlighting}
\end{Shaded}

\begin{verbatim}
## [1] 100
\end{verbatim}

Você pode atribuir novos valores dentro de um vetor existente. Tente mudar o terceiro valor \textbf{28}:

\begin{Shaded}
\begin{Highlighting}[]
\NormalTok{longo [}\DecValTok{3}\NormalTok{] <-}\StringTok{ }\DecValTok{28}
\end{Highlighting}
\end{Shaded}

Se você adicionar novos valores ao final, o vetor aumentará para acomodá-los. Vamos adicionar um valor no final

\begin{Shaded}
\begin{Highlighting}[]
\NormalTok{longo[}\DecValTok{101}\NormalTok{] <-}\StringTok{ }\DecValTok{83}
\end{Highlighting}
\end{Shaded}

Você pode usar um vetor entre os colchetes para acessar vários valores. Tente obter a primeira e a terceira palavras

\begin{Shaded}
\begin{Highlighting}[]
\NormalTok{longo[}\KeywordTok{c}\NormalTok{(}\DecValTok{1}\NormalTok{,}\DecValTok{3}\NormalTok{)]}
\end{Highlighting}
\end{Shaded}

\begin{verbatim}
## [1] 100  28
\end{verbatim}

Isso significa que você pode recuperar intervalos de valores. Obter a segunda a quarta palavras:

\begin{Shaded}
\begin{Highlighting}[]
\NormalTok{longo[}\DecValTok{2}\OperatorTok{:}\DecValTok{4}\NormalTok{]}
\end{Highlighting}
\end{Shaded}

\begin{verbatim}
## [1] 99 28 97
\end{verbatim}

Você também pode definir intervalos de valores; apenas forneça os valores em um vetor. Adicione valores 5 a 7:

\begin{Shaded}
\begin{Highlighting}[]
\NormalTok{longo[}\DecValTok{5}\OperatorTok{:}\DecValTok{7}\NormalTok{] <-}\StringTok{ }\KeywordTok{c}\NormalTok{(}\DecValTok{42}\NormalTok{,}\DecValTok{52}\NormalTok{,}\DecValTok{75}\NormalTok{)}
\end{Highlighting}
\end{Shaded}

Agora tente acessar o oitavo valor do vetor:

\begin{Shaded}
\begin{Highlighting}[]
\NormalTok{longo[}\DecValTok{8}\NormalTok{]}
\end{Highlighting}
\end{Shaded}

\begin{verbatim}
## [1] 93
\end{verbatim}

\hypertarget{nomes-de-vetores}{%
\subsection{Nomes de vetores}\label{nomes-de-vetores}}

Para esse desafio, criaremos um vetor de 3 itens e armazená-lo na variável solo.
Você pode atribuir nomes aos elementos de um vetor passando um segundo vetor preenchido com os nomes com a função \texttt{names\ ()}, assim:

\begin{Shaded}
\begin{Highlighting}[]
\NormalTok{solo <-}\StringTok{ }\DecValTok{1}\OperatorTok{:}\DecValTok{3}
\KeywordTok{names}\NormalTok{(solo) <-}\StringTok{ }\KeywordTok{c}\NormalTok{(}\StringTok{"Argila"}\NormalTok{, }\StringTok{"Areia"}\NormalTok{,}\StringTok{"Silte"}\NormalTok{ )}
\NormalTok{solo}
\end{Highlighting}
\end{Shaded}

\begin{verbatim}
## Argila  Areia  Silte 
##      1      2      3
\end{verbatim}

Agora, defina o valor atual para o \emph{silte} para um valor diferente usando o nome em vez da posição.

\begin{Shaded}
\begin{Highlighting}[]
\NormalTok{solo[}\StringTok{"Silte"}\NormalTok{]<-}\DecValTok{20}
\end{Highlighting}
\end{Shaded}

\hypertarget{plotando-um-vetor}{%
\subsection{Plotando um vetor}\label{plotando-um-vetor}}

A função \texttt{barplot\ ()} desenha um gráfico de barras com os valores de um vetor. Vamos criar um novo vetor para você e armazená-lo na variável chuva.

Agora, tente passar o vetor para a função \texttt{barplot}:

\begin{Shaded}
\begin{Highlighting}[]
\NormalTok{chuva <-}\StringTok{ }\KeywordTok{c}\NormalTok{(}\DecValTok{20}\NormalTok{,}\DecValTok{50}\NormalTok{,}\DecValTok{85}\NormalTok{)}
\KeywordTok{barplot}\NormalTok{(chuva)}
\end{Highlighting}
\end{Shaded}

\includegraphics{TudodoR_files/figure-latex/unnamed-chunk-32-1.pdf}

Se você atribuir nomes aos valores do vetor, o R usará esses nomes como rótulos no gráfico da barra. Vamos usar a função \texttt{names\ ()} novamente:

\begin{Shaded}
\begin{Highlighting}[]
\KeywordTok{names}\NormalTok{(chuva)<-}\StringTok{ }\KeywordTok{c}\NormalTok{(}\StringTok{"Rondonópolis", "}\NormalTok{Maringá}\StringTok{", "}\NormalTok{Cruzeiro do Sul}\StringTok{")}
\end{Highlighting}
\end{Shaded}

Agora, se você digitar \texttt{barplot\ (chuva)} com o vetor novamente, você verá os rótulos:

\begin{Shaded}
\begin{Highlighting}[]
\KeywordTok{barplot}\NormalTok{(chuva)}
\end{Highlighting}
\end{Shaded}

\includegraphics{TudodoR_files/figure-latex/unnamed-chunk-34-1.pdf}

Agora, tente chamar \texttt{barplot} em um vetor de números inteiros que variam de 1 a 100:

\begin{Shaded}
\begin{Highlighting}[]
\KeywordTok{barplot}\NormalTok{(}\DecValTok{1}\OperatorTok{:}\DecValTok{100}\NormalTok{)}
\end{Highlighting}
\end{Shaded}

\includegraphics{TudodoR_files/figure-latex/unnamed-chunk-35-1.pdf}

\hypertarget{operauxe7uxf5es-matemuxe1ticas}{%
\subsection{Operações matemáticas}\label{operauxe7uxf5es-matemuxe1ticas}}

A maioria das operações aritméticas funcionam tão bem em vetores quanto em valores únicos. Vamos fazer outro vetor de exemplo para você trabalhar e armazená-lo a variável \textbf{a}

Se você adicionar um escalar (um único valor) a um vetor, o escalar será adicionado a cada valor no vetor, retornando um novo vetor com os resultados. Tente adicionar 1 a cada elemento em nosso vetor:

\begin{Shaded}
\begin{Highlighting}[]
\NormalTok{a <-}\StringTok{ }\KeywordTok{c}\NormalTok{(}\DecValTok{1}\NormalTok{, }\DecValTok{2}\NormalTok{, }\DecValTok{3}\NormalTok{)}
\NormalTok{a }\OperatorTok{+}\StringTok{ }\DecValTok{1}
\end{Highlighting}
\end{Shaded}

\begin{verbatim}
## [1] 2 3 4
\end{verbatim}

O mesmo se aplica na divisão, multiplicação ou qualquer outra aritmética básica. Tente dividir nosso vetor por 2:

\begin{Shaded}
\begin{Highlighting}[]
\NormalTok{a }\OperatorTok{/}\StringTok{ }\DecValTok{2}
\end{Highlighting}
\end{Shaded}

\begin{verbatim}
## [1] 0.5 1.0 1.5
\end{verbatim}

Agora, tente multiplicar nosso vetor por 2:

\begin{Shaded}
\begin{Highlighting}[]
\NormalTok{a}\OperatorTok{*}\DecValTok{2}
\end{Highlighting}
\end{Shaded}

\begin{verbatim}
## [1] 2 4 6
\end{verbatim}

Se você adicionar dois vetores, R irá tirar cada valor de cada vetor e adicioná-los. Vamos fazer um segundo vetor para você experimentar e armazená-lo na variável \textbf{b}

Tente adicioná-lo ao vetor \textbf{a}:

\begin{Shaded}
\begin{Highlighting}[]
\NormalTok{b <-}\StringTok{ }\KeywordTok{c}\NormalTok{(}\DecValTok{4}\NormalTok{,}\DecValTok{5}\NormalTok{,}\DecValTok{6}\NormalTok{)}
\NormalTok{a}\OperatorTok{+}\NormalTok{b}
\end{Highlighting}
\end{Shaded}

\begin{verbatim}
## [1] 5 7 9
\end{verbatim}

Agora tente subtrair b de a:

\begin{Shaded}
\begin{Highlighting}[]
\NormalTok{a}\OperatorTok{-}\NormalTok{b}
\end{Highlighting}
\end{Shaded}

\begin{verbatim}
## [1] -3 -3 -3
\end{verbatim}

Você também pode tirar dois vetores e comparar cada item. Veja quais valores nos vetores são iguais aos de um segundo vetor

\begin{Shaded}
\begin{Highlighting}[]
\NormalTok{a }\OperatorTok{==}\StringTok{ }\KeywordTok{c}\NormalTok{(}\DecValTok{1}\NormalTok{, }\DecValTok{99}\NormalTok{, }\DecValTok{3}\NormalTok{)}
\end{Highlighting}
\end{Shaded}

\begin{verbatim}
## [1]  TRUE FALSE  TRUE
\end{verbatim}

Observe que R não testou se os vetores inteiros eram iguais; verificou cada valor no vetor a contra o valor no mesmo índice no nosso novo vetor.

Verifique se cada valor nos vetores são menores que o valor correspondente em outro vetor:

\begin{Shaded}
\begin{Highlighting}[]
\NormalTok{a }\OperatorTok{<}\StringTok{ }\KeywordTok{c}\NormalTok{(}\DecValTok{1}\NormalTok{, }\DecValTok{99}\NormalTok{, }\DecValTok{3}\NormalTok{)}
\end{Highlighting}
\end{Shaded}

\begin{verbatim}
## [1] FALSE  TRUE FALSE
\end{verbatim}

Funções que normalmente funcionam com escalares também podem operar em cada elemento de um vetor. Tente obter o seno de cada valor em nosso vetor:

\begin{Shaded}
\begin{Highlighting}[]
\KeywordTok{sin}\NormalTok{(a)}
\end{Highlighting}
\end{Shaded}

\begin{verbatim}
## [1] 0.8414710 0.9092974 0.1411200
\end{verbatim}

Agora tente obter as raízes quadradas com a função \texttt{sqrt}:

\begin{Shaded}
\begin{Highlighting}[]
\KeywordTok{sqrt}\NormalTok{(a)}
\end{Highlighting}
\end{Shaded}

\begin{verbatim}
## [1] 1.000000 1.414214 1.732051
\end{verbatim}

\hypertarget{parcelas-de-dispersuxe3o}{%
\subsection{Parcelas de dispersão}\label{parcelas-de-dispersuxe3o}}

A função \texttt{plot} leva dois vetores, um para valores X e um para valores Y, e desenha um gráfico deles.

Vamos desenhar um gráfico que mostra a relação de números e seus senos.

Primeiro, precisaremos de alguns dados de amostra. Criaremos um vetor com alguns valores fracionários entre 0 e 20, e armazenó-lo na variável x. E na variável y um segundo vetor com os senos de x:

\begin{Shaded}
\begin{Highlighting}[]
\NormalTok{x <-}\StringTok{ }\KeywordTok{seq}\NormalTok{(}\DecValTok{1}\NormalTok{, }\DecValTok{20}\NormalTok{, }\FloatTok{0.1}\NormalTok{)}
\NormalTok{y<-}\KeywordTok{sin}\NormalTok{(x)}
\end{Highlighting}
\end{Shaded}

Em seguida, basta chamar a função \texttt{plot} com seus dois vetores:

\begin{Shaded}
\begin{Highlighting}[]
\KeywordTok{plot}\NormalTok{(x, y)}
\end{Highlighting}
\end{Shaded}

\includegraphics{TudodoR_files/figure-latex/unnamed-chunk-46-1.pdf}

Observa=se sobre o gráfico que os valores do primeiro argumento \textbf{(x)} são usados para o eixo horizontal, e os valores do segundo \textbf{(y)} para o vertical.

Vamos criar um vetor com alguns valores negativos e positivos para você e armazenó-lo na variável \textbf{valores}.

Também criaremos um segundo vetor com os valores absolutos do primeiro e armazenó-lo na variável \textbf{absoluto}.

Tente traçar os vetores, com os \textbf{valores} no eixo horizontal e no eixo vertical os absoluto.

\begin{Shaded}
\begin{Highlighting}[]
\NormalTok{valores <-}\StringTok{ }\DecValTok{-10}\OperatorTok{:}\DecValTok{10}
\NormalTok{absoluto<-}\StringTok{ }\KeywordTok{abs}\NormalTok{(valores)}
\KeywordTok{plot}\NormalTok{(valores, absoluto)}
\end{Highlighting}
\end{Shaded}

\includegraphics{TudodoR_files/figure-latex/unnamed-chunk-47-1.pdf}

\hypertarget{valores-faltantes}{%
\subsection{Valores Faltantes}\label{valores-faltantes}}

As vezes, ao trabalhar com dados de amostra, um determinado valor não está disponível. Mas não é uma boa idéia apenas tirar esses valores. R tem um valor que indica explicitamente uma amostra não estava disponível: \textbf{NA}. Muitas funções que funcionam com vetores tratam esse valor especialmente.

Vamos criar um vetor para você com uma amostra ausente e armazenó-lo na variével \textbf{a}.

Tente obter a soma de seus valores e veja qual é o resultado:

\begin{Shaded}
\begin{Highlighting}[]
\NormalTok{a <-}\StringTok{ }\KeywordTok{c}\NormalTok{(}\DecValTok{1}\NormalTok{, }\DecValTok{3}\NormalTok{, }\OtherTok{NA}\NormalTok{, }\DecValTok{7}\NormalTok{, }\DecValTok{9}\NormalTok{)}
\KeywordTok{sum}\NormalTok{(a)}
\end{Highlighting}
\end{Shaded}

\begin{verbatim}
## [1] NA
\end{verbatim}

A soma é considerada \emph{``não disponível''} por padrão porque um dos valores do vetor foi \textbf{NA}.

Lembre-se desse comando para mostrar ajuda para uma função. Apresente a ajuda para a função \texttt{sum}:

\begin{Shaded}
\begin{Highlighting}[]
\KeywordTok{help}\NormalTok{(sum)}
\end{Highlighting}
\end{Shaded}

Como você vê na documentação, \texttt{sum} pode tomar um argumento opcional \textbf{na.rm},. ? configurado \textbf{FALSE} por padrão, mas se você configurá-lo com \textbf{TRUE}, todos os argumentos \textbf{NA} serão removidos do vetor antes do cálculo ser executado.

Tente rondar \texttt{sum} novamente, com o \textbf{na.rm} conjunto para \textbf{TRUE}:

\begin{Shaded}
\begin{Highlighting}[]
\KeywordTok{sum}\NormalTok{(a, }\DataTypeTok{na.rm =}\NormalTok{ T)}
\end{Highlighting}
\end{Shaded}

\begin{verbatim}
## [1] 20
\end{verbatim}

\hypertarget{matrizes}{%
\section{Matrizes}\label{matrizes}}

Há varias formas de criar uma matriz. O comando \texttt{matriz()} recebe um vetor como argumento e o transfoma em uma matrix de acordo com as dimensões.
Vamos fazer uma matriz de 3 linhas de altura por 4 colunas de largura, com todos os seus campos definidos 0.

\begin{Shaded}
\begin{Highlighting}[]
\KeywordTok{matrix}\NormalTok{(}\DecValTok{0}\NormalTok{,}\DecValTok{3}\NormalTok{,}\DecValTok{4}\NormalTok{)}
\end{Highlighting}
\end{Shaded}

\begin{verbatim}
##      [,1] [,2] [,3] [,4]
## [1,]    0    0    0    0
## [2,]    0    0    0    0
## [3,]    0    0    0    0
\end{verbatim}

Você também pode usar um vetor para inicializar o valor de uma matriz. Para preencher uma matriz de 3x4, você precisará de um vetor de 12 itens.

\begin{Shaded}
\begin{Highlighting}[]
\NormalTok{a <-}\StringTok{ }\NormalTok{(}\DecValTok{1}\OperatorTok{:}\DecValTok{12}\NormalTok{)}

\KeywordTok{print}\NormalTok{ (a)}
\end{Highlighting}
\end{Shaded}

\begin{verbatim}
##  [1]  1  2  3  4  5  6  7  8  9 10 11 12
\end{verbatim}

Agora chame matrix com o vetor, o número de linhas e o número de colunas:

\begin{Shaded}
\begin{Highlighting}[]
\KeywordTok{matrix}\NormalTok{ (a,}\CommentTok{# chama o vetor}
        \DecValTok{3}\NormalTok{,}\CommentTok{# linha}
        \DecValTok{4}\NormalTok{) }\CommentTok{#coluna}
\end{Highlighting}
\end{Shaded}

\begin{verbatim}
##      [,1] [,2] [,3] [,4]
## [1,]    1    4    7   10
## [2,]    2    5    8   11
## [3,]    3    6    9   12
\end{verbatim}

Você também pode usar um vetor para inicializar o valor de uma matriz. Para preencher uma matriz 3x4, você precisará de um vetor de 12 itens. Nós vamos fazer isso para você agora:

\begin{Shaded}
\begin{Highlighting}[]
\NormalTok{a <-}\DecValTok{1}\OperatorTok{:}\DecValTok{12}
\NormalTok{a}
\end{Highlighting}
\end{Shaded}

\begin{verbatim}
##  [1]  1  2  3  4  5  6  7  8  9 10 11 12
\end{verbatim}

Agora chame \textbf{matrix} com o vetor, o número de linhas e o número de colunas:

\begin{Shaded}
\begin{Highlighting}[]
\KeywordTok{matrix}\NormalTok{ (a,}\DecValTok{3}\NormalTok{,}\DecValTok{4}\NormalTok{)}
\end{Highlighting}
\end{Shaded}

\begin{verbatim}
##      [,1] [,2] [,3] [,4]
## [1,]    1    4    7   10
## [2,]    2    5    8   11
## [3,]    3    6    9   12
\end{verbatim}

\hypertarget{outras-formas}{%
\subsection{Outras formas}\label{outras-formas}}

\begin{Shaded}
\begin{Highlighting}[]
\KeywordTok{matrix}\NormalTok{ (a, }\DecValTok{3}\NormalTok{)}
\end{Highlighting}
\end{Shaded}

\begin{verbatim}
##      [,1] [,2] [,3] [,4]
## [1,]    1    4    7   10
## [2,]    2    5    8   11
## [3,]    3    6    9   12
\end{verbatim}

Note que as matrizes são preenchidas ao longo das colunas. Para que a matriz seja preenchida por linhas deve-se alterar o argumento \textbf{byrow}, que, por padrão, está definido como \textbf{FALSE}, passe para \textbf{TRUE}

\begin{Shaded}
\begin{Highlighting}[]
\KeywordTok{matrix}\NormalTok{(a,}\DecValTok{3}\NormalTok{, }\DataTypeTok{byrow=}\NormalTok{T)}
\end{Highlighting}
\end{Shaded}

\begin{verbatim}
##      [,1] [,2] [,3] [,4]
## [1,]    1    2    3    4
## [2,]    5    6    7    8
## [3,]    9   10   11   12
\end{verbatim}

Os valores do vetor são copiados para a nova matriz, um por um. Você também pode reformular o próprio \textbf{vetor} em uma \textbf{matriz}. Crie um vetor de 8 itens:

\begin{Shaded}
\begin{Highlighting}[]
\NormalTok{foliar <-}\StringTok{ }\DecValTok{1}\OperatorTok{:}\DecValTok{8}
\end{Highlighting}
\end{Shaded}

A função \texttt{dim} define as \textbf{dim}ensões para uma matriz. Ele aceita um vetor com o número de linhas e o n?mero de colunas a serem atribu?das.
Atribua novas dimens?es para \textbf{foliar} passando um vetor especificando 2 linhas e 4 colunas ( c(2, 4)):

\begin{Shaded}
\begin{Highlighting}[]
\KeywordTok{dim}\NormalTok{(foliar) <-}\StringTok{ }\KeywordTok{c}\NormalTok{(}\DecValTok{2}\NormalTok{,}\DecValTok{4}\NormalTok{)}
\end{Highlighting}
\end{Shaded}

O vetor não é mais unidimensional. Foi convertido, no local, para uma matriz.
Agora, use a função \textbf{matrix} para criar uma matriz \textbf{5x5}, com seus campos inicializados para qualquer valor que você desejar.

\begin{Shaded}
\begin{Highlighting}[]
\KeywordTok{matrix}\NormalTok{ (}\DecValTok{2}\NormalTok{,}\DecValTok{5}\NormalTok{,}\DecValTok{5}\NormalTok{)}
\end{Highlighting}
\end{Shaded}

\begin{verbatim}
##      [,1] [,2] [,3] [,4] [,5]
## [1,]    2    2    2    2    2
## [2,]    2    2    2    2    2
## [3,]    2    2    2    2    2
## [4,]    2    2    2    2    2
## [5,]    2    2    2    2    2
\end{verbatim}

\hypertarget{acesso-a-matriz}{%
\subsection{Acesso a Matriz}\label{acesso-a-matriz}}

Obter valores de matrizes não é diferente de vetores; você só precisa fornecer dois índices em vez de um. Abra a matriz foliar:

\begin{Shaded}
\begin{Highlighting}[]
\KeywordTok{print}\NormalTok{ (foliar)}
\end{Highlighting}
\end{Shaded}

\begin{verbatim}
##      [,1] [,2] [,3] [,4]
## [1,]    1    3    5    7
## [2,]    2    4    6    8
\end{verbatim}

Tente obter o valor da segunda linha na terceira coluna da matriz foliar;

\begin{Shaded}
\begin{Highlighting}[]
\NormalTok{foliar[}\DecValTok{2}\NormalTok{,}\DecValTok{3}\NormalTok{]}
\end{Highlighting}
\end{Shaded}

\begin{verbatim}
## [1] 6
\end{verbatim}

O valor da primeira linha da quarta coluna

\begin{Shaded}
\begin{Highlighting}[]
\NormalTok{foliar[}\DecValTok{1}\NormalTok{,}\DecValTok{4}\NormalTok{]}
\end{Highlighting}
\end{Shaded}

\begin{verbatim}
## [1] 7
\end{verbatim}

Você pode obter uma linha inteira da matriz omitindo o índice da coluna (mas mantenha a virgula). Tente recuperar a segunda linha:

\begin{Shaded}
\begin{Highlighting}[]
\NormalTok{foliar[}\DecValTok{2}\NormalTok{,]}
\end{Highlighting}
\end{Shaded}

\begin{verbatim}
## [1] 2 4 6 8
\end{verbatim}

Para obter uma coluna inteira, omita o índice da linha. Recupere a quarta coluna:

\begin{Shaded}
\begin{Highlighting}[]
\NormalTok{foliar[,}\DecValTok{4}\NormalTok{]}
\end{Highlighting}
\end{Shaded}

\begin{verbatim}
## [1] 7 8
\end{verbatim}

Você pode ler várias linhas ou colunas, fornecendo um vetor ou sequência com seus índices. Tente recuperar as colunas de 2 a 4:

\begin{Shaded}
\begin{Highlighting}[]
\NormalTok{foliar[,}\DecValTok{2}\OperatorTok{:}\DecValTok{4}\NormalTok{]}
\end{Highlighting}
\end{Shaded}

\begin{verbatim}
##      [,1] [,2] [,3]
## [1,]    3    5    7
## [2,]    4    6    8
\end{verbatim}

O comando \texttt{summary} pode ser usado para obter informações da matriz

\begin{Shaded}
\begin{Highlighting}[]
\KeywordTok{summary}\NormalTok{(foliar)}
\end{Highlighting}
\end{Shaded}

\begin{verbatim}
##        V1             V2             V3             V4      
##  Min.   :1.00   Min.   :3.00   Min.   :5.00   Min.   :7.00  
##  1st Qu.:1.25   1st Qu.:3.25   1st Qu.:5.25   1st Qu.:7.25  
##  Median :1.50   Median :3.50   Median :5.50   Median :7.50  
##  Mean   :1.50   Mean   :3.50   Mean   :5.50   Mean   :7.50  
##  3rd Qu.:1.75   3rd Qu.:3.75   3rd Qu.:5.75   3rd Qu.:7.75  
##  Max.   :2.00   Max.   :4.00   Max.   :6.00   Max.   :8.00
\end{verbatim}

Se desejar um resumo de todos os elementos da matriz, basta transformá-la em um vetor

\begin{Shaded}
\begin{Highlighting}[]
\KeywordTok{summary}\NormalTok{(}\KeywordTok{as.vector}\NormalTok{(foliar))}
\end{Highlighting}
\end{Shaded}

\begin{verbatim}
##    Min. 1st Qu.  Median    Mean 3rd Qu.    Max. 
##    1.00    2.75    4.50    4.50    6.25    8.00
\end{verbatim}

\hypertarget{visualizauxe7uxf5es-em-dados-matriciais}{%
\subsection{Visualizações em dados matriciais}\label{visualizauxe7uxf5es-em-dados-matriciais}}

Com um mapa de elevação. Tudo fica a 1 metro acima do nível do mar. Vamos criar uma matriz de 10 por 10 com todos os seus valores inicializados para 1 para você:

\begin{Shaded}
\begin{Highlighting}[]
\NormalTok{elevacao <-}\StringTok{ }\KeywordTok{matrix}\NormalTok{ (}\DecValTok{1}\NormalTok{,}\DecValTok{10}\NormalTok{,}\DecValTok{10}\NormalTok{)}
\end{Highlighting}
\end{Shaded}

Na quarta linha, sexta coluna, defina a elevação para 0:

\begin{Shaded}
\begin{Highlighting}[]
\NormalTok{elevacao [}\DecValTok{4}\NormalTok{, }\DecValTok{6}\NormalTok{] <-}\StringTok{ }\DecValTok{0}
\end{Highlighting}
\end{Shaded}

Mapa de contorno dos valores passando a matriz para a função \texttt{contour}

\begin{Shaded}
\begin{Highlighting}[]
\KeywordTok{contour}\NormalTok{(elevacao)}
\end{Highlighting}
\end{Shaded}

\includegraphics{TudodoR_files/figure-latex/unnamed-chunk-71-1.pdf}

Criar um gráfico em perspectiva 3D com a função \texttt{persp}:

\begin{Shaded}
\begin{Highlighting}[]
\KeywordTok{persp}\NormalTok{ (elevacao)}
\end{Highlighting}
\end{Shaded}

\includegraphics{TudodoR_files/figure-latex/unnamed-chunk-72-1.pdf}

Podemos consertar isso especificando nosso próprio valor para o parâmetro \textbf{expand}.

\begin{Shaded}
\begin{Highlighting}[]
\KeywordTok{persp}\NormalTok{ (elevacao, }\DataTypeTok{expand =}\FloatTok{0.2}\NormalTok{)}
\end{Highlighting}
\end{Shaded}

\includegraphics{TudodoR_files/figure-latex/unnamed-chunk-73-1.pdf}

R inclui alguns conjuntos de dados de amostra. Um deles é o \emph{volcanoum} mapa 3D de um vulcão adormecido da Nova Zelândia.

É simplesmente uma matriz de 87x61 com valores de elevão, mas mostra o poder das visualizações de matriz do R. Criar um mapa de calor:

\begin{Shaded}
\begin{Highlighting}[]
\KeywordTok{contour}\NormalTok{(volcano)}
\end{Highlighting}
\end{Shaded}

\includegraphics{TudodoR_files/figure-latex/unnamed-chunk-74-1.pdf}

Gráfico em perspectiva:

\begin{Shaded}
\begin{Highlighting}[]
\KeywordTok{persp}\NormalTok{(volcano, }\DataTypeTok{expand=}\FloatTok{0.2}\NormalTok{)}
\end{Highlighting}
\end{Shaded}

\includegraphics{TudodoR_files/figure-latex/unnamed-chunk-75-1.pdf}

A função \texttt{image} criar um mapa de calor:

\begin{Shaded}
\begin{Highlighting}[]
\KeywordTok{image}\NormalTok{(volcano)}
\end{Highlighting}
\end{Shaded}

\includegraphics{TudodoR_files/figure-latex/unnamed-chunk-76-1.pdf}

\hypertarget{mais-informauxe7uxf5es-sobre-construuxe7uxf5es-de-matrizes}{%
\subsection{Mais informações sobre construções de Matrizes}\label{mais-informauxe7uxf5es-sobre-construuxe7uxf5es-de-matrizes}}

Há outros comandos que podem ser usados para construir matrizes como \texttt{cbind()} e \texttt{rbind\ ()}. Esses comandos concatenam colunas ou linhas, respectivamente, na matriz (ou vetor).

\begin{Shaded}
\begin{Highlighting}[]
\NormalTok{a <-}\StringTok{ }\KeywordTok{matrix}\NormalTok{ (}\DecValTok{10}\OperatorTok{:}\DecValTok{1}\NormalTok{,}\DataTypeTok{ncol=}\DecValTok{2}\NormalTok{) }\CommentTok{#construir uma matriz qualquer}
\NormalTok{a}
\end{Highlighting}
\end{Shaded}

\begin{verbatim}
##      [,1] [,2]
## [1,]   10    5
## [2,]    9    4
## [3,]    8    3
## [4,]    7    2
## [5,]    6    1
\end{verbatim}

\begin{Shaded}
\begin{Highlighting}[]
\NormalTok{b <-}\StringTok{ }\KeywordTok{cbind}\NormalTok{ (a,}\DecValTok{1}\OperatorTok{:}\DecValTok{5}\NormalTok{) }\CommentTok{#adicionar uma terceira coluna}
\NormalTok{b}
\end{Highlighting}
\end{Shaded}

\begin{verbatim}
##      [,1] [,2] [,3]
## [1,]   10    5    1
## [2,]    9    4    2
## [3,]    8    3    3
## [4,]    7    2    4
## [5,]    6    1    5
\end{verbatim}

\begin{Shaded}
\begin{Highlighting}[]
\NormalTok{c<-}\StringTok{ }\KeywordTok{rbind}\NormalTok{(b,}\KeywordTok{c}\NormalTok{(}\DecValTok{28}\NormalTok{,}\DecValTok{28}\NormalTok{,}\DecValTok{28}\NormalTok{))}
\NormalTok{c}
\end{Highlighting}
\end{Shaded}

\begin{verbatim}
##      [,1] [,2] [,3]
## [1,]   10    5    1
## [2,]    9    4    2
## [3,]    8    3    3
## [4,]    7    2    4
## [5,]    6    1    5
## [6,]   28   28   28
\end{verbatim}

Opcionalmente matrizes podem ter nomes associados ás linhas e colunas (``rownames''e ``colnames''). Cada um destes componentes da matrix é um vetor de nomes.

\begin{Shaded}
\begin{Highlighting}[]
\NormalTok{m1 <-}\StringTok{ }\KeywordTok{matrix}\NormalTok{(}\DecValTok{1}\OperatorTok{:}\DecValTok{12}\NormalTok{, }\DataTypeTok{ncol =} \DecValTok{3}\NormalTok{) }

\KeywordTok{dimnames}\NormalTok{(m1) <-}\StringTok{ }\KeywordTok{list}\NormalTok{(}\KeywordTok{c}\NormalTok{(}\StringTok{"L1"}\NormalTok{, }\StringTok{"L2"}\NormalTok{, }\StringTok{"L3"}\NormalTok{, }\StringTok{"L4"}\NormalTok{), }\KeywordTok{c}\NormalTok{(}\StringTok{"C1"}\NormalTok{, }\StringTok{"C2"}\NormalTok{, }\StringTok{"C3"}\NormalTok{)) }
\KeywordTok{dimnames}\NormalTok{(m1)}
\end{Highlighting}
\end{Shaded}

\begin{verbatim}
## [[1]]
## [1] "L1" "L2" "L3" "L4"
## 
## [[2]]
## [1] "C1" "C2" "C3"
\end{verbatim}

Matrizes são muitas vezes utilizadas para armazenar frequências de cruzamentos entre variáveis. Desta forma é comum surgir a necessidade de obter os totais marginais, isto é a soma dos elementos das linhas e/ou colunas das matrizes, o que pode ser diretamente obtido com \texttt{margin.table(\ )}.

\begin{Shaded}
\begin{Highlighting}[]
 \KeywordTok{margin.table}\NormalTok{(m1, }\DataTypeTok{margin =} \DecValTok{1}\NormalTok{)}
\end{Highlighting}
\end{Shaded}

\begin{verbatim}
## L1 L2 L3 L4 
## 15 18 21 24
\end{verbatim}

\begin{Shaded}
\begin{Highlighting}[]
 \KeywordTok{margin.table}\NormalTok{(m1, }\DataTypeTok{margin =} \DecValTok{2}\NormalTok{)}
\end{Highlighting}
\end{Shaded}

\begin{verbatim}
## C1 C2 C3 
## 10 26 42
\end{verbatim}

\begin{Shaded}
\begin{Highlighting}[]
 \KeywordTok{apply}\NormalTok{(m1, }\DecValTok{2}\NormalTok{, median)}
\end{Highlighting}
\end{Shaded}

\begin{verbatim}
##   C1   C2   C3 
##  2.5  6.5 10.5
\end{verbatim}

\hypertarget{fatores}{%
\section{Fatores}\label{fatores}}

Os fatores são vetores em que os elementos pertencem a uma ou mais categorias temáticas. Por exemplo: ao criar um vetor de indicadores de \textbf{``tratamentos''} em uma análise de experimentos devemos declarar este vetor como um \textbf{``fator''}.
Pode criar um fator usando o comando \textbf{factor()}, ou o comando \textbf{gl}.

\begin{Shaded}
\begin{Highlighting}[]
\KeywordTok{factor}\NormalTok{(}\KeywordTok{rep}\NormalTok{(}\KeywordTok{paste}\NormalTok{(}\StringTok{"T"}\NormalTok{, }\DecValTok{1}\OperatorTok{:}\DecValTok{3}\NormalTok{, }\DataTypeTok{sep =} \StringTok{""}\NormalTok{), }\KeywordTok{c}\NormalTok{(}\DecValTok{4}\NormalTok{, }\DecValTok{4}\NormalTok{, }\DecValTok{3}\NormalTok{)))}
\end{Highlighting}
\end{Shaded}

\begin{verbatim}
##  [1] T1 T1 T1 T1 T2 T2 T2 T2 T3 T3 T3
## Levels: T1 T2 T3
\end{verbatim}

\begin{Shaded}
\begin{Highlighting}[]
\NormalTok{peso  <-}\StringTok{ }\KeywordTok{c}\NormalTok{(}\FloatTok{134.8}\NormalTok{, }\FloatTok{139.7}\NormalTok{, }\FloatTok{147.6}\NormalTok{, }\FloatTok{132.3}\NormalTok{, }\FloatTok{161.7}\NormalTok{, }\FloatTok{157.7}\NormalTok{, }\FloatTok{150.3}\NormalTok{, }\FloatTok{144.7}\NormalTok{,}
           \FloatTok{160.7}\NormalTok{, }\FloatTok{172.7}\NormalTok{, }\FloatTok{163.4}\NormalTok{, }\FloatTok{161.3}\NormalTok{, }\FloatTok{169.8}\NormalTok{, }\FloatTok{168.2}\NormalTok{, }\FloatTok{160.7}\NormalTok{, }\FloatTok{161.0}\NormalTok{,}
           \FloatTok{165.7}\NormalTok{, }\FloatTok{160.0}\NormalTok{, }\FloatTok{158.2}\NormalTok{, }\FloatTok{151.0}\NormalTok{, }\FloatTok{171.8}\NormalTok{, }\FloatTok{157.3}\NormalTok{, }\FloatTok{150.4}\NormalTok{, }\FloatTok{160.4}\NormalTok{,}
           \FloatTok{154.5}\NormalTok{, }\FloatTok{160.4}\NormalTok{, }\FloatTok{148.8}\NormalTok{, }\FloatTok{154.0}\NormalTok{)}
\NormalTok{trat  <-}\StringTok{ }\KeywordTok{rep}\NormalTok{(}\KeywordTok{seq}\NormalTok{(}\DecValTok{0}\NormalTok{,}\DecValTok{300}\NormalTok{,}\DecValTok{50}\NormalTok{), }\DataTypeTok{each=}\DecValTok{4}\NormalTok{)  }\CommentTok{#?each}
\NormalTok{dados <-}\StringTok{  }\KeywordTok{data.frame}\NormalTok{(peso, }\DataTypeTok{trat=}\KeywordTok{as.factor}\NormalTok{(trat))}
\end{Highlighting}
\end{Shaded}

\hypertarget{array}{%
\section{Array}\label{array}}

O conceito de array generaliza a idéia de matrix. Enquanto em uma matrix os elementos são organizados em duas dimensões (linhas e colunas), em um array os elementos podem ser organizados em um número arbitrário de dimensões.
No R um array é definido utilizando a função \texttt{array()}.

\begin{Shaded}
\begin{Highlighting}[]
\NormalTok{ar1 <-}\StringTok{ }\KeywordTok{array}\NormalTok{(}\DecValTok{1}\OperatorTok{:}\DecValTok{24}\NormalTok{, }\DataTypeTok{dim =} \KeywordTok{c}\NormalTok{(}\DecValTok{3}\NormalTok{, }\DecValTok{4}\NormalTok{, }\DecValTok{2}\NormalTok{)) }
\NormalTok{ar1}
\end{Highlighting}
\end{Shaded}

\begin{verbatim}
## , , 1
## 
##      [,1] [,2] [,3] [,4]
## [1,]    1    4    7   10
## [2,]    2    5    8   11
## [3,]    3    6    9   12
## 
## , , 2
## 
##      [,1] [,2] [,3] [,4]
## [1,]   13   16   19   22
## [2,]   14   17   20   23
## [3,]   15   18   21   24
\end{verbatim}

Veja agora um exemplo de dados já incluído no R no formato de array. Para ``carregar'' e visualizar os dados digite:

\begin{Shaded}
\begin{Highlighting}[]
\KeywordTok{data}\NormalTok{(Titanic) }
\NormalTok{Titanic}
\end{Highlighting}
\end{Shaded}

\begin{verbatim}
## , , Age = Child, Survived = No
## 
##       Sex
## Class  Male Female
##   1st     0      0
##   2nd     0      0
##   3rd    35     17
##   Crew    0      0
## 
## , , Age = Adult, Survived = No
## 
##       Sex
## Class  Male Female
##   1st   118      4
##   2nd   154     13
##   3rd   387     89
##   Crew  670      3
## 
## , , Age = Child, Survived = Yes
## 
##       Sex
## Class  Male Female
##   1st     5      1
##   2nd    11     13
##   3rd    13     14
##   Crew    0      0
## 
## , , Age = Adult, Survived = Yes
## 
##       Sex
## Class  Male Female
##   1st    57    140
##   2nd    14     80
##   3rd    75     76
##   Crew  192     20
\end{verbatim}

Para obter maiores informações sobre estes dados digite: \texttt{help(Titanic)}

Agora vamos responder ás seguintes perguntas, mostrando os comandos do R utilizados sobre o array de dados.

\begin{enumerate}
\def\labelenumi{\arabic{enumi}.}
\tightlist
\item
  Quantas pessoas havia no total?
\end{enumerate}

\begin{Shaded}
\begin{Highlighting}[]
\KeywordTok{sum}\NormalTok{(Titanic)}
\end{Highlighting}
\end{Shaded}

\begin{verbatim}
## [1] 2201
\end{verbatim}

\begin{enumerate}
\def\labelenumi{\arabic{enumi}.}
\setcounter{enumi}{1}
\tightlist
\item
  Quantas pessoas havia na tripulação (crew)?
\end{enumerate}

\begin{Shaded}
\begin{Highlighting}[]
\KeywordTok{sum}\NormalTok{(Titanic[}\DecValTok{4}\NormalTok{, , , ])}
\end{Highlighting}
\end{Shaded}

\begin{verbatim}
## [1] 885
\end{verbatim}

\begin{enumerate}
\def\labelenumi{\arabic{enumi}.}
\setcounter{enumi}{2}
\tightlist
\item
  Quantas pessoas sobreviveram e quantas morreram?
\end{enumerate}

\begin{Shaded}
\begin{Highlighting}[]
\KeywordTok{apply}\NormalTok{(Titanic, }\DecValTok{4}\NormalTok{, sum)}
\end{Highlighting}
\end{Shaded}

\begin{verbatim}
##   No  Yes 
## 1490  711
\end{verbatim}

\begin{enumerate}
\def\labelenumi{\arabic{enumi}.}
\setcounter{enumi}{3}
\tightlist
\item
  Quais as proporções de sobreviventes entre homens e mulheres?
\end{enumerate}

\begin{Shaded}
\begin{Highlighting}[]
\KeywordTok{margin.table}\NormalTok{(Titanic, }\DataTypeTok{margin =} \DecValTok{1}\NormalTok{)}
\end{Highlighting}
\end{Shaded}

\begin{verbatim}
## Class
##  1st  2nd  3rd Crew 
##  325  285  706  885
\end{verbatim}

\begin{Shaded}
\begin{Highlighting}[]
\KeywordTok{margin.table}\NormalTok{(Titanic, }\DataTypeTok{margin =} \DecValTok{2}\NormalTok{)}
\end{Highlighting}
\end{Shaded}

\begin{verbatim}
## Sex
##   Male Female 
##   1731    470
\end{verbatim}

\begin{Shaded}
\begin{Highlighting}[]
\KeywordTok{margin.table}\NormalTok{(Titanic, }\DataTypeTok{margin =} \DecValTok{3}\NormalTok{)}
\end{Highlighting}
\end{Shaded}

\begin{verbatim}
## Age
## Child Adult 
##   109  2092
\end{verbatim}

\begin{Shaded}
\begin{Highlighting}[]
\KeywordTok{margin.table}\NormalTok{(Titanic, }\DataTypeTok{margin =} \DecValTok{4}\NormalTok{)}
\end{Highlighting}
\end{Shaded}

\begin{verbatim}
## Survived
##   No  Yes 
## 1490  711
\end{verbatim}

Esta função admite ainda índices múltiplos que permitem outros resumos da tabela de dados. Por exemplo mostramos a seguir como obter o total de sobreviventes e não sobreviventes, separados por sexo e depois as porcentagens de sobreviventes para cada sexo.

\begin{Shaded}
\begin{Highlighting}[]
\KeywordTok{margin.table}\NormalTok{(Titanic, }\DataTypeTok{margin =} \KeywordTok{c}\NormalTok{(}\DecValTok{2}\NormalTok{, }\DecValTok{4}\NormalTok{))}
\end{Highlighting}
\end{Shaded}

\begin{verbatim}
##         Survived
## Sex        No  Yes
##   Male   1364  367
##   Female  126  344
\end{verbatim}

\begin{Shaded}
\begin{Highlighting}[]
\KeywordTok{prop.table}\NormalTok{(}\KeywordTok{margin.table}\NormalTok{(Titanic, }\DataTypeTok{margin =} \KeywordTok{c}\NormalTok{(}\DecValTok{2}\NormalTok{, }\DecValTok{4}\NormalTok{)), }\DataTypeTok{margin =} \DecValTok{1}\NormalTok{)}
\end{Highlighting}
\end{Shaded}

\begin{verbatim}
##         Survived
## Sex             No       Yes
##   Male   0.7879838 0.2120162
##   Female 0.2680851 0.7319149
\end{verbatim}

\begin{Shaded}
\begin{Highlighting}[]
\KeywordTok{prop.table}\NormalTok{(}\KeywordTok{margin.table}\NormalTok{(Titanic, }\DataTypeTok{margin =} \KeywordTok{c}\NormalTok{(}\DecValTok{2}\NormalTok{, }\DecValTok{1}\NormalTok{)), }\DataTypeTok{margin =} \DecValTok{1}\NormalTok{)}
\end{Highlighting}
\end{Shaded}

\begin{verbatim}
##         Class
## Sex             1st        2nd        3rd       Crew
##   Male   0.10398614 0.10340843 0.29462738 0.49797805
##   Female 0.30851064 0.22553191 0.41702128 0.04893617
\end{verbatim}

\hypertarget{data.frame}{%
\section{Data.frame}\label{data.frame}}

Os datas.frames são muitos semelhantes ás matrizes, pois têm linhas e colunas e, portanto, duas dimensões. Entretando, diferentemente das matrizes, colunas diferentes podem armazenar elementos de tipos diferentes. Por exemplo, a primeira coluna pode ser numérica, enquanto a segunda, constituida de caracteres. Cada coluna precisa ter o mesmo tamanho.
Criar o vetor nomes

\begin{Shaded}
\begin{Highlighting}[]
\NormalTok{nome <-}\StringTok{ }\KeywordTok{c}\NormalTok{(}\StringTok{"Melissa José"}\NormalTok{,}
          \StringTok{"Jennifer Linhares"}\NormalTok{,}
          \StringTok{"Gedilene Ponciano"}\NormalTok{,}
          \StringTok{"Edinar da Silva"}\NormalTok{,}
          \StringTok{"Osmar Emidio"}\NormalTok{,}
          \StringTok{"Jeeziel Vieira"}\NormalTok{)}
\end{Highlighting}
\end{Shaded}

Criar vetor idade

\begin{Shaded}
\begin{Highlighting}[]
\NormalTok{idade <-}\StringTok{ }\KeywordTok{c}\NormalTok{(}\DecValTok{17}\NormalTok{,}\DecValTok{18}\NormalTok{,}\DecValTok{16}\NormalTok{,}\DecValTok{15}\NormalTok{,}\DecValTok{15}\NormalTok{,}\DecValTok{18}\NormalTok{)}
\end{Highlighting}
\end{Shaded}

Criar vetor sexo (categoria=fator)

\begin{Shaded}
\begin{Highlighting}[]
\NormalTok{sexo <-}\StringTok{ }\KeywordTok{factor}\NormalTok{(}\KeywordTok{c}\NormalTok{(}\StringTok{"F"}\NormalTok{,}\StringTok{"F"}\NormalTok{,}\StringTok{"F"}\NormalTok{,}\StringTok{"F"}\NormalTok{,}\StringTok{"M"}\NormalTok{,}\StringTok{"M"}\NormalTok{))}
\end{Highlighting}
\end{Shaded}

Criar vetor altura

\begin{Shaded}
\begin{Highlighting}[]
\NormalTok{alt <-}\StringTok{ }\KeywordTok{c}\NormalTok{(}\DecValTok{180}\NormalTok{,}\DecValTok{170}\NormalTok{,}\DecValTok{160}\NormalTok{,}\DecValTok{150}\NormalTok{,}\DecValTok{140}\NormalTok{,}\DecValTok{168}\NormalTok{)}
\end{Highlighting}
\end{Shaded}

Reunir tudo em um data.frame

\begin{Shaded}
\begin{Highlighting}[]
\NormalTok{dados <-}\StringTok{ }\KeywordTok{data.frame}\NormalTok{(nome, idade, sexo, alt)}
\end{Highlighting}
\end{Shaded}

Ver atributos da tabela

\begin{Shaded}
\begin{Highlighting}[]
\KeywordTok{str}\NormalTok{(dados)}
\end{Highlighting}
\end{Shaded}

\begin{verbatim}
## 'data.frame':    6 obs. of  4 variables:
##  $ nome : Factor w/ 6 levels "Edinar da Silva",..: 5 4 2 1 6 3
##  $ idade: num  17 18 16 15 15 18
##  $ sexo : Factor w/ 2 levels "F","M": 1 1 1 1 2 2
##  $ alt  : num  180 170 160 150 140 168
\end{verbatim}

Adicionar nome as linhas com o comando \texttt{row.names()}

\begin{Shaded}
\begin{Highlighting}[]
\KeywordTok{row.names}\NormalTok{(dados) <-}\StringTok{ }\KeywordTok{c}\NormalTok{(}\DecValTok{1}\NormalTok{,}\DecValTok{2}\NormalTok{,}\DecValTok{3}\NormalTok{,}\DecValTok{4}\NormalTok{,}\DecValTok{5}\NormalTok{,}\DecValTok{6}\NormalTok{)}
\NormalTok{dados}
\end{Highlighting}
\end{Shaded}

\begin{verbatim}
##                nome idade sexo alt
## 1      Melissa José    17    F 180
## 2 Jennifer Linhares    18    F 170
## 3 Gedilene Ponciano    16    F 160
## 4   Edinar da Silva    15    F 150
## 5      Osmar Emidio    15    M 140
## 6    Jeeziel Vieira    18    M 168
\end{verbatim}

\begin{Shaded}
\begin{Highlighting}[]
\KeywordTok{names}\NormalTok{(dados) <-}\StringTok{ }\KeywordTok{c}\NormalTok{(}\StringTok{"Nome"}\NormalTok{, }\StringTok{"Idade"}\NormalTok{, }\StringTok{"Sexo"}\NormalTok{, }\StringTok{"altura"}\NormalTok{)}
\NormalTok{dados}
\end{Highlighting}
\end{Shaded}

\begin{verbatim}
##                Nome Idade Sexo altura
## 1      Melissa José    17    F    180
## 2 Jennifer Linhares    18    F    170
## 3 Gedilene Ponciano    16    F    160
## 4   Edinar da Silva    15    F    150
## 5      Osmar Emidio    15    M    140
## 6    Jeeziel Vieira    18    M    168
\end{verbatim}

\hypertarget{uxedndice-dos-data.frames}{%
\subsection{Índice dos Data.frames}\label{uxedndice-dos-data.frames}}

Buscar elementos

\begin{Shaded}
\begin{Highlighting}[]
\NormalTok{dados[}\DecValTok{2}\NormalTok{,}\DecValTok{1}\NormalTok{] }\CommentTok{#elemento da  linha  2, coluna 1}
\end{Highlighting}
\end{Shaded}

\begin{verbatim}
## [1] Jennifer Linhares
## 6 Levels: Edinar da Silva Gedilene Ponciano ... Osmar Emidio
\end{verbatim}

\begin{Shaded}
\begin{Highlighting}[]
\NormalTok{dados[}\DecValTok{2}\NormalTok{,] }\CommentTok{#toda linha dois}
\end{Highlighting}
\end{Shaded}

\begin{verbatim}
##                Nome Idade Sexo altura
## 2 Jennifer Linhares    18    F    170
\end{verbatim}

Repare que apesar de ``Nomes'' ter sido criado como vetor de caracterer o R passou a entender como um fator dentro do data.frame.

\begin{Shaded}
\begin{Highlighting}[]
\NormalTok{dados[,}\DecValTok{1}\NormalTok{]}
\end{Highlighting}
\end{Shaded}

\begin{verbatim}
## [1] Melissa José      Jennifer Linhares Gedilene Ponciano Edinar da Silva  
## [5] Osmar Emidio      Jeeziel Vieira   
## 6 Levels: Edinar da Silva Gedilene Ponciano ... Osmar Emidio
\end{verbatim}

Transformar para caracterer

\begin{Shaded}
\begin{Highlighting}[]
\NormalTok{dados[,}\DecValTok{1}\NormalTok{] <-}\StringTok{ }\KeywordTok{as.character}\NormalTok{(dados[,}\DecValTok{1}\NormalTok{])}
\NormalTok{dados[,}\DecValTok{1}\NormalTok{]}
\end{Highlighting}
\end{Shaded}

\begin{verbatim}
## [1] "Melissa José"      "Jennifer Linhares" "Gedilene Ponciano"
## [4] "Edinar da Silva"   "Osmar Emidio"      "Jeeziel Vieira"
\end{verbatim}

Acessando aos dados

\begin{Shaded}
\begin{Highlighting}[]
\NormalTok{dados}\OperatorTok{$}\NormalTok{Nome}
\end{Highlighting}
\end{Shaded}

\begin{verbatim}
## [1] "Melissa José"      "Jennifer Linhares" "Gedilene Ponciano"
## [4] "Edinar da Silva"   "Osmar Emidio"      "Jeeziel Vieira"
\end{verbatim}

\begin{Shaded}
\begin{Highlighting}[]
\NormalTok{dados}\OperatorTok{$}\NormalTok{Nome[}\DecValTok{3}\NormalTok{]}
\end{Highlighting}
\end{Shaded}

\begin{verbatim}
## [1] "Gedilene Ponciano"
\end{verbatim}

\begin{Shaded}
\begin{Highlighting}[]
\NormalTok{dados}\OperatorTok{$}\NormalTok{Nome [}\DecValTok{1}\OperatorTok{:}\DecValTok{3}\NormalTok{]}
\end{Highlighting}
\end{Shaded}

\begin{verbatim}
## [1] "Melissa José"      "Jennifer Linhares" "Gedilene Ponciano"
\end{verbatim}

\begin{Shaded}
\begin{Highlighting}[]
\KeywordTok{str}\NormalTok{(dados)}
\end{Highlighting}
\end{Shaded}

\begin{verbatim}
## 'data.frame':    6 obs. of  4 variables:
##  $ Nome  : chr  "Melissa José" "Jennifer Linhares" "Gedilene Ponciano" "Edinar da Silva" ...
##  $ Idade : num  17 18 16 15 15 18
##  $ Sexo  : Factor w/ 2 levels "F","M": 1 1 1 1 2 2
##  $ altura: num  180 170 160 150 140 168
\end{verbatim}

\hypertarget{manipulando-um-data.frame}{%
\subsection{Manipulando um Data.frame}\label{manipulando-um-data.frame}}

Você pode manipular um data.frame add ou eliminando colunas ou linhas, assim como em matrizes. Podem-se usar os comandos \texttt{cbind()} e \texttt{rbind\ ()} para adcionar colunas e linhas rescpectivamente, a um data.frame.

\begin{Shaded}
\begin{Highlighting}[]
\NormalTok{dados <-}\StringTok{ }\KeywordTok{cbind}\NormalTok{ (dados, }\CommentTok{#adicionar uma coluna}
               \DataTypeTok{Conceito=}\KeywordTok{c}\NormalTok{(}\StringTok{"A"}\NormalTok{,}\StringTok{"A"}\NormalTok{,}\StringTok{"A"}\NormalTok{,}\StringTok{"C"}\NormalTok{,}\StringTok{"A"}\NormalTok{,}\StringTok{"B"}\NormalTok{))}
\end{Highlighting}
\end{Shaded}

\begin{Shaded}
\begin{Highlighting}[]
\NormalTok{dados <-}\StringTok{ }\KeywordTok{rbind}\NormalTok{ (dados, }\CommentTok{#adicionar uma linha}
                \StringTok{"7"}\NormalTok{=}\StringTok{ }\KeywordTok{c}\NormalTok{(}\StringTok{"Caio Pinto"}\NormalTok{, }\DecValTok{21}\NormalTok{, }\StringTok{"M"}\NormalTok{, }\DecValTok{172}\NormalTok{, }\StringTok{"C"}\NormalTok{))}
\NormalTok{dados}
\end{Highlighting}
\end{Shaded}

\begin{verbatim}
##                Nome Idade Sexo altura Conceito
## 1      Melissa José    17    F    180        A
## 2 Jennifer Linhares    18    F    170        A
## 3 Gedilene Ponciano    16    F    160        A
## 4   Edinar da Silva    15    F    150        C
## 5      Osmar Emidio    15    M    140        A
## 6    Jeeziel Vieira    18    M    168        B
## 7        Caio Pinto    21    M    172        C
\end{verbatim}

Assim como para vetores e matrizes voce pode selecinar um subgrupo de um data.frame e armazena-lo em um outro objeto ou utilizar índices como o sinal negativo para eliminar linhas ou colunas de um data.frame.

\begin{Shaded}
\begin{Highlighting}[]
\NormalTok{dados<-}\StringTok{ }\NormalTok{dados [}\DecValTok{1}\OperatorTok{:}\DecValTok{6}\NormalTok{,] }\CommentTok{#selecionar linha de 1 a 6}
\NormalTok{dados<-}\StringTok{ }\NormalTok{dados [,}\OperatorTok{-}\DecValTok{5}\NormalTok{] }\CommentTok{#excluir a quinta coluna}
\NormalTok{dados}
\end{Highlighting}
\end{Shaded}

\begin{verbatim}
##                Nome Idade Sexo altura
## 1      Melissa José    17    F    180
## 2 Jennifer Linhares    18    F    170
## 3 Gedilene Ponciano    16    F    160
## 4   Edinar da Silva    15    F    150
## 5      Osmar Emidio    15    M    140
## 6    Jeeziel Vieira    18    M    168
\end{verbatim}

\begin{Shaded}
\begin{Highlighting}[]
\NormalTok{dados[dados}\OperatorTok{$}\NormalTok{Sexo}\OperatorTok{==}\StringTok{"F"}\NormalTok{,] }\CommentTok{#exibir só masculinos}
\end{Highlighting}
\end{Shaded}

\begin{verbatim}
##                Nome Idade Sexo altura
## 1      Melissa José    17    F    180
## 2 Jennifer Linhares    18    F    170
## 3 Gedilene Ponciano    16    F    160
## 4   Edinar da Silva    15    F    150
\end{verbatim}

A ordenação das linhas de um \textbf{data.frame} segundo os dados contidos em determinadas coluna também é extremamente útil

\begin{Shaded}
\begin{Highlighting}[]
\NormalTok{dados [}\KeywordTok{order}\NormalTok{(dados}\OperatorTok{$}\NormalTok{altura),]}
\end{Highlighting}
\end{Shaded}

\begin{verbatim}
##                Nome Idade Sexo altura
## 5      Osmar Emidio    15    M    140
## 4   Edinar da Silva    15    F    150
## 3 Gedilene Ponciano    16    F    160
## 6    Jeeziel Vieira    18    M    168
## 2 Jennifer Linhares    18    F    170
## 1      Melissa José    17    F    180
\end{verbatim}

\begin{Shaded}
\begin{Highlighting}[]
\NormalTok{dados [}\KeywordTok{rev}\NormalTok{(}\KeywordTok{order}\NormalTok{(dados}\OperatorTok{$}\NormalTok{altura)),]}
\end{Highlighting}
\end{Shaded}

\begin{verbatim}
##                Nome Idade Sexo altura
## 1      Melissa José    17    F    180
## 2 Jennifer Linhares    18    F    170
## 6    Jeeziel Vieira    18    M    168
## 3 Gedilene Ponciano    16    F    160
## 4   Edinar da Silva    15    F    150
## 5      Osmar Emidio    15    M    140
\end{verbatim}

\hypertarget{separando-um-data.frame-por-grupos}{%
\subsection{Separando um data.frame por grupos}\label{separando-um-data.frame-por-grupos}}

\begin{Shaded}
\begin{Highlighting}[]
\KeywordTok{split}\NormalTok{ (dados, sexo)}
\end{Highlighting}
\end{Shaded}

\begin{verbatim}
## $F
##                Nome Idade Sexo altura
## 1      Melissa José    17    F    180
## 2 Jennifer Linhares    18    F    170
## 3 Gedilene Ponciano    16    F    160
## 4   Edinar da Silva    15    F    150
## 
## $M
##             Nome Idade Sexo altura
## 5   Osmar Emidio    15    M    140
## 6 Jeeziel Vieira    18    M    168
\end{verbatim}

\hypertarget{lista}{%
\section{Lista}\label{lista}}

Lista são objetos muito úteis, pois são usados para combinar diferente estruturas de dados em um mesmo objeto, ou seja, vetores, matrizes, arrays, data.frames e ate mesmo outras listas.

\begin{Shaded}
\begin{Highlighting}[]
\NormalTok{pes <-}\StringTok{ }\KeywordTok{list}\NormalTok{ (}\DataTypeTok{idade=}\DecValTok{32}\NormalTok{, }\DataTypeTok{nome=}\StringTok{"Maria"}\NormalTok{, }\DataTypeTok{notas=}\KeywordTok{c}\NormalTok{(}\DecValTok{98}\NormalTok{,}\DecValTok{95}\NormalTok{,}\DecValTok{78}\NormalTok{), }\DataTypeTok{B=}\KeywordTok{matrix}\NormalTok{(}\DecValTok{1}\OperatorTok{:}\DecValTok{4}\NormalTok{,}\DecValTok{2}\NormalTok{,}\DecValTok{2}\NormalTok{))}
\NormalTok{pes}
\end{Highlighting}
\end{Shaded}

\begin{verbatim}
## $idade
## [1] 32
## 
## $nome
## [1] "Maria"
## 
## $notas
## [1] 98 95 78
## 
## $B
##      [,1] [,2]
## [1,]    1    3
## [2,]    2    4
\end{verbatim}

Lista são construidas com o comando \texttt{list\ ()}. Quando você exibe um objeto que é uma lista, cada componente é mostrado com seu nome \textbf{\$} ou \textbf{{[} {]}}

\hypertarget{alguns-comandos-que-retornam-listas}{%
\subsection{Alguns comandos que retornam listas}\label{alguns-comandos-que-retornam-listas}}

Muitos comando do R retornam seu resultado na forma de listas. Um exemplo pode ser mostrado com o uso do comando \texttt{t.tes()}, que retorna um objeto que é uma lista.

\begin{Shaded}
\begin{Highlighting}[]
\NormalTok{x <-}\StringTok{ }\KeywordTok{c}\NormalTok{(}\DecValTok{1}\NormalTok{,}\DecValTok{3}\NormalTok{,}\DecValTok{2}\NormalTok{,}\DecValTok{3}\NormalTok{,}\DecValTok{4}\NormalTok{)}
\NormalTok{y <-}\StringTok{ }\KeywordTok{c}\NormalTok{(}\DecValTok{4}\NormalTok{,}\DecValTok{5}\NormalTok{,}\DecValTok{5}\NormalTok{,}\DecValTok{4}\NormalTok{,}\DecValTok{4}\NormalTok{)}
\NormalTok{tt <-}\StringTok{ }\KeywordTok{t.test}\NormalTok{ (x,y, }\DataTypeTok{var.equal=}\NormalTok{T)}
\NormalTok{tt}
\end{Highlighting}
\end{Shaded}

\begin{verbatim}
## 
##  Two Sample t-test
## 
## data:  x and y
## t = -3.182, df = 8, p-value = 0.01296
## alternative hypothesis: true difference in means is not equal to 0
## 95 percent confidence interval:
##  -3.1044729 -0.4955271
## sample estimates:
## mean of x mean of y 
##       2.6       4.4
\end{verbatim}

Comprovar que é uma lista

\begin{Shaded}
\begin{Highlighting}[]
\KeywordTok{is.list}\NormalTok{(tt)}
\end{Highlighting}
\end{Shaded}

\begin{verbatim}
## [1] TRUE
\end{verbatim}

\begin{Shaded}
\begin{Highlighting}[]
\KeywordTok{mode}\NormalTok{ (tt)}
\end{Highlighting}
\end{Shaded}

\begin{verbatim}
## [1] "list"
\end{verbatim}

Exibir o componentes da lista

\begin{Shaded}
\begin{Highlighting}[]
\KeywordTok{names}\NormalTok{(tt)}
\end{Highlighting}
\end{Shaded}

\begin{verbatim}
##  [1] "statistic"   "parameter"   "p.value"     "conf.int"    "estimate"   
##  [6] "null.value"  "stderr"      "alternative" "method"      "data.name"
\end{verbatim}

\begin{Shaded}
\begin{Highlighting}[]
\NormalTok{tt}\OperatorTok{$}\NormalTok{conf.int }\CommentTok{#intervalo de confianca }
\end{Highlighting}
\end{Shaded}

\begin{verbatim}
## [1] -3.1044729 -0.4955271
## attr(,"conf.level")
## [1] 0.95
\end{verbatim}

\hypertarget{referuxeancia-1}{%
\section{Referência}\label{referuxeancia-1}}

MELO, M. P.; PETERNELI, L. A. \textbf{Conhecendo o R: Um visão mais que estatística}. Viçosa, MG: UFV, 2013. 222p.

\textbf{Prof.~Paulo Justiniando Ribeiro} \textgreater{}\url{http://www.leg.ufpr.br/~paulojus/}\textless{}

\textbf{Prof.~Adriano Azevedo Filho} \textgreater{}\url{http://rpubs.com/adriano/esalq2012inicial}\textless{}

\textbf{Prof.~Fernando de Pol Mayer} \textgreater{}\url{https://fernandomayer.github.io/ce083-2016-2/}\textless{}

\textbf{Site Interativo Datacamp} \textgreater{}\url{https://www.datacamp.com/}\textless{}

\hypertarget{entrada-de-dados}{%
\chapter{Entrada de dados}\label{entrada-de-dados}}

Este terceiro Capitulo foi baseado no livro \href{https://www.editoraufv.com.br/produto/conhecendo-o-r-uma-visao-mais-que-estatistica/1109294}{\textbf{Conhecendo o R: Um visão mais que estatística}}, e na página do \href{http://www.leg.ufpr.br/~paulojus/}{\textbf{Prof.~Paulo Justiniando Ribeiro}} modificações foram realizadas utilizando outros materiais que se encontram referenciado no final do Capitulo.

O diretorio de trabalho é aquele usado pelo R para gravar, ler, importar e exportar arquivos quando nenhum outro caminho é explicitado.

\hypertarget{onde-os-dados-devem-estar}{%
\section{Onde os dados devem estar?}\label{onde-os-dados-devem-estar}}

Para saber onde os diretorios estão basta digitar o comando \texttt{getwd()}

\begin{Shaded}
\begin{Highlighting}[]
 \KeywordTok{getwd}\NormalTok{() }\CommentTok{#para verificar  diretório de trabalho}
\end{Highlighting}
\end{Shaded}

\begin{verbatim}
## [1] "D:/livro/TudodoRa"
\end{verbatim}

Caso queira alterar o diretorio de trabalho para um outro qualquer digite o comando \texttt{setwd()}

\begin{Shaded}
\begin{Highlighting}[]
\KeywordTok{setwd}\NormalTok{(}\StringTok{"C:/R_Curso"}\NormalTok{) }\CommentTok{#para  altear o diretório de trabalho}
\end{Highlighting}
\end{Shaded}

Outra forma de mudar o caminho é com o comando:

\begin{Shaded}
\begin{Highlighting}[]
\NormalTok{caminho<-}\KeywordTok{file.choose}\NormalTok{() }\CommentTok{# ou usando as teclhas shift + Crtl + H}
\end{Highlighting}
\end{Shaded}

Este comando irá abrir uma tela para que o usuário navegue nas pastas e escolha o arquivo a ser aberto.

Você pode exibir o conteudo do diretório com o comando \texttt{dir()}

\begin{Shaded}
\begin{Highlighting}[]
\KeywordTok{dir}\NormalTok{()}
\end{Highlighting}
\end{Shaded}

\begin{verbatim}
##  [1] "_bookdown.yml"      "_bookdown_files"    "_output.yml"       
##  [4] "01-intro.Rmd"       "02-literature.Rmd"  "04-Estrutura.Rmd"  
##  [7] "05-Figura.Rmd"      "07-Figura_2.Rmd"    "08-teste_esta.Rmd" 
## [10] "09-teste_esta2.Rmd" "10-references.Rmd"  "book.bib"          
## [13] "docs"               "index.Rmd"          "LICENSE"           
## [16] "meu grafico.bmp"    "meu grafico.jpg"    "meu grafico.png"   
## [19] "meu grafico.ps"     "meu grafico.wmf"    "meugráfico.pdf"    
## [22] "packages.bib"       "preamble.tex"       "README.md"         
## [25] "search_index.json"  "style.css"          "TudodoR.log"       
## [28] "TudodoR.pdf"        "TudodoR.Rmd"        "TudodoR.Rproj"     
## [31] "TudodoR.tex"        "TudodoR_files"
\end{verbatim}

\hypertarget{entrando-com-dados}{%
\section{Entrando com dados}\label{entrando-com-dados}}

O formato mais adequado vai depender do tamanho do conjunto de dados, e se os dados já existem em outro formato para serem importados ou se serão digitados diretamente no R.

A seguir são descritas formas de entrada de dados com indicão de quando cada uma das formas deve ser usada.

\hypertarget{vetores}{%
\subsection{Vetores}\label{vetores}}

Podemos entrar com dados definindo vetores com o comando \texttt{c()}, conforme visto no capítulo 3.

\begin{Shaded}
\begin{Highlighting}[]
\NormalTok{vetor <-}\StringTok{ }\KeywordTok{c}\NormalTok{(}\DecValTok{2}\NormalTok{,}\DecValTok{5}\NormalTok{,}\DecValTok{7}\NormalTok{)}
\end{Highlighting}
\end{Shaded}

Esta forma de entrada de dados é conveniente quando se tem um pequeno número de dados. Quando os dados tem algum elementos repetidos, números sequenciais pode-se usar mecanismos do R para facilitar a entrada dos dados como vetores.

\begin{Shaded}
\begin{Highlighting}[]
\NormalTok{vetor <-}\StringTok{ }\KeywordTok{rep}\NormalTok{(}\KeywordTok{c}\NormalTok{(}\DecValTok{2}\NormalTok{,}\DecValTok{5}\NormalTok{), }\DecValTok{5}\NormalTok{)  }\CommentTok{# cria vetor repetindo 5 vezes 2 e 5 alternadamente}
\NormalTok{vetor}
\end{Highlighting}
\end{Shaded}

\begin{verbatim}
##  [1] 2 5 2 5 2 5 2 5 2 5
\end{verbatim}

\begin{Shaded}
\begin{Highlighting}[]
\NormalTok{vetor <-}\StringTok{ }\KeywordTok{rep}\NormalTok{(}\KeywordTok{c}\NormalTok{(}\DecValTok{5}\NormalTok{,}\DecValTok{8}\NormalTok{), }\DataTypeTok{each=}\DecValTok{3}\NormalTok{)  }\CommentTok{# cria vetor repetindo 3 vezes 5 e depois 8}
\NormalTok{vetor}
\end{Highlighting}
\end{Shaded}

\begin{verbatim}
## [1] 5 5 5 8 8 8
\end{verbatim}

\hypertarget{usando-a-funuxe7uxe3o-scan}{%
\subsection{Usando a função `scan'}\label{usando-a-funuxe7uxe3o-scan}}

Esta função coloca o modo prompt onde o usuário deve digitar cada dado seguido da tecla . Para encerrar a entrada de dados basta digitar duas vezes consecutivas. Veja o seguinte resultado:

\begin{Shaded}
\begin{Highlighting}[]
\NormalTok{y <-}\StringTok{ }\KeywordTok{scan}\NormalTok{()}

\CommentTok{#1: 11}
\CommentTok{#2: 24}
\CommentTok{#3: 35}
\CommentTok{#4: 29}
\CommentTok{#5: 39}
\CommentTok{#6: 47}
\CommentTok{#7:}
\CommentTok{#Read 6 items}

\NormalTok{y}
\end{Highlighting}
\end{Shaded}

\begin{verbatim}
## numeric(0)
\end{verbatim}

\begin{Shaded}
\begin{Highlighting}[]
\CommentTok{#[1] 11 24 35 29 39 47}
\end{Highlighting}
\end{Shaded}

Este formato é mais ágil que o anterior e mais conveniente para digitar vetores longos.

\hypertarget{copiar-e-colar-usando-scan}{%
\subsection{Copiar e colar usando scan()}\label{copiar-e-colar-usando-scan}}

Pode usar o recurso ``copiar e colar'' com o comando \texttt{scan}.
Após copiar os dados (crtl+C), digite no \textbf{prompt}/\textbf{console} o comando \texttt{scan()}, aperte \textgreater ENTER\textless, depois cole o texto e, aperte \textgreater ENTER\textless{} novamente.

\hypertarget{lendo-dados-atravuxe9s-da-uxe1rea-de-transferuxeancia}{%
\subsection{Lendo dados através da área de transferência}\label{lendo-dados-atravuxe9s-da-uxe1rea-de-transferuxeancia}}

Funções como \texttt{scan()}, \texttt{read.table()} e outras podem usadas para ler os dados diretamente da área de transferência passando-se ao \emph{``clipboard''} ao primeiro argumento.

\hypertarget{usando-a-funuxe7uxe3o-edit}{%
\subsection{Usando a função edit}\label{usando-a-funuxe7uxe3o-edit}}

O comando \texttt{edit(data.frame())} abre uma planilha para digitação de dados que são armazanados como data-frames.

\begin{Shaded}
\begin{Highlighting}[]
\NormalTok{dados <-}\StringTok{ }\KeywordTok{edit}\NormalTok{(}\KeywordTok{data.frame}\NormalTok{())}
\end{Highlighting}
\end{Shaded}

\begin{figure}
\centering
\includegraphics{https://www.dropbox.com/s/cbsqhtze2715m8t/scan1.PNG?dl=1}
\caption{data-frame}
\end{figure}

Se voce precisar abrir novamente planilha com os dados, para fazer modificações e/ou inserir mais dados use o comando \texttt{fix}.

\begin{Shaded}
\begin{Highlighting}[]
\KeywordTok{fix}\NormalTok{(dados)}
\KeywordTok{head}\NormalTok{(dados)}
\end{Highlighting}
\end{Shaded}

\begin{verbatim}
## data frame with 0 columns and 0 rows
\end{verbatim}

\hypertarget{exemplo-1}{%
\subsubsection{Exemplo 1}\label{exemplo-1}}

\begin{Shaded}
\begin{Highlighting}[]
\NormalTok{teste <-}\StringTok{ }\KeywordTok{c}\NormalTok{(}\DecValTok{10}\NormalTok{,}\DecValTok{20}\NormalTok{,}\DecValTok{30}\NormalTok{,}\DecValTok{40}\NormalTok{,}\DecValTok{50}\NormalTok{)}
\NormalTok{teste}
\end{Highlighting}
\end{Shaded}

\begin{verbatim}
## [1] 10 20 30 40 50
\end{verbatim}

Porém houve um erro: o último elemento deveria ser 60 e não 50, você não precisar criar novamente um objeto, use o comando \texttt{edit()}

\begin{Shaded}
\begin{Highlighting}[]
\NormalTok{teste2 <-}\StringTok{ }\KeywordTok{edit}\NormalTok{(teste)}
\end{Highlighting}
\end{Shaded}

\begin{figure}
\centering
\includegraphics{https://www.dropbox.com/s/l6eh9s5f46wr71x/edit2.PNG?dl=1}
\caption{edit}
\end{figure}

\hypertarget{exemplo-2}{%
\subsubsection{Exemplo 2}\label{exemplo-2}}

Com uma planilha com três colunas de dados. Os valores numéricos da coluna poderiam ser importados para o R utilizando-se o mesmo processo ora descrito com o uso do comando \texttt{scan()}. Abra o arquivo . \href{https://www.dropbox.com/s/6504oo4olw34dw9/EVI_Prec.xlsx?dl=1}{EVI-prec.xlsx}.

Uma matrix com os dados poderá ser obtida com o comando \texttt{cbind}

\begin{Shaded}
\begin{Highlighting}[]
\NormalTok{dados <-}\StringTok{ }\KeywordTok{cbind}\NormalTok{(ano, chuva, evi)}
\end{Highlighting}
\end{Shaded}

Os objeto dados é um \textbf{data.frame}

\begin{Shaded}
\begin{Highlighting}[]
\KeywordTok{is.data.frame}\NormalTok{(dados)}
\end{Highlighting}
\end{Shaded}

Transforme para um data.frame com o comando \textbf{as.data.frame}

\begin{Shaded}
\begin{Highlighting}[]
\NormalTok{dados_m <-}\StringTok{ }\KeywordTok{as.data.frame}\NormalTok{(dados)}
\end{Highlighting}
\end{Shaded}

Poderia usar o comando data.frame direto

\begin{Shaded}
\begin{Highlighting}[]
\NormalTok{dados=}\KeywordTok{data.frame}\NormalTok{ (ano, chuva, evi)}
\end{Highlighting}
\end{Shaded}

\hypertarget{lendo-dados-de-um-arquivo-texto}{%
\subsection{Lendo dados de um arquivo texto}\label{lendo-dados-de-um-arquivo-texto}}

É muito importante ter os dados tabulados em um arquivo-texto ou em outros formatos que permitem a conversão para dados texto. O comando \texttt{read.table\ ()} é extremamente útil por ler dados de um arquivo-texto no formato de um \textbf{data.frame}

Usando o Comando \texttt{read.table\ ()}

\hypertarget{exemplo-1-1}{%
\subsubsection{Exemplo 1}\label{exemplo-1-1}}

Como primeiro exemplo considere importar para o R os dados do arquivo texto \href{https://www.dropbox.com/s/m7jivbbggei5y0x/exemplo1.txt?dl=1}{exemplo1.txt}.

\begin{Shaded}
\begin{Highlighting}[]
\NormalTok{ex01 <-}\StringTok{ }\KeywordTok{read.table}\NormalTok{(}\StringTok{"exemplo1.txt"}\NormalTok{) }

\CommentTok{#Use os comandos}
\NormalTok{  ex01}
  \KeywordTok{class}\NormalTok{(ex01)}
  \KeywordTok{names}\NormalTok{(ex01)}
  \KeywordTok{dim}\NormalTok{(ex01)}
  \KeywordTok{str}\NormalTok{(ex01)}
  \KeywordTok{head}\NormalTok{(ex01)}
\end{Highlighting}
\end{Shaded}

\hypertarget{exemplo-2-1}{%
\subsubsection{Exemplo 2}\label{exemplo-2-1}}

Como primeiro exemplo considere importar para o R os dados do arquivo de texto \href{https://www.dropbox.com/s/bi4b0j2nnnetc1r/exemplo2.txt?dl=1}{exemplo2.txt}.

\begin{Shaded}
\begin{Highlighting}[]
\NormalTok{ex02 <-}\StringTok{ }\KeywordTok{read.table}\NormalTok{(}\StringTok{"exemplo2.txt"}\NormalTok{) }
\NormalTok{ex02}
\end{Highlighting}
\end{Shaded}

Note que este arquivo difere do anterior em um aspecto: os nomes das variáveis estão na primeira linha. Para que o R considere isto corretamente temos que informá-lo disto com o argumento \emph{head=T}. Portanto para importar este arquivo usamos:

\begin{Shaded}
\begin{Highlighting}[]
\NormalTok{ex02 <-}\StringTok{ }\KeywordTok{read.table}\NormalTok{(}\StringTok{"exemplo02.txt"}\NormalTok{, }\DataTypeTok{head=}\NormalTok{T) }
\NormalTok{ex02}
\end{Highlighting}
\end{Shaded}

\hypertarget{dados-do-tipo-csv}{%
\subsection{Dados do tipo CSV}\label{dados-do-tipo-csv}}

\href{https://www.dropbox.com/s/mv13cmkysw2nizm/exemplo3.csv?dl=1}{Exemplo3.csv}: Vamos utilizar um arquivo de tipo \textbf{CSV}.

\begin{Shaded}
\begin{Highlighting}[]
\NormalTok{ex03 <-}\StringTok{ }\KeywordTok{read.table}\NormalTok{(}\StringTok{"exemplo3.csv."}\NormalTok{, }\DataTypeTok{head=}\NormalTok{T, }\DataTypeTok{sep=}\StringTok{":"}\NormalTok{, }\DataTypeTok{dec=}\StringTok{","}\NormalTok{) }
\NormalTok{ex03}
\end{Highlighting}
\end{Shaded}

Note que este arquivo difere do primeiro em outros aspectos.
\emph{read.table.}

\begin{Shaded}
\begin{Highlighting}[]
\NormalTok{ex03 <-}\StringTok{ }\KeywordTok{read.table}\NormalTok{(       }\CommentTok{# lê dados de um arquivo texto}
  \StringTok{"exemplo3.csv"}\NormalTok{,         }\CommentTok{# nome do arquivo ou o caminho c:/R.exemplo3.csv}
  \DataTypeTok{head=}\NormalTok{T,                 }\CommentTok{# primeira linha ? cabe?alho}
  \DataTypeTok{sep=}\StringTok{":"}\NormalTok{,                }\CommentTok{# separador de coluna }
  \DataTypeTok{dec=}\StringTok{","}\NormalTok{)                }\CommentTok{# virgula como separador}
\NormalTok{ex03                      }\CommentTok{# exibe o objeto}
\end{Highlighting}
\end{Shaded}

1.\textbf{sep}: caractere utilizado para separação dos campos e valores. Normalmente é utilizado o ponto e virgula (;)

1.\textbf{dec}: caractere utilizado para separar as casas decimais. Normalmente ponto (.) ou virgula (,).

1.\textbf{header}: TRUE, assume que a primeira linha da tabela contêm rotulos das variáveis. `FALSE', assume que os dados se iniciam na primeira linha.

\hypertarget{a-seguir-listamos-algumas-destas-funuxe7uxf5es}{%
\subsection{A seguir listamos algumas destas funções:}\label{a-seguir-listamos-algumas-destas-funuxe7uxf5es}}

\begin{enumerate}
\def\labelenumi{\arabic{enumi}.}
\tightlist
\item
  \emph{read.dbf()} para arquivos DBASE
\item
  \emph{read.epiinfo()} para arquivos .REC do Epi-Info
\item
  \emph{read.mtp()} para arquivos ``Minitab Portable Worksheet''
\item
  \emph{read.S()} para arquivos do S-PLUS, e restore.data() para ``dumps'' do S-PLUS
\item
  \emph{read.spss()} para dados do SPSS
\item
  \emph{read.systat()} para dados do SYSTAT
\item
  \emph{read.dta()} para dados do STATA
\item
  \emph{read.octave()} para dados do OCTAVE (um clone do MATLAB)
\item
  \emph{read.csv}(file, header = TRUE, sep=``,'', dec=``.'')
\item
  \emph{read.csv2}(file, header = TRUE, sep=``;'', dec=``,'')
\item
  \emph{read.delim}(file, header = TRUE, sep=``\t", dec=''.")
\item
  \emph{read.delim2}(file, header = TRUE, sep=``\t", dec='',")
\end{enumerate}

\hypertarget{lendo-dados-disponuxedveis-na-web}{%
\subsection{Lendo dados disponíveis na web}\label{lendo-dados-disponuxedveis-na-web}}

\textbf{Exemplo 4}: As funções permitem ler ainda dados diretamente disponíveis na web. Por exemplo os dados do \href{https://www.dropbox.com/s/m7jivbbggei5y0x/exemplo1.txt?dl=1}{exemplo1.txt} poderiam ser lidos diretamente com o comando a seguir

\hypertarget{lendo-dados-de-uma-planilha-eletruxf4nica}{%
\subsection{Lendo dados de uma planilha eletrônica}\label{lendo-dados-de-uma-planilha-eletruxf4nica}}

Com o \textbf{pacote xlsx} é possivel ler os dados diretamente da planilha eletrônica do Excel.

\begin{Shaded}
\begin{Highlighting}[]
\KeywordTok{install.packages}\NormalTok{(}\StringTok{""}\NormalTok{)}
\KeywordTok{require}\NormalTok{(}\StringTok{"xlsx"}\NormalTok{)}
\end{Highlighting}
\end{Shaded}

O comando \emph{read.xlsx()}, do \textbf{pacote xlsx}, lê o conteúdo de uma planilha eletrônica para o R com a estrutura de dados de um \emph{data.frame}.

\begin{Shaded}
\begin{Highlighting}[]
\NormalTok{dados <-}\StringTok{ }\KeywordTok{read.xlsx}\NormalTok{(}
                    \DataTypeTok{file=}\StringTok{"C:/R/EVI_Prec.xlsx"}\NormalTok{,     }\CommentTok{#comando que lê planilhas}
                    \DataTypeTok{sheetName =} \StringTok{"Conbea"}\NormalTok{,          }\CommentTok{#nome da planilha}
                    \DataTypeTok{h=}\NormalTok{T)                           }\CommentTok{#sem cabeçalho  }
\end{Highlighting}
\end{Shaded}

\hypertarget{exercuxedcios}{%
\subsection{Exercícios}\label{exercuxedcios}}

\begin{enumerate}
\def\labelenumi{\arabic{enumi}.}
\tightlist
\item
  Baixe os seguintes arquivos:

  \begin{itemize}
  \tightlist
  \item
    \href{https://www.dropbox.com/s/uq1n2sv8an2eoan/BanzattoQd1.2.3.txt?dl=1}{BanzattoQd1.2.3.txt}
  \item
    \href{https://www.dropbox.com/s/jjyo8dhyy0qt3ft/BanzattoQd3.2.1.txt?dl=1}{BanzattoQd3.2.1.txt}
  \item
    \href{https://www.dropbox.com/s/yv5clm6qljurzbw/BanzattoQd3.4.1.txt?dl=1}{BanzattoQd3.4.1.txt}
  \end{itemize}
\end{enumerate}

Coloque os arquivos em um local apropriado (de preferncia no mesmo diretorio de trabalho que voce definiu no início da sessão), faça a importação usando a função de sua escolha, e confira a estrutura dos dados com ´str()´.

\hypertarget{salvar-objetos-de-dados}{%
\section{Salvar objetos de dados}\label{salvar-objetos-de-dados}}

Salvar objetos de dados nos formatos \textbf{.txt} ou \textbf{.csv}
função: \textbf{write.table}
sintaxe da função:
\emph{write.table}(x, file, sep="``, dec=''", rownames = T, col.names = T)

Principais argumentos:
1. x - matriz ou data frame
1. file - nome do arquivo ou caminho do arquivo
1. sep - separador da coluna
1. dec - separador deciminal

\hypertarget{outras-funuxe7uxf5es}{%
\subsection{Outras funções}\label{outras-funuxe7uxf5es}}

\texttt{write.csv()}
\texttt{write.csv2()}
\texttt{write.xlsx\ ()}

\textbf{Exemplo:}
write.xlsx(dados,``tabela salva.xlsx'')

\hypertarget{referuxeancia-2}{%
\section{Referência}\label{referuxeancia-2}}

MELO, M. P.; PETERNELI, L. A. \textbf{Conhecendo o R: Um visão mais que estatística}. Viçosa, MG: UFV, 2013. 222p.

\textbf{Prof.~Paulo Justiniando Ribeiro} \textgreater{}\url{http://www.leg.ufpr.br/~paulojus/}\textless{}

\textbf{Prof.~Adriano Azevedo Filho} \textgreater{}\url{http://rpubs.com/adriano/esalq2012inicial}\textless{}

\textbf{Prof.~Fernando de Pol Mayer} \textgreater{}\url{https://fernandomayer.github.io/ce083-2016-2/}\textless{}

\hypertarget{criando-gruxe1ficos-com-o-r}{%
\chapter{Criando Gráficos com o R}\label{criando-gruxe1ficos-com-o-r}}

Este capitulo foi baseado nos livros

\begin{itemize}
\item
  \href{https://www.editoraufv.com.br/produto/conhecendo-o-r-uma-visao-mais-que-estatistica/1109294}{\textbf{Conhecendo o R: Um visão mais que estatística}}
\item
  AQUINO, J. A. \textbf{R para cientistas sociais}. - Ilhéus, BA: EDITUS, 2014. 157.
\item
  ANJOS, A. \textbf{Análise gráfica com uso do R}. Apostila. Dep. de Estatistica da UFPR, 2016. 127p.
\end{itemize}

Sites

\begin{itemize}
\item
  \url{https://www.statmethods.net/index.html}
\item
  \url{http://curso-r.github.io/index.html} PET Estatística UFPR (2016). \textbf{labestData: Biblioteca de Dados para Aprendizado de Estatística}. R package version x.y-z.w.
\item
  \url{https://www.statmethods.net/index.html}
\end{itemize}

Modificações foram realizadas utilizando outros materiais que se encontram referenciado no final do Capitulo.

O R é uma poderosa ferramenta no que diz respeito à confeção de gráficos. Iremos abordar três categorias de comandos gráficos, com o uso do pacote báscico do R o \emph{graphics". Alguns pacotes foram desenvolvidos especialmente para manipulação de gráficos, como
}lattice\emph{, }ggplot2\emph{, }ggobi* e \emph{rgl}.

O R possui diferentes funções geradoras de gráficos, e essas são classificados como:

\begin{itemize}
\item
  \emph{Funções gráficas de alto nível}: criam novos gráficos na janela, definindo eixos, título, etc. Exemplos: \emph{plot, hist, image, contour, persp etc}.
\item
  \emph{Funções gráficas de baixo nível}: permitem adicionar novas informações em gráficos já criados, como novos dados, linhas etc. Exemplos: \emph{points, lines, abline,} \emph{polygon, legend etc}.
\item
  \emph{Funções gráficas iterativas}: permitem retirar ou adicionar informações aos gráficos já existentes, usando por exemplo o cursor do mouse. Exemplos: \emph{locator e identify}.
\end{itemize}

\hypertarget{exemplos-de-gruxe1ficos-com-o-r}{%
\section{Exemplos de gráficos com o R}\label{exemplos-de-gruxe1ficos-com-o-r}}

Você pode ver alguns exemplos de gráficos que podem ser criados no R com os seguintes comandos:

\begin{Shaded}
\begin{Highlighting}[]
\KeywordTok{demo}\NormalTok{(image)}
\end{Highlighting}
\end{Shaded}

\begin{verbatim}
## 
## 
##  demo(image)
##  ---- ~~~~~
## 
## > #  Copyright (C) 1997-2009 The R Core Team
## > 
## > require(datasets)
## 
## > require(grDevices); require(graphics)
## 
## > x <- 10*(1:nrow(volcano)); x.at <- seq(100, 800, by=100)
## 
## > y <- 10*(1:ncol(volcano)); y.at <- seq(100, 600, by=100)
## 
## >                    # Using Terrain Colors
## > 
## > image(x, y, volcano, col=terrain.colors(100),axes=FALSE)
\end{verbatim}

\includegraphics{TudodoR_files/figure-latex/unnamed-chunk-145-1.pdf}

\begin{verbatim}
## 
## > contour(x, y, volcano, levels=seq(90, 200, by=5), add=TRUE, col="brown")
## 
## > axis(1, at=x.at)
## 
## > axis(2, at=y.at)
## 
## > box()
## 
## > title(main="Maunga Whau Volcano", sub = "col=terrain.colors(100)", font.main=4)
## 
## >                    # Using Heat Colors
## > 
## > image(x, y, volcano, col=heat.colors(100), axes=FALSE)
\end{verbatim}

\includegraphics{TudodoR_files/figure-latex/unnamed-chunk-145-2.pdf}

\begin{verbatim}
## 
## > contour(x, y, volcano, levels=seq(90, 200, by=5), add=TRUE, col="brown")
## 
## > axis(1, at=x.at)
## 
## > axis(2, at=y.at)
## 
## > box()
## 
## > title(main="Maunga Whau Volcano", sub = "col=heat.colors(100)", font.main=4)
## 
## >                    # Using Gray Scale
## > 
## > image(x, y, volcano, col=gray(100:200/200), axes=FALSE)
\end{verbatim}

\includegraphics{TudodoR_files/figure-latex/unnamed-chunk-145-3.pdf}

\begin{verbatim}
## 
## > contour(x, y, volcano, levels=seq(90, 200, by=5), add=TRUE, col="black")
## 
## > axis(1, at=x.at)
## 
## > axis(2, at=y.at)
## 
## > box()
## 
## > title(main="Maunga Whau Volcano \n col=gray(100:200/200)", font.main=4)
## 
## > ## Filled Contours are even nicer sometimes :
## > example(filled.contour)
## 
## flld.c> require("grDevices") # for colours
## 
## flld.c> filled.contour(volcano, asp = 1) # simple
\end{verbatim}

\includegraphics{TudodoR_files/figure-latex/unnamed-chunk-145-4.pdf}

\begin{verbatim}
## 
## flld.c> x <- 10*1:nrow(volcano)
## 
## flld.c> y <- 10*1:ncol(volcano)
## 
## flld.c> filled.contour(x, y, volcano, color = function(n) hcl.colors(n, "terrain"),
## flld.c+     plot.title = title(main = "The Topography of Maunga Whau",
## flld.c+     xlab = "Meters North", ylab = "Meters West"),
## flld.c+     plot.axes = { axis(1, seq(100, 800, by = 100))
## flld.c+                   axis(2, seq(100, 600, by = 100)) },
## flld.c+     key.title = title(main = "Height\n(meters)"),
## flld.c+     key.axes = axis(4, seq(90, 190, by = 10)))  # maybe also asp = 1
\end{verbatim}

\includegraphics{TudodoR_files/figure-latex/unnamed-chunk-145-5.pdf}

\begin{verbatim}
## 
## flld.c> mtext(paste("filled.contour(.) from", R.version.string),
## flld.c+       side = 1, line = 4, adj = 1, cex = .66)
## 
## flld.c> # Annotating a filled contour plot
## flld.c> a <- expand.grid(1:20, 1:20)
## 
## flld.c> b <- matrix(a[,1] + a[,2], 20)
## 
## flld.c> filled.contour(x = 1:20, y = 1:20, z = b,
## flld.c+                plot.axes = { axis(1); axis(2); points(10, 10) })
\end{verbatim}

\includegraphics{TudodoR_files/figure-latex/unnamed-chunk-145-6.pdf}

\begin{verbatim}
## 
## flld.c> ## Persian Rug Art:
## flld.c> x <- y <- seq(-4*pi, 4*pi, len = 27)
## 
## flld.c> r <- sqrt(outer(x^2, y^2, "+"))
## 
## flld.c> filled.contour(cos(r^2)*exp(-r/(2*pi)), axes = FALSE)
\end{verbatim}

\includegraphics{TudodoR_files/figure-latex/unnamed-chunk-145-7.pdf}

\begin{verbatim}
## 
## flld.c> ## rather, the key *should* be labeled:
## flld.c> filled.contour(cos(r^2)*exp(-r/(2*pi)), frame.plot = FALSE,
## flld.c+                plot.axes = {})
\end{verbatim}

\includegraphics{TudodoR_files/figure-latex/unnamed-chunk-145-8.pdf}

\begin{Shaded}
\begin{Highlighting}[]
\KeywordTok{demo}\NormalTok{(persp)}
\end{Highlighting}
\end{Shaded}

\begin{verbatim}
## 
## 
##  demo(persp)
##  ---- ~~~~~
## 
## > ### Demos for  persp()  plots   -- things not in  example(persp)
## > ### -------------------------
## > 
## > require(datasets)
## 
## > require(grDevices); require(graphics)
## 
## > ## (1) The Obligatory Mathematical surface.
## > ##     Rotated sinc function.
## > 
## > x <- seq(-10, 10, length.out = 50)
## 
## > y <- x
## 
## > rotsinc <- function(x,y)
## + {
## +     sinc <- function(x) { y <- sin(x)/x ; y[is.na(y)] <- 1; y }
## +     10 * sinc( sqrt(x^2+y^2) )
## + }
## 
## > sinc.exp <- expression(z == Sinc(sqrt(x^2 + y^2)))
## 
## > z <- outer(x, y, rotsinc)
## 
## > oldpar <- par(bg = "white")
## 
## > persp(x, y, z, theta = 30, phi = 30, expand = 0.5, col = "lightblue")
\end{verbatim}

\includegraphics{TudodoR_files/figure-latex/unnamed-chunk-146-1.pdf}

\begin{verbatim}
## 
## > title(sub=".")## work around persp+plotmath bug
## 
## > title(main = sinc.exp)
## 
## > persp(x, y, z, theta = 30, phi = 30, expand = 0.5, col = "lightblue",
## +       ltheta = 120, shade = 0.75, ticktype = "detailed",
## +       xlab = "X", ylab = "Y", zlab = "Z")
\end{verbatim}

\includegraphics{TudodoR_files/figure-latex/unnamed-chunk-146-2.pdf}

\begin{verbatim}
## 
## > title(sub=".")## work around persp+plotmath bug
## 
## > title(main = sinc.exp)
## 
## > ## (2) Visualizing a simple DEM model
## > 
## > z <- 2 * volcano        # Exaggerate the relief
## 
## > x <- 10 * (1:nrow(z))   # 10 meter spacing (S to N)
## 
## > y <- 10 * (1:ncol(z))   # 10 meter spacing (E to W)
## 
## > persp(x, y, z, theta = 120, phi = 15, scale = FALSE, axes = FALSE)
\end{verbatim}

\includegraphics{TudodoR_files/figure-latex/unnamed-chunk-146-3.pdf}

\begin{verbatim}
## 
## > ## (3) Now something more complex
## > ##     We border the surface, to make it more "slice like"
## > ##     and color the top and sides of the surface differently.
## > 
## > z0 <- min(z) - 20
## 
## > z <- rbind(z0, cbind(z0, z, z0), z0)
## 
## > x <- c(min(x) - 1e-10, x, max(x) + 1e-10)
## 
## > y <- c(min(y) - 1e-10, y, max(y) + 1e-10)
## 
## > fill <- matrix("green3", nrow = nrow(z)-1, ncol = ncol(z)-1)
## 
## > fill[ , i2 <- c(1,ncol(fill))] <- "gray"
## 
## > fill[i1 <- c(1,nrow(fill)) , ] <- "gray"
## 
## > par(bg = "lightblue")
## 
## > persp(x, y, z, theta = 120, phi = 15, col = fill, scale = FALSE, axes = FALSE)
\end{verbatim}

\includegraphics{TudodoR_files/figure-latex/unnamed-chunk-146-4.pdf}

\begin{verbatim}
## 
## > title(main = "Maunga Whau\nOne of 50 Volcanoes in the Auckland Region.",
## +       font.main = 4)
## 
## > par(bg = "slategray")
## 
## > persp(x, y, z, theta = 135, phi = 30, col = fill, scale = FALSE,
## +       ltheta = -120, lphi = 15, shade = 0.65, axes = FALSE)
\end{verbatim}

\includegraphics{TudodoR_files/figure-latex/unnamed-chunk-146-5.pdf}

\begin{verbatim}
## 
## > ## Don't draw the grid lines :  border = NA
## > persp(x, y, z, theta = 135, phi = 30, col = "green3", scale = FALSE,
## +       ltheta = -120, shade = 0.75, border = NA, box = FALSE)
\end{verbatim}

\includegraphics{TudodoR_files/figure-latex/unnamed-chunk-146-6.pdf}

\begin{verbatim}
## 
## > ## `color gradient in the soil' :
## > fcol <- fill ; fcol[] <- terrain.colors(nrow(fcol))
## 
## > persp(x, y, z, theta = 135, phi = 30, col = fcol, scale = FALSE,
## +       ltheta = -120, shade = 0.3, border = NA, box = FALSE)
\end{verbatim}

\includegraphics{TudodoR_files/figure-latex/unnamed-chunk-146-7.pdf}

\begin{verbatim}
## 
## > ## `image like' colors on top :
## > fcol <- fill
## 
## > zi <- volcano[ -1,-1] + volcano[ -1,-61] +
## +            volcano[-87,-1] + volcano[-87,-61]  ## / 4
## 
## > fcol[-i1,-i2] <-
## +     terrain.colors(20)[cut(zi,
## +                            stats::quantile(zi, seq(0,1, length.out = 21)),
## +                            include.lowest = TRUE)]
## 
## > persp(x, y, 2*z, theta = 110, phi = 40, col = fcol, scale = FALSE,
## +       ltheta = -120, shade = 0.4, border = NA, box = FALSE)
\end{verbatim}

\includegraphics{TudodoR_files/figure-latex/unnamed-chunk-146-8.pdf}

\begin{verbatim}
## 
## > ## reset par():
## > par(oldpar)
\end{verbatim}

\begin{Shaded}
\begin{Highlighting}[]
\KeywordTok{demo}\NormalTok{(graphics)}
\end{Highlighting}
\end{Shaded}

\begin{verbatim}
## 
## 
##  demo(graphics)
##  ---- ~~~~~~~~
## 
## > #  Copyright (C) 1997-2009 The R Core Team
## > 
## > require(datasets)
## 
## > require(grDevices); require(graphics)
## 
## > ## Here is some code which illustrates some of the differences between
## > ## R and S graphics capabilities.  Note that colors are generally specified
## > ## by a character string name (taken from the X11 rgb.txt file) and that line
## > ## textures are given similarly.  The parameter "bg" sets the background
## > ## parameter for the plot and there is also an "fg" parameter which sets
## > ## the foreground color.
## > 
## > 
## > x <- stats::rnorm(50)
## 
## > opar <- par(bg = "white")
## 
## > plot(x, ann = FALSE, type = "n")
\end{verbatim}

\includegraphics{TudodoR_files/figure-latex/unnamed-chunk-147-1.pdf}

\begin{verbatim}
## 
## > abline(h = 0, col = gray(.90))
## 
## > lines(x, col = "green4", lty = "dotted")
## 
## > points(x, bg = "limegreen", pch = 21)
## 
## > title(main = "Simple Use of Color In a Plot",
## +       xlab = "Just a Whisper of a Label",
## +       col.main = "blue", col.lab = gray(.8),
## +       cex.main = 1.2, cex.lab = 1.0, font.main = 4, font.lab = 3)
## 
## > ## A little color wheel.    This code just plots equally spaced hues in
## > ## a pie chart.    If you have a cheap SVGA monitor (like me) you will
## > ## probably find that numerically equispaced does not mean visually
## > ## equispaced.  On my display at home, these colors tend to cluster at
## > ## the RGB primaries.  On the other hand on the SGI Indy at work the
## > ## effect is near perfect.
## > 
## > par(bg = "gray")
## 
## > pie(rep(1,24), col = rainbow(24), radius = 0.9)
\end{verbatim}

\includegraphics{TudodoR_files/figure-latex/unnamed-chunk-147-2.pdf}

\begin{verbatim}
## 
## > title(main = "A Sample Color Wheel", cex.main = 1.4, font.main = 3)
## 
## > title(xlab = "(Use this as a test of monitor linearity)",
## +       cex.lab = 0.8, font.lab = 3)
## 
## > ## We have already confessed to having these.  This is just showing off X11
## > ## color names (and the example (from the postscript manual) is pretty "cute".
## > 
## > pie.sales <- c(0.12, 0.3, 0.26, 0.16, 0.04, 0.12)
## 
## > names(pie.sales) <- c("Blueberry", "Cherry",
## +              "Apple", "Boston Cream", "Other", "Vanilla Cream")
## 
## > pie(pie.sales,
## +     col = c("purple","violetred1","green3","cornsilk","cyan","white"))
\end{verbatim}

\includegraphics{TudodoR_files/figure-latex/unnamed-chunk-147-3.pdf}

\begin{verbatim}
## 
## > title(main = "January Pie Sales", cex.main = 1.8, font.main = 1)
## 
## > title(xlab = "(Don't try this at home kids)", cex.lab = 0.8, font.lab = 3)
## 
## > ## Boxplots:  I couldn't resist the capability for filling the "box".
## > ## The use of color seems like a useful addition, it focuses attention
## > ## on the central bulk of the data.
## > 
## > par(bg="cornsilk")
## 
## > n <- 10
## 
## > g <- gl(n, 100, n*100)
## 
## > x <- rnorm(n*100) + sqrt(as.numeric(g))
## 
## > boxplot(split(x,g), col="lavender", notch=TRUE)
\end{verbatim}

\includegraphics{TudodoR_files/figure-latex/unnamed-chunk-147-4.pdf}

\begin{verbatim}
## 
## > title(main="Notched Boxplots", xlab="Group", font.main=4, font.lab=1)
## 
## > ## An example showing how to fill between curves.
## > 
## > par(bg="white")
## 
## > n <- 100
## 
## > x <- c(0,cumsum(rnorm(n)))
## 
## > y <- c(0,cumsum(rnorm(n)))
## 
## > xx <- c(0:n, n:0)
## 
## > yy <- c(x, rev(y))
## 
## > plot(xx, yy, type="n", xlab="Time", ylab="Distance")
\end{verbatim}

\includegraphics{TudodoR_files/figure-latex/unnamed-chunk-147-5.pdf}

\begin{verbatim}
## 
## > polygon(xx, yy, col="gray")
## 
## > title("Distance Between Brownian Motions")
## 
## > ## Colored plot margins, axis labels and titles.    You do need to be
## > ## careful with these kinds of effects.    It's easy to go completely
## > ## over the top and you can end up with your lunch all over the keyboard.
## > ## On the other hand, my market research clients love it.
## > 
## > x <- c(0.00, 0.40, 0.86, 0.85, 0.69, 0.48, 0.54, 1.09, 1.11, 1.73, 2.05, 2.02)
## 
## > par(bg="lightgray")
## 
## > plot(x, type="n", axes=FALSE, ann=FALSE)
\end{verbatim}

\includegraphics{TudodoR_files/figure-latex/unnamed-chunk-147-6.pdf}

\begin{verbatim}
## 
## > usr <- par("usr")
## 
## > rect(usr[1], usr[3], usr[2], usr[4], col="cornsilk", border="black")
## 
## > lines(x, col="blue")
## 
## > points(x, pch=21, bg="lightcyan", cex=1.25)
## 
## > axis(2, col.axis="blue", las=1)
## 
## > axis(1, at=1:12, lab=month.abb, col.axis="blue")
## 
## > box()
## 
## > title(main= "The Level of Interest in R", font.main=4, col.main="red")
## 
## > title(xlab= "1996", col.lab="red")
## 
## > ## A filled histogram, showing how to change the font used for the
## > ## main title without changing the other annotation.
## > 
## > par(bg="cornsilk")
## 
## > x <- rnorm(1000)
## 
## > hist(x, xlim=range(-4, 4, x), col="lavender", main="")
\end{verbatim}

\includegraphics{TudodoR_files/figure-latex/unnamed-chunk-147-7.pdf}

\begin{verbatim}
## 
## > title(main="1000 Normal Random Variates", font.main=3)
## 
## > ## A scatterplot matrix
## > ## The good old Iris data (yet again)
## > 
## > pairs(iris[1:4], main="Edgar Anderson's Iris Data", font.main=4, pch=19)
\end{verbatim}

\includegraphics{TudodoR_files/figure-latex/unnamed-chunk-147-8.pdf}

\begin{verbatim}
## 
## > pairs(iris[1:4], main="Edgar Anderson's Iris Data", pch=21,
## +       bg = c("red", "green3", "blue")[unclass(iris$Species)])
\end{verbatim}

\includegraphics{TudodoR_files/figure-latex/unnamed-chunk-147-9.pdf}

\begin{verbatim}
## 
## > ## Contour plotting
## > ## This produces a topographic map of one of Auckland's many volcanic "peaks".
## > 
## > x <- 10*1:nrow(volcano)
## 
## > y <- 10*1:ncol(volcano)
## 
## > lev <- pretty(range(volcano), 10)
## 
## > par(bg = "lightcyan")
## 
## > pin <- par("pin")
## 
## > xdelta <- diff(range(x))
## 
## > ydelta <- diff(range(y))
## 
## > xscale <- pin[1]/xdelta
## 
## > yscale <- pin[2]/ydelta
## 
## > scale <- min(xscale, yscale)
## 
## > xadd <- 0.5*(pin[1]/scale - xdelta)
## 
## > yadd <- 0.5*(pin[2]/scale - ydelta)
## 
## > plot(numeric(0), numeric(0),
## +      xlim = range(x)+c(-1,1)*xadd, ylim = range(y)+c(-1,1)*yadd,
## +      type = "n", ann = FALSE)
\end{verbatim}

\includegraphics{TudodoR_files/figure-latex/unnamed-chunk-147-10.pdf}

\begin{verbatim}
## 
## > usr <- par("usr")
## 
## > rect(usr[1], usr[3], usr[2], usr[4], col="green3")
## 
## > contour(x, y, volcano, levels = lev, col="yellow", lty="solid", add=TRUE)
## 
## > box()
## 
## > title("A Topographic Map of Maunga Whau", font= 4)
## 
## > title(xlab = "Meters North", ylab = "Meters West", font= 3)
## 
## > mtext("10 Meter Contour Spacing", side=3, line=0.35, outer=FALSE,
## +       at = mean(par("usr")[1:2]), cex=0.7, font=3)
## 
## > ## Conditioning plots
## > 
## > par(bg="cornsilk")
## 
## > coplot(lat ~ long | depth, data = quakes, pch = 21, bg = "green3")
\end{verbatim}

\includegraphics{TudodoR_files/figure-latex/unnamed-chunk-147-11.pdf}

\begin{verbatim}
## 
## > par(opar)
\end{verbatim}

\hypertarget{entrada-de-dados-1}{%
\section{Entrada de dados}\label{entrada-de-dados-1}}

Nesse tópico utlizaremos o arquivo de dados \href{https://www.dropbox.com/s/zg7fyg1iewtji49/dadosfisio.csv?dl=1}{dadosfisio.csv}.

Dados fisico hidrico de 3 solos com textutas diferentes.

\begin{longtable}[]{@{}lllll@{}}
\toprule
Cod. & Solo & Areia & Silte & Argila\tabularnewline
\midrule
\endhead
Z1 & NITOSSOLO & 122 & 121 & 757\tabularnewline
Z2 & LATOSSOLO & 710 & 80 & 210\tabularnewline
Z3 & LATOSSOLO & 892 & 10 & 98\tabularnewline
\bottomrule
\end{longtable}

Ler dados via web.

\begin{Shaded}
\begin{Highlighting}[]
\NormalTok{solo <-}\StringTok{ }\KeywordTok{read.table}\NormalTok{(}\StringTok{"https://www.dropbox.com/s/zg7fyg1iewtji49/dadosfisio.csv?dl=1"}\NormalTok{, }\DataTypeTok{sep =} \StringTok{";"}\NormalTok{, }\DataTypeTok{header =}\NormalTok{ T, }\DataTypeTok{dec =} \StringTok{","}\NormalTok{)}
\end{Highlighting}
\end{Shaded}

Verificar a estrutura de dados.

\begin{Shaded}
\begin{Highlighting}[]
\KeywordTok{str}\NormalTok{(solo)}
\end{Highlighting}
\end{Shaded}

\begin{verbatim}
## 'data.frame':    108 obs. of  16 variables:
##  $ z     : int  1 1 1 1 1 1 1 1 1 1 ...
##  $ x     : int  1 1 1 1 1 1 3 3 3 3 ...
##  $ y     : int  1 3 5 7 9 11 1 3 5 7 ...
##  $ cota  : num  9.15 8.95 8.78 8.59 8.48 8.41 8.93 8.76 8.58 8.48 ...
##  $ ds    : num  1.5 1.47 1.47 1.39 1.38 ...
##  $ cc    : num  0.398 0.382 0.351 0.372 0.356 ...
##  $ ma    : num  0.129 0.153 0.185 0.188 0.208 ...
##  $ ptotal: num  0.526 0.535 0.537 0.561 0.564 ...
##  $ tibo  : num  46.1 19.2 172.8 96 30.7 ...
##  $ tibe  : num  26.8 26.1 113.9 74.8 37.2 ...
##  $ a     : num  926 384 275 1207 151 ...
##  $ b     : num  -0.529 -0.418 -0.131 -0.376 -0.227 ...
##  $ X3    : num  518 243 238 798 118 ...
##  $ X60   : num  153.2 92.7 176.5 335.4 69.9 ...
##  $ X90   : num  106.2 69.4 161.2 258.4 59.8 ...
##  $ X120  : num  73.6 52 147.2 199 51.1 ...
\end{verbatim}

Resumo estatástico da coluna 5 a coluna 8 de todos os solos

\begin{Shaded}
\begin{Highlighting}[]
\KeywordTok{summary}\NormalTok{(solo[}\DecValTok{5}\OperatorTok{:}\DecValTok{8}\NormalTok{])}
\end{Highlighting}
\end{Shaded}

\begin{verbatim}
##        ds              cc               ma               ptotal      
##  Min.   :1.263   Min.   :0.1501   Min.   :0.004834   Min.   :0.2257  
##  1st Qu.:1.500   1st Qu.:0.2505   1st Qu.:0.047689   1st Qu.:0.3090  
##  Median :1.722   Median :0.2712   Median :0.081510   Median :0.3284  
##  Mean   :1.660   Mean   :0.2998   Mean   :0.090675   Mean   :0.3905  
##  3rd Qu.:1.787   3rd Qu.:0.3579   3rd Qu.:0.129955   3rd Qu.:0.5269  
##  Max.   :1.960   Max.   :0.4997   Max.   :0.238551   Max.   :0.6015
\end{verbatim}

Neste exemplo vamos analisar cada solo separadamente usando o comando \texttt{subset()}

\begin{Shaded}
\begin{Highlighting}[]
\NormalTok{solo1 <-}\StringTok{ }\KeywordTok{subset}\NormalTok{(solo, z}\OperatorTok{==}\DecValTok{1}\NormalTok{)}
\NormalTok{solo2 <-}\StringTok{ }\KeywordTok{subset}\NormalTok{(solo, z}\OperatorTok{==}\DecValTok{2}\NormalTok{)}
\NormalTok{solo3 <-}\StringTok{ }\KeywordTok{subset}\NormalTok{(solo, z}\OperatorTok{==}\DecValTok{3}\NormalTok{)}
\end{Highlighting}
\end{Shaded}

\hypertarget{usando-a-funuxe7uxe3o-plot}{%
\section{\texorpdfstring{Usando a função \texttt{plot()}}{Usando a função plot()}}\label{usando-a-funuxe7uxe3o-plot}}

A função \texttt{plot()} inicia um novo gráfico. Em sua forma mais simples a função
recebe valores de coordenadas \emph{ds} (densidade do solo) e \emph{ptotal} (porosidade total do solo) do solo z1.

\begin{Shaded}
\begin{Highlighting}[]
\KeywordTok{plot}\NormalTok{(solo1}\OperatorTok{$}\NormalTok{ds,solo1}\OperatorTok{$}\NormalTok{ptotal)}
\end{Highlighting}
\end{Shaded}

\includegraphics{TudodoR_files/figure-latex/unnamed-chunk-152-1.pdf}

Vamos no gráfico inserir linhas ligando os pontos. Use o argumento *type=``l'' na função \texttt{plot()}

\begin{Shaded}
\begin{Highlighting}[]
\KeywordTok{plot}\NormalTok{(solo1}\OperatorTok{$}\NormalTok{ds,solo1}\OperatorTok{$}\NormalTok{ptotal, }\DataTypeTok{type =} \StringTok{"l"}\NormalTok{)}
\end{Highlighting}
\end{Shaded}

\includegraphics{TudodoR_files/figure-latex/unnamed-chunk-153-1.pdf}

Verifique outras opcões para os gráfico

\begin{itemize}
\tightlist
\item
  \emph{type = ``p''} especifica o tipo de plotagem
\item
  \emph{``p''}: pontos,
\item
  \emph{``l''}: linhas,
\item
  \emph{``b''}: pontos conectados por linhas,
\item
  \emph{``o''}: id. mas as linhas estão acima dos pontos,
\item
  \emph{``h''}: linhas verticais,
\item
  \emph{``s''}: passos, os dados são representados pelo topo das linhas verticais,
\item
  \emph{``S''}: id. mas os dados são representados pela parte inferior das linhas verticais
\end{itemize}

\begin{Shaded}
\begin{Highlighting}[]
\NormalTok{x <-}\StringTok{ }\DecValTok{0}\OperatorTok{:}\DecValTok{12}
\NormalTok{y <-}\StringTok{ }\KeywordTok{sin}\NormalTok{(pi}\OperatorTok{/}\DecValTok{5} \OperatorTok{*}\StringTok{ }\NormalTok{x)}
\NormalTok{op <-}\StringTok{ }\KeywordTok{par}\NormalTok{(}\DataTypeTok{mfrow =} \KeywordTok{c}\NormalTok{(}\DecValTok{3}\NormalTok{,}\DecValTok{3}\NormalTok{), }\DataTypeTok{mar =} \FloatTok{.1}\OperatorTok{+}\StringTok{ }\KeywordTok{c}\NormalTok{(}\DecValTok{2}\NormalTok{,}\DecValTok{2}\NormalTok{,}\DecValTok{3}\NormalTok{,}\DecValTok{1}\NormalTok{))}
\ControlFlowTok{for}\NormalTok{ (tp }\ControlFlowTok{in} \KeywordTok{c}\NormalTok{(}\StringTok{"p"}\NormalTok{,}\StringTok{"l"}\NormalTok{,}\StringTok{"b"}\NormalTok{,  }\StringTok{"c"}\NormalTok{,}\StringTok{"o"}\NormalTok{,}\StringTok{"h"}\NormalTok{,  }\StringTok{"s"}\NormalTok{,}\StringTok{"S"}\NormalTok{,}\StringTok{"n"}\NormalTok{)) \{}
  \KeywordTok{plot}\NormalTok{(y }\OperatorTok{~}\StringTok{ }\NormalTok{x, }\DataTypeTok{type =}\NormalTok{ tp, }\DataTypeTok{main =} \KeywordTok{paste0}\NormalTok{(}\StringTok{"plot(*, type = }\CharTok{\textbackslash{}"}\StringTok{"}\NormalTok{, tp, }\StringTok{"}\CharTok{\textbackslash{}"}\StringTok{)"}\NormalTok{))}
  \ControlFlowTok{if}\NormalTok{(tp }\OperatorTok{==}\StringTok{ "S"}\NormalTok{) \{}
    \KeywordTok{lines}\NormalTok{(x, y, }\DataTypeTok{type =} \StringTok{"s"}\NormalTok{, }\DataTypeTok{col =} \StringTok{"red"}\NormalTok{, }\DataTypeTok{lty =} \DecValTok{2}\NormalTok{)}
    \KeywordTok{mtext}\NormalTok{(}\StringTok{"lines(*, type = }\CharTok{\textbackslash{}"}\StringTok{s}\CharTok{\textbackslash{}"}\StringTok{, ...)"}\NormalTok{, }\DataTypeTok{col =} \StringTok{"red"}\NormalTok{, }\DataTypeTok{cex =} \FloatTok{0.8}\NormalTok{)}
\NormalTok{  \}}
\NormalTok{\}}
\end{Highlighting}
\end{Shaded}

\includegraphics{TudodoR_files/figure-latex/unnamed-chunk-154-1.pdf}

\begin{Shaded}
\begin{Highlighting}[]
\KeywordTok{par}\NormalTok{(op)}
\end{Highlighting}
\end{Shaded}

\hypertarget{mudando-o-padruxe3o-dos-pontos-pch}{%
\subsection{\texorpdfstring{Mudando o padrão dos pontos \texttt{pch=}}{Mudando o padrão dos pontos pch=}}\label{mudando-o-padruxe3o-dos-pontos-pch}}

Pode-se usar diferentes padrões para os pontos usando o argumento \texttt{pch=}.Diferentes tipos de símbolos são associados a diferentes números. Pode-se ainda usar caracteres como o simbolo desejado.
Use a opção \texttt{pch\ =} para especificar simbolos a serem usados ao traçar pontos. Para os simbolos de 21 a 25, especifique a cor da borda \texttt{(col\ =)}.

\begin{Shaded}
\begin{Highlighting}[]
\KeywordTok{plot}\NormalTok{(solo1}\OperatorTok{$}\NormalTok{ds,solo1}\OperatorTok{$}\NormalTok{ptotal, }\DataTypeTok{pch=}\DecValTok{21}\NormalTok{, }\DataTypeTok{ylim =} \KeywordTok{c}\NormalTok{(}\DecValTok{0}\NormalTok{,}\FloatTok{0.6}\NormalTok{), }\DataTypeTok{xlim =} \KeywordTok{c}\NormalTok{(}\DecValTok{1}\NormalTok{,}\DecValTok{2}\NormalTok{))}
\end{Highlighting}
\end{Shaded}

\includegraphics{TudodoR_files/figure-latex/unnamed-chunk-155-1.pdf}

\begin{Shaded}
\begin{Highlighting}[]
\KeywordTok{plot}\NormalTok{(solo2}\OperatorTok{$}\NormalTok{ds,solo2}\OperatorTok{$}\NormalTok{ptotal,}\DataTypeTok{pch=}\DecValTok{2}\NormalTok{, }\DataTypeTok{col=}\StringTok{"blue"}\NormalTok{) }
\end{Highlighting}
\end{Shaded}

\includegraphics{TudodoR_files/figure-latex/unnamed-chunk-155-2.pdf}

\begin{Shaded}
\begin{Highlighting}[]
\KeywordTok{plot}\NormalTok{(solo3}\OperatorTok{$}\NormalTok{ds,solo3}\OperatorTok{$}\NormalTok{ptotal,}\DataTypeTok{pch=}\StringTok{"%"}\NormalTok{)}
\end{Highlighting}
\end{Shaded}

\includegraphics{TudodoR_files/figure-latex/unnamed-chunk-155-3.pdf}

Neste exemplo acima note, que foi adicionado o argumento \texttt{ylim} e \texttt{xlim} eles limitam os valores minimos e maximos:

\begin{Shaded}
\begin{Highlighting}[]
\NormalTok{xlim=}\KeywordTok{c}\NormalTok{(xmin, xmax) ylim=}\KeywordTok{c}\NormalTok{(ymin, ymax)}\ErrorTok{)}
\end{Highlighting}
\end{Shaded}

Veja um exemplo do padrão dos pontos.

\begin{Shaded}
\begin{Highlighting}[]
\KeywordTok{plot}\NormalTok{ (}\DecValTok{0}\OperatorTok{:}\DecValTok{20}\NormalTok{,                         }\CommentTok{#coord. eixo X}
      \KeywordTok{rep}\NormalTok{ (}\DecValTok{0}\NormalTok{,}\DecValTok{21}\NormalTok{),                   }\CommentTok{#coord. eixo y}
      \DataTypeTok{pch =} \DecValTok{0}\OperatorTok{:}\DecValTok{20}\NormalTok{,                   }\CommentTok{#padrão dos pontos variando}
      \DataTypeTok{cex =} \DecValTok{2}\NormalTok{,                      }\CommentTok{#tamanho dos pontos}
      \DataTypeTok{main =} \StringTok{"Padrão dos pontos"}\NormalTok{, }\CommentTok{#Titulo (note o \textbackslash{}n)}
      \DataTypeTok{xlab =} \StringTok{"pch = "}\NormalTok{,              }\CommentTok{#texto do eixo de x}
      \DataTypeTok{ylab =} \StringTok{""}\NormalTok{)                    }\CommentTok{#texto do eixo de y}
\end{Highlighting}
\end{Shaded}

\includegraphics{TudodoR_files/figure-latex/unnamed-chunk-157-1.pdf}

\hypertarget{mudando-as-linhas-lwd-e-lty}{%
\subsection{\texorpdfstring{Mudando as linhas (\texttt{lwd\ e\ lty})}{Mudando as linhas (lwd e lty)}}\label{mudando-as-linhas-lwd-e-lty}}

Você pode alterar linhas usando as seguintes opções. Isso é particularmente útil para linhas de referência, eixos e linhas de ajuste. A largura das linhas pode ser mudada com o argumento \texttt{lwd=}, enquanto os estilos das linhas podem ser modificados com o argumento \texttt{lty=}.

\begin{Shaded}
\begin{Highlighting}[]
\KeywordTok{plot}\NormalTok{(solo3}\OperatorTok{$}\NormalTok{ds,solo3}\OperatorTok{$}\NormalTok{ptotal, }\DataTypeTok{lwd=}\DecValTok{2}\NormalTok{) }\CommentTok{# linha grossa}
\end{Highlighting}
\end{Shaded}

\includegraphics{TudodoR_files/figure-latex/unnamed-chunk-158-1.pdf}

\begin{Shaded}
\begin{Highlighting}[]
\KeywordTok{plot}\NormalTok{(solo2}\OperatorTok{$}\NormalTok{ds,solo2}\OperatorTok{$}\NormalTok{ptotal, }\DataTypeTok{lty=}\DecValTok{2}\NormalTok{) }\CommentTok{#linha interrompida}
\end{Highlighting}
\end{Shaded}

\includegraphics{TudodoR_files/figure-latex/unnamed-chunk-158-2.pdf}

\begin{Shaded}
\begin{Highlighting}[]
\NormalTok{x <-}\StringTok{ }\DecValTok{1}\OperatorTok{:}\DecValTok{9}
\NormalTok{y <-}\StringTok{ }\DecValTok{1}\OperatorTok{:}\DecValTok{9}
  \KeywordTok{plot}\NormalTok{(x, y, }\DataTypeTok{type =} \StringTok{"n"}\NormalTok{)}
    \KeywordTok{lines}\NormalTok{(}\KeywordTok{c}\NormalTok{(}\DecValTok{2}\NormalTok{, }\DecValTok{8}\NormalTok{), }\KeywordTok{c}\NormalTok{(}\DecValTok{8}\NormalTok{, }\DecValTok{8}\NormalTok{), }\DataTypeTok{lwd =} \DecValTok{2}\NormalTok{)}
    \KeywordTok{lines}\NormalTok{(}\KeywordTok{c}\NormalTok{(}\DecValTok{2}\NormalTok{, }\DecValTok{8}\NormalTok{), }\KeywordTok{c}\NormalTok{(}\DecValTok{7}\NormalTok{, }\DecValTok{7}\NormalTok{), }\DataTypeTok{lty =} \DecValTok{2}\NormalTok{, }\DataTypeTok{lwd =} \DecValTok{2}\NormalTok{)}
    \KeywordTok{lines}\NormalTok{(}\KeywordTok{c}\NormalTok{(}\DecValTok{2}\NormalTok{, }\DecValTok{8}\NormalTok{), }\KeywordTok{c}\NormalTok{(}\DecValTok{6}\NormalTok{, }\DecValTok{6}\NormalTok{), }\DataTypeTok{lty =} \DecValTok{3}\NormalTok{, }\DataTypeTok{lwd =} \DecValTok{2}\NormalTok{)}
    \KeywordTok{lines}\NormalTok{(}\KeywordTok{c}\NormalTok{(}\DecValTok{2}\NormalTok{, }\DecValTok{8}\NormalTok{), }\KeywordTok{c}\NormalTok{(}\DecValTok{5}\NormalTok{, }\DecValTok{5}\NormalTok{), }\DataTypeTok{lty =} \DecValTok{4}\NormalTok{, }\DataTypeTok{lwd =} \DecValTok{2}\NormalTok{)}
    \KeywordTok{lines}\NormalTok{(}\KeywordTok{c}\NormalTok{(}\DecValTok{2}\NormalTok{, }\DecValTok{8}\NormalTok{), }\KeywordTok{c}\NormalTok{(}\DecValTok{4}\NormalTok{, }\DecValTok{4}\NormalTok{), }\DataTypeTok{lty =} \DecValTok{5}\NormalTok{, }\DataTypeTok{lwd =} \DecValTok{2}\NormalTok{)}
    \KeywordTok{lines}\NormalTok{(}\KeywordTok{c}\NormalTok{(}\DecValTok{2}\NormalTok{, }\DecValTok{8}\NormalTok{), }\KeywordTok{c}\NormalTok{(}\DecValTok{3}\NormalTok{, }\DecValTok{3}\NormalTok{), }\DataTypeTok{lty =} \DecValTok{6}\NormalTok{, }\DataTypeTok{lwd =} \DecValTok{2}\NormalTok{)}
\end{Highlighting}
\end{Shaded}

\includegraphics{TudodoR_files/figure-latex/unnamed-chunk-159-1.pdf}

\hypertarget{adicionando-linhas-a-um-grafico-de-pontos}{%
\subsection{Adicionando linhas a um grafico de pontos}\label{adicionando-linhas-a-um-grafico-de-pontos}}

A função utilizada para inserir linhas é \texttt{abline()}.
Vamos usar a função \texttt{abline} para inserir uma linha que mostra a média dos dados do eixo Y.
o h é de linha horizontal. Fará uma linha na horizontal que passa pela média de y.

\begin{Shaded}
\begin{Highlighting}[]
\KeywordTok{plot}\NormalTok{(solo3}\OperatorTok{$}\NormalTok{ds,solo3}\OperatorTok{$}\NormalTok{ptotal, }\KeywordTok{abline}\NormalTok{(}\DataTypeTok{h=}\KeywordTok{mean}\NormalTok{(solo3}\OperatorTok{$}\NormalTok{ptotal))) }
\end{Highlighting}
\end{Shaded}

\includegraphics{TudodoR_files/figure-latex/unnamed-chunk-160-1.pdf}

Para passar uma linha que passa pela média de x

\begin{Shaded}
\begin{Highlighting}[]
\KeywordTok{plot}\NormalTok{(solo3}\OperatorTok{$}\NormalTok{ds,solo3}\OperatorTok{$}\NormalTok{ptotal)}
\end{Highlighting}
\end{Shaded}

\includegraphics{TudodoR_files/figure-latex/unnamed-chunk-161-1.pdf}

\begin{Shaded}
\begin{Highlighting}[]
\KeywordTok{plot}\NormalTok{(solo3}\OperatorTok{$}\NormalTok{ds,solo3}\OperatorTok{$}\NormalTok{ptotal, }\KeywordTok{abline}\NormalTok{(}\DataTypeTok{v=}\KeywordTok{mean}\NormalTok{(solo3}\OperatorTok{$}\NormalTok{ds))) }\CommentTok{## o v é de vertical}
\end{Highlighting}
\end{Shaded}

\includegraphics{TudodoR_files/figure-latex/unnamed-chunk-162-1.pdf}

Também é possível inserir as duas linhas ao mesmo tempo.

\begin{Shaded}
\begin{Highlighting}[]
\KeywordTok{plot}\NormalTok{(solo3}\OperatorTok{$}\NormalTok{ds,solo3}\OperatorTok{$}\NormalTok{ptotal, }\KeywordTok{abline}\NormalTok{(}\DataTypeTok{h=}\KeywordTok{mean}\NormalTok{(solo3}\OperatorTok{$}\NormalTok{ptotal), }\DataTypeTok{v=}\KeywordTok{mean}\NormalTok{(solo3}\OperatorTok{$}\NormalTok{ds),}\DataTypeTok{col=}\StringTok{"red"}\NormalTok{))}
\end{Highlighting}
\end{Shaded}

\includegraphics{TudodoR_files/figure-latex/unnamed-chunk-163-1.pdf}

Com cores diferentes

\begin{Shaded}
\begin{Highlighting}[]
\KeywordTok{plot}\NormalTok{(solo3}\OperatorTok{$}\NormalTok{ds,solo3}\OperatorTok{$}\NormalTok{ptotal, }\KeywordTok{abline}\NormalTok{(}\DataTypeTok{h=}\KeywordTok{mean}\NormalTok{(solo3}\OperatorTok{$}\NormalTok{ptotal), }\DataTypeTok{v=}\KeywordTok{mean}\NormalTok{(solo3}\OperatorTok{$}\NormalTok{ds),}\DataTypeTok{col=}\KeywordTok{c}\NormalTok{(}\DecValTok{2}\NormalTok{,}\DecValTok{4}\NormalTok{)))}
\end{Highlighting}
\end{Shaded}

\includegraphics{TudodoR_files/figure-latex/unnamed-chunk-164-1.pdf}

\hypertarget{definindo-o-intervalo-dos-eixos}{%
\subsection{Definindo o intervalo dos eixos}\label{definindo-o-intervalo-dos-eixos}}

Se você quiser preencher um mesmo gráfico com linhas e pontos que possuem diferentes amplitudes como nosso exemplo do solos, deve usar o argumento \texttt{type=n}. Com este argumento um gráfico em branco é criado.

\begin{Shaded}
\begin{Highlighting}[]
\KeywordTok{plot}\NormalTok{(}\KeywordTok{c}\NormalTok{(}\FloatTok{1.55}\NormalTok{,}\DecValTok{2}\NormalTok{),}\KeywordTok{c}\NormalTok{(}\DecValTok{0}\NormalTok{,}\FloatTok{0.6}\NormalTok{),}\DataTypeTok{type=}\StringTok{'n'}\NormalTok{)}
\KeywordTok{points}\NormalTok{(solo3}\OperatorTok{$}\NormalTok{ds,solo3}\OperatorTok{$}\NormalTok{ptotal, }\DataTypeTok{pch=}\DecValTok{2}\NormalTok{)}
\KeywordTok{points}\NormalTok{(solo2}\OperatorTok{$}\NormalTok{ds,solo2}\OperatorTok{$}\NormalTok{ptotal)}
\end{Highlighting}
\end{Shaded}

\includegraphics{TudodoR_files/figure-latex/unnamed-chunk-165-1.pdf}

\hypertarget{personalizando-os-gruxe1ficos}{%
\subsection{Personalizando os gráficos}\label{personalizando-os-gruxe1ficos}}

Alguns parâmetros podem ser usados no intuito de personalizar um gráfico no R.

Exemplo:

\begin{Shaded}
\begin{Highlighting}[]
\KeywordTok{plot}\NormalTok{(solo1}\OperatorTok{$}\NormalTok{ptotal,solo1}\OperatorTok{$}\NormalTok{ds)}
\end{Highlighting}
\end{Shaded}

\includegraphics{TudodoR_files/figure-latex/unnamed-chunk-166-1.pdf}

\begin{Shaded}
\begin{Highlighting}[]
\KeywordTok{plot}\NormalTok{(solo1}\OperatorTok{$}\NormalTok{ptotal,solo1}\OperatorTok{$}\NormalTok{ds,          }\CommentTok{#plota ds e ptotal}
\DataTypeTok{xlab=}\StringTok{"Macroporosdiade (%)"}\NormalTok{,          }\CommentTok{#nomeia o eixo x}
\DataTypeTok{ylab=}\KeywordTok{expression}\NormalTok{(Ds}\OperatorTok{~}\NormalTok{(mg}\OperatorTok{~}\NormalTok{Kg}\OperatorTok{^}\NormalTok{\{}\OperatorTok{-}\DecValTok{1}\NormalTok{\})),    }\CommentTok{#nomeia o eixo y}
\DataTypeTok{main=}\StringTok{"Como personalizar um gráfico"}\NormalTok{, }\CommentTok{#referente ao título}
\DataTypeTok{xlim=}\KeywordTok{c}\NormalTok{(}\FloatTok{0.48}\NormalTok{,}\FloatTok{0.64}\NormalTok{),                   }\CommentTok{#limites do eixo x}
\DataTypeTok{ylim=}\KeywordTok{c}\NormalTok{(}\DecValTok{0}\NormalTok{,}\DecValTok{2}\NormalTok{), }\DataTypeTok{col=}\StringTok{"red"}\NormalTok{,              }\CommentTok{#limites do eixo y}
\DataTypeTok{pch=}\DecValTok{22}\NormalTok{,                              }\CommentTok{#padrão dos pontos}
\DataTypeTok{bg=}\StringTok{"yellow"}\NormalTok{,                         }\CommentTok{#cor de preenchimento}
\DataTypeTok{tcl=}\FloatTok{0.4}\NormalTok{,                             }\CommentTok{#tamanho dos traços dos eixos}
\DataTypeTok{las=}\DecValTok{1}\NormalTok{,                               }\CommentTok{#orientação do texto em y}
\DataTypeTok{cex=}\FloatTok{1.5}\NormalTok{,                             }\CommentTok{#tamanho do objeto do ponto}
\DataTypeTok{bty=}\StringTok{"l"}\NormalTok{,                             }\CommentTok{#altera as bordas}
\KeywordTok{abline}\NormalTok{(}\KeywordTok{lm}\NormalTok{(solo1}\OperatorTok{$}\NormalTok{ds}\OperatorTok{~}\NormalTok{solo1}\OperatorTok{$}\NormalTok{ptotal)))   }\CommentTok{#regressao dos pontos}
\end{Highlighting}
\end{Shaded}

\includegraphics{TudodoR_files/figure-latex/unnamed-chunk-166-2.pdf}

Veja o \texttt{demo(plotmath)} para saber mais sobre anotações em gráficos.

\hypertarget{histogramas}{%
\section{Histogramas}\label{histogramas}}

A função \texttt{hist()} produz um histograma dos dados informados em seu argumento enquanto a função \texttt{barplot()} produz um gráfico de barras.

\begin{Shaded}
\begin{Highlighting}[]
\KeywordTok{hist}\NormalTok{(solo1}\OperatorTok{$}\NormalTok{ds)}
\KeywordTok{rug}\NormalTok{(solo1}\OperatorTok{$}\NormalTok{ds)}
\end{Highlighting}
\end{Shaded}

\includegraphics{TudodoR_files/figure-latex/unnamed-chunk-167-1.pdf}

\hypertarget{personalizando-gruxe1ficos}{%
\subsection{Personalizando gráficos}\label{personalizando-gruxe1ficos}}

Os histogramas criados no R seguem um certo padrão (conhecido como parâmetros
default) que podem ser alterados de acordo com a preferência do usuário. Você pode obter
informações detalhadas desses parâmetros se usar os recursos de ajuda do R.

\begin{Shaded}
\begin{Highlighting}[]
\KeywordTok{hist}\NormalTok{(solo1}\OperatorTok{$}\NormalTok{ds, }\CommentTok{#histograma de ds}
     \DataTypeTok{main=}\StringTok{"Histograma Personalizado}\CharTok{\textbackslash{}n}\StringTok{densidade do solo"}\NormalTok{,}\CommentTok{#título}
     \DataTypeTok{xlab=}\KeywordTok{expression}\NormalTok{(Ds}\OperatorTok{~}\NormalTok{(mg}\OperatorTok{~}\NormalTok{Kg}\OperatorTok{^}\NormalTok{\{}\OperatorTok{-}\DecValTok{1}\NormalTok{\})), }\CommentTok{#texto do eixo das abscissas}
     \DataTypeTok{ylab=}\StringTok{"Probabilidades"}\NormalTok{, }\CommentTok{#texto do eixo das ordenadas}
     \DataTypeTok{xlim=}\KeywordTok{c}\NormalTok{(}\DecValTok{1}\NormalTok{,}\DecValTok{2}\NormalTok{), }\CommentTok{#limites do eixo de x}
     \DataTypeTok{ylim=}\KeywordTok{c}\NormalTok{(}\DecValTok{0}\NormalTok{,}\DecValTok{10}\NormalTok{), }\CommentTok{#limites do eixo y}
     \DataTypeTok{col=}\StringTok{"lightblue"}\NormalTok{, }\CommentTok{#cor das colunas}
     \DataTypeTok{border=}\StringTok{"white"}\NormalTok{, }\CommentTok{#cor das bordas das colunas}
     \DataTypeTok{adj=}\DecValTok{0}\NormalTok{, }\CommentTok{#alinhamento dos textos 0, 0.5 e 1}
     \DataTypeTok{col.axis=}\StringTok{"red"}\NormalTok{) }\CommentTok{#cor do texto nos eixos}
\end{Highlighting}
\end{Shaded}

\includegraphics{TudodoR_files/figure-latex/unnamed-chunk-168-1.pdf}

\hypertarget{gruxe1ficos-de-barras}{%
\section{Gráficos de Barras}\label{gruxe1ficos-de-barras}}

Assemelha-se ao histograma, Porém, nesse caso, os dados referem-se a categoria ou aos tratamentos

\begin{Shaded}
\begin{Highlighting}[]
\KeywordTok{barplot}\NormalTok{(solo}\OperatorTok{$}\NormalTok{ptotal,}\DataTypeTok{names.arg=}\NormalTok{solo}\OperatorTok{$}\NormalTok{z, }\DataTypeTok{horiz =}\NormalTok{ T)}
\end{Highlighting}
\end{Shaded}

\includegraphics{TudodoR_files/figure-latex/unnamed-chunk-169-1.pdf}

\hypertarget{boxplots}{%
\section{Boxplots}\label{boxplots}}

Dados de um experimento visando controle de pulgão (\emph{Aphis gossypii Glover}) em cultura de pepino, instalado em \emph{delineamento inteiramente casualizado} com 6 repetições. A resposta observada foi o número de pulgões após a aplicação de produtos indicados para seu controle.

\begin{Shaded}
\begin{Highlighting}[]
\NormalTok{dados <-}\StringTok{ }\KeywordTok{read.table}\NormalTok{(}\StringTok{"https://www.dropbox.com/s/jjyo8dhyy0qt3ft/BanzattoQd3.2.1.txt?dl=1"}\NormalTok{)}
\KeywordTok{str}\NormalTok{(dados)}
\end{Highlighting}
\end{Shaded}

\begin{verbatim}
## 'data.frame':    30 obs. of  3 variables:
##  $ trat   : Factor w/ 5 levels "Azinfos etilico",..: 5 1 3 4 2 5 1 3 4 2 ...
##  $ rept   : int  1 1 1 1 1 2 2 2 2 2 ...
##  $ pulgoes: int  2370 1282 562 173 193 1687 1527 321 127 71 ...
\end{verbatim}

\emph{trat}
Fator de níveis nominais. Tratamento aplicado para controle do pulgão.

\emph{rept}
Número inteiro que identifica as repetições de cada tratamento.

\emph{pulgões}
Número de pulgões coletados 36 horas após a pulverização dos tratamentos.

Boxplots podem ser criados para variáveis individuais ou para variáveis por grupo. O formato é \texttt{boxplot} \texttt{(\ x\ ,\ data\ =)} , em que \texttt{x} é uma fórmula e \texttt{data\ =} denota o quadro de dados que fornece os dados.

Um exemplo de uma fórmula é \texttt{y\ \textasciitilde{}\ group} onde um boxplot separado para a variável numérica é gerado para cada valor de group.

\begin{Shaded}
\begin{Highlighting}[]
\KeywordTok{x11}\NormalTok{()}
\KeywordTok{boxplot}\NormalTok{(pulgoes}\OperatorTok{~}\NormalTok{trat,              }\CommentTok{#formula do boxplot}
        \DataTypeTok{data =}\NormalTok{ dados,              }\CommentTok{#conjunto de dados}
        \DataTypeTok{main=}\StringTok{"boxplot"}\NormalTok{,            }\CommentTok{#título}
        \DataTypeTok{xlab=}\StringTok{"Controle do pulgão"}\NormalTok{, }\CommentTok{#texto do eixo x }
        \DataTypeTok{ylab=}\StringTok{"Numero de plugões",  #texto do eixo y}
\StringTok{        col=3)                     #cor verde  }
\end{Highlighting}
\end{Shaded}

\includegraphics{TudodoR_files/figure-latex/unnamed-chunk-171-1.pdf}

Adicione \texttt{horizontal\ =\ TRUE} para inverter a orientação do eixo.

\begin{Shaded}
\begin{Highlighting}[]
\KeywordTok{boxplot}\NormalTok{(pulgoes}\OperatorTok{~}\NormalTok{trat,              }\CommentTok{#formula do boxplot}
        \DataTypeTok{data =}\NormalTok{ dados,              }\CommentTok{#conjunto de dados}
        \DataTypeTok{main=}\StringTok{"boxplot"}\NormalTok{,            }\CommentTok{#t?tulo}
        \DataTypeTok{xlab=}\StringTok{"Controle do pulgão"}\NormalTok{, }\CommentTok{#texto do eixo x }
        \DataTypeTok{ylab=}\StringTok{"Numero de plugões",  #texto do eixo y}
\StringTok{        col=3, horizontal = T,     #cor verde  }
\StringTok{        notch=T)                   #teste para mediana}
\end{Highlighting}
\end{Shaded}

\begin{verbatim}
## Warning in bxp(list(stats = structure(c(825, 871, 972.5, 1282, 1527, 44, : some
## notches went outside hinges ('box'): maybe set notch=FALSE
\end{verbatim}

\includegraphics{TudodoR_files/figure-latex/unnamed-chunk-172-1.pdf}

\hypertarget{boxplot-com-fatorial}{%
\subsection{Boxplot com fatorial}\label{boxplot-com-fatorial}}

Boxplot com 2 fatores, com caixas coloridas para facilitar a interpretação.

\textbf{Efeito de Recipientes para duas Espécies de Eucalipto}

Experimento em esquema fatorial 3x2 para estudar o efeito de 3 tipos de recipientes para a produção de mudas de duas espécies de Eucalipto. O experimento foi instalado em delineamento inteiramente casualizado.

\emph{recipie}
São os níveis de recipiente estudados:
- SPP - saco plástico pequeno;
- SPG - saco plástico grande; e
- Lam - laminado.

\emph{especie}
São as espécies de Eucalipto: \emph{Eucalyptus citriodora} e \emph{Eucalyptus grandis}

\emph{rept}
Identifica as repetições de cada combinação dos fatores recipiente e espécie.

\emph{alt}
Altura das mudas aos 80 dias de idade (cm).

Baixar dados via web.

\begin{Shaded}
\begin{Highlighting}[]
\NormalTok{fat <-}\StringTok{ }\KeywordTok{read.table}\NormalTok{(}\StringTok{"https://www.dropbox.com/s/sahc5n80rlkcfx4/BanzattoQd5.2.4.txt?dl=1"}\NormalTok{)}
\KeywordTok{str}\NormalTok{(fat)}
\end{Highlighting}
\end{Shaded}

\begin{verbatim}
## 'data.frame':    24 obs. of  4 variables:
##  $ recipie: Factor w/ 3 levels "Lam","SPG","SPP": 3 3 2 2 1 1 3 3 2 2 ...
##  $ especie: Factor w/ 2 levels "E. citriodora",..: 1 2 1 2 1 2 1 2 1 2 ...
##  $ rept   : int  1 1 1 1 1 1 2 2 2 2 ...
##  $ alt    : num  26.2 24.8 25.7 19.6 22.8 19.8 26 24.6 26.3 21.1 ...
\end{verbatim}

Gerar o gráfico boxpolt com o comando abaixo.

\begin{Shaded}
\begin{Highlighting}[]
\KeywordTok{boxplot}\NormalTok{(fat}\OperatorTok{$}\NormalTok{alt}\OperatorTok{~}\NormalTok{fat}\OperatorTok{$}\NormalTok{recipie}\OperatorTok{*}\NormalTok{especie, }\DataTypeTok{data=}\NormalTok{fat, }\DataTypeTok{notch=}\NormalTok{F, }
        \DataTypeTok{col=}\NormalTok{(}\KeywordTok{c}\NormalTok{(}\StringTok{"gold"}\NormalTok{,}\StringTok{"darkgreen"}\NormalTok{,}\StringTok{"brown"}\NormalTok{)),}
        \DataTypeTok{main=}\StringTok{"Fatorial"}\NormalTok{, }\DataTypeTok{xlab=}\StringTok{"Recipiente e Espécies"}\NormalTok{,}
        \DataTypeTok{ylab=}\StringTok{"Altura de plantas (cm)"}\NormalTok{)}
\end{Highlighting}
\end{Shaded}

\includegraphics{TudodoR_files/figure-latex/unnamed-chunk-174-1.pdf}

\hypertarget{cores}{%
\section{Cores}\label{cores}}

Gráficos em preto e branco são bons na maioria dos casos, mas cores podem ser mudadas usando \texttt{col="red"} (escrevendo o nome da cor) ou \texttt{col=2} (usando números).
O comando abaixo mostra os números que especificam algumas cores.

\begin{Shaded}
\begin{Highlighting}[]
\KeywordTok{pie}\NormalTok{(}\KeywordTok{rep}\NormalTok{(}\DecValTok{1}\NormalTok{,}\DecValTok{30}\NormalTok{),}\DataTypeTok{col=}\KeywordTok{rainbow}\NormalTok{(}\DecValTok{30}\NormalTok{))}
\end{Highlighting}
\end{Shaded}

\includegraphics{TudodoR_files/figure-latex/unnamed-chunk-175-1.pdf}

Veja sua tabela de cores executando o script \href{https://www.dropbox.com/s/e9a27z97buqjovz/paletadecores.R?dl=1}{paletedecores.R}.

Podemos também criar cores personalizadas usando a função do \texttt{rgb()}, que recebe como argumentos as quantidades de vermelho \emph{(red)}, verde \emph{(green)} e azul \emph{(blue)} e, opcionalmente, o grau de opacidade (alpha). Os valores devem ser números reais entre 0 e 1.

Exemplos:

\begin{Shaded}
\begin{Highlighting}[]
\NormalTok{goiaba <-}\StringTok{ }\KeywordTok{rgb}\NormalTok{(}\FloatTok{0.94}\NormalTok{, }\FloatTok{0.41}\NormalTok{, }\FloatTok{0.40}\NormalTok{)}
\NormalTok{goiaba.semitrans <-}\StringTok{ }\KeywordTok{rgb}\NormalTok{(}\FloatTok{0.94}\NormalTok{, }\FloatTok{0.41}\NormalTok{, }\FloatTok{0.40}\NormalTok{, }\DataTypeTok{alpha =} \FloatTok{0.5}\NormalTok{)}
\NormalTok{vitamina <-}\StringTok{ }\KeywordTok{rgb}\NormalTok{(}\DataTypeTok{red =} \KeywordTok{c}\NormalTok{(}\FloatTok{0.87}\NormalTok{, }\FloatTok{0.70}\NormalTok{), }\DataTypeTok{green =} \KeywordTok{c}\NormalTok{(}\FloatTok{0.83}\NormalTok{, }\FloatTok{0.77}\NormalTok{),}
\DataTypeTok{blue =} \KeywordTok{c}\NormalTok{(}\FloatTok{0.71}\NormalTok{, }\FloatTok{0.30}\NormalTok{), }\DataTypeTok{names =} \KeywordTok{c}\NormalTok{(}\StringTok{"leite"}\NormalTok{, }\StringTok{"abacate"}\NormalTok{))}
\end{Highlighting}
\end{Shaded}

\hypertarget{interagindo-com-a-janela-gruxe1fica}{%
\section{Interagindo com a Janela gráfica}\label{interagindo-com-a-janela-gruxe1fica}}

Poderemos com o mouse marcar o ponte desejado usando a função \texttt{identify\ ()}

\begin{Shaded}
\begin{Highlighting}[]
\KeywordTok{plot}\NormalTok{(solo1}\OperatorTok{$}\NormalTok{ds}\OperatorTok{~}\NormalTok{solo1}\OperatorTok{$}\NormalTok{ptotal)}
\KeywordTok{identify}\NormalTok{(solo1}\OperatorTok{$}\NormalTok{ds,}\DataTypeTok{n=}\DecValTok{1}\NormalTok{)}
\end{Highlighting}
\end{Shaded}

\includegraphics{TudodoR_files/figure-latex/unnamed-chunk-177-1.pdf}

\begin{verbatim}
## integer(0)
\end{verbatim}

\hypertarget{texto-e-tamanho-do-suxedmbolo}{%
\section{Texto e tamanho do símbolo}\label{texto-e-tamanho-do-suxedmbolo}}

As seguintes opções podem ser usadas para controlar o tamanho do texto e do símbolo em gráficos.

\texttt{cex} número que indica o valor pelo qual o texto e os símbolos de plotagem devem ser dimensionados em relação ao padrão.
\emph{1 = padrão, 1,5 é 50\% maior, 0,5 é 50\% menor, etc.}

\hypertarget{visualizar-vuxe1rios-gruxe1ficos}{%
\section{Visualizar vários gráficos}\label{visualizar-vuxe1rios-gruxe1ficos}}

\begin{Shaded}
\begin{Highlighting}[]
\KeywordTok{x11}\NormalTok{()}
\KeywordTok{boxplot}\NormalTok{(pulgoes}\OperatorTok{~}\NormalTok{trat,              }\CommentTok{#formula do boxplot}
        \DataTypeTok{data =}\NormalTok{ dados,              }\CommentTok{#conjunto de dados}
        \DataTypeTok{main=}\StringTok{"boxplot"}\NormalTok{,            }\CommentTok{#título}
        \DataTypeTok{xlab=}\StringTok{"Controle do pulgão"}\NormalTok{, }\CommentTok{#texto do eixo x }
        \DataTypeTok{ylab=}\StringTok{"Numero de plugões",  #texto do eixo y}
\StringTok{        col=3,                     #cor verde  }
\StringTok{        notch=F)                   #teste para mediana}
\end{Highlighting}
\end{Shaded}

\includegraphics{TudodoR_files/figure-latex/unnamed-chunk-179-1.pdf}

\hypertarget{varios-gruxe1ficos-na-mesma-janela-gruxe1fica}{%
\subsection{Varios gráficos na mesma janela gráfica}\label{varios-gruxe1ficos-na-mesma-janela-gruxe1fica}}

Você pode dar instruções para o programa mostrar diversos gráficos pequenos em uma mesma janela ao invês de um apenas. Para isto use a função \texttt{par()}.

\textbf{Exemplo 1}

\begin{Shaded}
\begin{Highlighting}[]
\KeywordTok{par}\NormalTok{(}\DataTypeTok{mfrow =} \KeywordTok{c}\NormalTok{(}\DecValTok{2}\NormalTok{,}\DecValTok{2}\NormalTok{)) }\CommentTok{#2 linhas e 2 colunas}
\KeywordTok{plot}\NormalTok{(solo1}\OperatorTok{$}\NormalTok{ptotal,solo1}\OperatorTok{$}\NormalTok{ds)}
\KeywordTok{boxplot}\NormalTok{(solo1}\OperatorTok{$}\NormalTok{ds,solo2}\OperatorTok{$}\NormalTok{ds, solo3}\OperatorTok{$}\NormalTok{ds)}
\KeywordTok{hist}\NormalTok{(solo}\OperatorTok{$}\NormalTok{ptotal)}
\KeywordTok{plot}\NormalTok{(solo}\OperatorTok{$}\NormalTok{ptotal,solo}\OperatorTok{$}\NormalTok{ds)}
\end{Highlighting}
\end{Shaded}

\includegraphics{TudodoR_files/figure-latex/unnamed-chunk-180-1.pdf}

\textbf{Exemplo 2}

\begin{Shaded}
\begin{Highlighting}[]
\KeywordTok{par}\NormalTok{(}\DataTypeTok{mfrow =} \KeywordTok{c}\NormalTok{(}\DecValTok{2}\NormalTok{,}\DecValTok{3}\NormalTok{))}
\KeywordTok{pairs}\NormalTok{(solo)}
\end{Highlighting}
\end{Shaded}

\includegraphics{TudodoR_files/figure-latex/unnamed-chunk-181-1.pdf}

\begin{Shaded}
\begin{Highlighting}[]
\KeywordTok{hist}\NormalTok{(solo}\OperatorTok{$}\NormalTok{ds)}
\KeywordTok{plot}\NormalTok{(solo}\OperatorTok{$}\NormalTok{ds, }\DataTypeTok{col=}\NormalTok{solo}\OperatorTok{$}\NormalTok{z)}
\KeywordTok{plot}\NormalTok{(}\KeywordTok{density}\NormalTok{(solo}\OperatorTok{$}\NormalTok{ds))}
\end{Highlighting}
\end{Shaded}

\includegraphics{TudodoR_files/figure-latex/unnamed-chunk-181-2.pdf}

\hypertarget{salvando-gruxe1ficos}{%
\section{Salvando gráficos}\label{salvando-gruxe1ficos}}

Você pode salvar o gráfico em vários formatos no menu
\emph{Arquivo -\textgreater{} Salvar como}.

Você também pode salvar o gráfico via código usando uma das seguintes funções.

\texttt{pdf\ (file\ =\ "meugráfico.pdf")} \#ficheiro PDF

\texttt{win.metafile\ ("meu\ grafico.wmf")} \#metarquivo do windows

\texttt{png\ ("meu\ grafico.png")} \#arquivo png

\texttt{jpeg\ ("meu\ grafico.jpg")} \#arquivo jpeg

\texttt{bmp\ ("meu\ grafico.bmp")} \#arquivo bmp

\texttt{postscript\ ("meu\ grafico.ps")} \#arquivo postscript

\hypertarget{gruxe1ficos-com-ggplot2}{%
\chapter{Gráficos com ggplot2}\label{gruxe1ficos-com-ggplot2}}

Existem muitas maneiras de fazer Gráficos em R, cada um com suas vantagens e desvantagens. O foco aqui está no pacote ggplot2, que é baseado na \emph{Grammar of Graphics} (Gramática dos Gráficos) para descrever os gráficos de dados.

Utilize o codigo abaixo para instalar o pacote ggplot2

\begin{Shaded}
\begin{Highlighting}[]
\KeywordTok{install.packages}\NormalTok{(}\StringTok{"ggplot2"}\NormalTok{)}
\end{Highlighting}
\end{Shaded}

Sempre carregue o pacote antes de utilizá-lo.

\begin{Shaded}
\begin{Highlighting}[]
\KeywordTok{library}\NormalTok{(ggplot2)}
\end{Highlighting}
\end{Shaded}

Utilizaremos o banco de dados:
\href{https://www.dropbox.com/s/zg7fyg1iewtji49/dadosfisio.csv?dl=1}{dadosfisio}

Baixar os dados

\begin{Shaded}
\begin{Highlighting}[]
\NormalTok{fisio <-}\StringTok{ }\KeywordTok{read.csv2}\NormalTok{(}\StringTok{"https://www.dropbox.com/s/zg7fyg1iewtji49/dadosfisio.csv?dl=1"}\NormalTok{)}
\end{Highlighting}
\end{Shaded}

Veja as primeiras linhas.

\begin{Shaded}
\begin{Highlighting}[]
\KeywordTok{head}\NormalTok{(fisio)}
\end{Highlighting}
\end{Shaded}

\begin{verbatim}
##   z x  y cota       ds        cc        ma    ptotal   tibo   tibe         a
## 1 1 1  1 9.15 1.501258 0.3975615 0.1288555 0.5264170  46.08  26.78  926.3955
## 2 1 1  3 8.95 1.474362 0.3818767 0.1530250 0.5349017  19.20  26.10  383.8130
## 3 1 1  5 8.78 1.469118 0.3514075 0.1851484 0.5365559 172.80 113.92  275.3272
## 4 1 1  7 8.59 1.392845 0.3724094 0.1882073 0.5606167  96.00  74.83 1206.7585
## 5 1 1  9 8.48 1.383309 0.3559554 0.2076696 0.5636250  30.72  37.20  151.4032
## 6 1 1 11 8.41 1.417010 0.3144429 0.2385509 0.5529937 151.32 124.52  368.7992
##            b       X3       X60       X90      X120
## 1 -0.5290035 518.0810 153.24778 106.20581  73.60416
## 2 -0.4176008 242.5903  92.74121  69.43243  51.98188
## 3 -0.1307566 238.4857 176.48417 161.19222 147.22529
## 4 -0.3764298 798.0272 335.41915 258.38731 199.04648
## 5 -0.2270424 117.9800  69.94649  59.76121  51.05906
## 6 -0.1549382 311.0754 217.73464 195.56291 175.64890
\end{verbatim}

O código abaixo é um exemplo de um gráfico bem simples, construído a partir das duas principais camadas. O eixo y representa a densidade do solo e ao eixo x a variavel capacidade de campo.

\begin{Shaded}
\begin{Highlighting}[]
\KeywordTok{ggplot}\NormalTok{(}\DataTypeTok{data =}\NormalTok{ fisio, }\KeywordTok{aes}\NormalTok{(}\DataTypeTok{x =}\NormalTok{ ds, }\DataTypeTok{y =}\NormalTok{ cc)) }\OperatorTok{+}
\StringTok{  }\KeywordTok{geom_point}\NormalTok{()}
\end{Highlighting}
\end{Shaded}

\includegraphics{TudodoR_files/figure-latex/unnamed-chunk-186-1.pdf}

Aqui, essas formas geomótricas são pontos, selecionados pela função \texttt{geom\_point()}, gerando, assim, um gráfico de dispersão.

A função \texttt{aes()} vem da palavra \emph{Aesthetics} define a relação entre os dados e cada aspecto visual do gráfico, como qual variavel será representada no eixo x, qual será representada no eixo y, a cor e o tamanho dos componentes com a função \texttt{colour}.

Outro aspecto que pode ser mapeado nesse gráfico é a cor dos pontos.

\begin{Shaded}
\begin{Highlighting}[]
\KeywordTok{ggplot}\NormalTok{(}\DataTypeTok{data =}\NormalTok{ fisio, }\KeywordTok{aes}\NormalTok{(}\DataTypeTok{x =}\NormalTok{ ds, }\DataTypeTok{y =}\NormalTok{ cc, }\DataTypeTok{colour =} \KeywordTok{as.factor}\NormalTok{(z))) }\OperatorTok{+}
\StringTok{  }\KeywordTok{geom_point}\NormalTok{()}
\end{Highlighting}
\end{Shaded}

\includegraphics{TudodoR_files/figure-latex/unnamed-chunk-187-1.pdf}

Agora, a variável \emph{z} (classe de solo) foi mapeada a cor dos pontos, sendo que pontos vermelhos correspondem ao Nitossolo (valor 1) e pontos azuis e verdes os Latossolos. Observe que inserimos a variável \emph{z} como um fator, pois temos interesse apenas nos valores ``1'', ``2'' e ``3''. No entanto, tambem podemos mapear uma variável contínua a cor dos pontos:

\begin{Shaded}
\begin{Highlighting}[]
\KeywordTok{ggplot}\NormalTok{(}\DataTypeTok{data =}\NormalTok{ fisio, }\KeywordTok{aes}\NormalTok{(}\DataTypeTok{x =}\NormalTok{ ds, }\DataTypeTok{y =}\NormalTok{ cc, }\DataTypeTok{colour =}\NormalTok{ ptotal)) }\OperatorTok{+}
\StringTok{  }\KeywordTok{geom_point}\NormalTok{()}
\end{Highlighting}
\end{Shaded}

\includegraphics{TudodoR_files/figure-latex/unnamed-chunk-188-1.pdf}

A porosidade do solo (ptotal), é representado pela tonalidade da cor azul.

Também podemos mapear o tamanho dos pontos a uma variável de interesse.

\begin{Shaded}
\begin{Highlighting}[]
\KeywordTok{ggplot}\NormalTok{(}\DataTypeTok{data =}\NormalTok{ fisio, }\KeywordTok{aes}\NormalTok{(}\DataTypeTok{x =}\NormalTok{ ds, }\DataTypeTok{y =}\NormalTok{ cc, }\DataTypeTok{colour =}\NormalTok{ ptotal, }\DataTypeTok{size =}\NormalTok{ ma)) }\OperatorTok{+}
\StringTok{  }\KeywordTok{geom_point}\NormalTok{()}
\end{Highlighting}
\end{Shaded}

\includegraphics{TudodoR_files/figure-latex/unnamed-chunk-189-1.pdf}

Outros \texttt{geoms} bastante utilizados:

\begin{itemize}
\tightlist
\item
  geom\_line: para retas definidas por pares (x,y)
\item
  geom\_abline: para retas definidas por um intercepto e uma inclinação
\item
  geom\_hline: para retas horizontais
\item
  geom\_boxplot: para boxplots
\item
  geom\_histogram: para histogramas
\item
  geom\_density: para densidades
\item
  geom\_area: para áreas
\item
  geom\_bar: para barras
\end{itemize}

Veja a seguir como é fácil gerar diversos Gráficos diferentes utilizando a mesma estrutura do gráfico de dispersão acima:

\begin{Shaded}
\begin{Highlighting}[]
\KeywordTok{ggplot}\NormalTok{(}\DataTypeTok{data =}\NormalTok{ fisio, }\KeywordTok{aes}\NormalTok{(}\DataTypeTok{x =} \KeywordTok{factor}\NormalTok{(z), }\DataTypeTok{y =}\NormalTok{ ds)) }\OperatorTok{+}
\StringTok{  }\KeywordTok{geom_boxplot}\NormalTok{()}
\end{Highlighting}
\end{Shaded}

\includegraphics{TudodoR_files/figure-latex/unnamed-chunk-190-1.pdf}

\begin{Shaded}
\begin{Highlighting}[]
\NormalTok{gra <-}\StringTok{ }\KeywordTok{ggplot}\NormalTok{(}\DataTypeTok{data =}\NormalTok{ fisio, }\KeywordTok{aes}\NormalTok{(}\DataTypeTok{x =}\NormalTok{ ds)) }
\end{Highlighting}
\end{Shaded}

\begin{Shaded}
\begin{Highlighting}[]
\NormalTok{gra }\OperatorTok{+}\StringTok{  }\KeywordTok{geom_histogram}\NormalTok{()}
\end{Highlighting}
\end{Shaded}

\begin{verbatim}
## `stat_bin()` using `bins = 30`. Pick better value with `binwidth`.
\end{verbatim}

\includegraphics{TudodoR_files/figure-latex/unnamed-chunk-192-1.pdf}

\begin{Shaded}
\begin{Highlighting}[]
\NormalTok{gra }\OperatorTok{+}\StringTok{  }\KeywordTok{geom_histogram}\NormalTok{(}\DataTypeTok{binwidth=}\NormalTok{.}\DecValTok{05}\NormalTok{, }\DataTypeTok{colour=}\StringTok{"black"}\NormalTok{, }\DataTypeTok{fill=}\StringTok{"white"}\NormalTok{)}
\end{Highlighting}
\end{Shaded}

\includegraphics{TudodoR_files/figure-latex/unnamed-chunk-193-1.pdf}

\begin{Shaded}
\begin{Highlighting}[]
\NormalTok{gra }\OperatorTok{+}\StringTok{ }\KeywordTok{geom_density}\NormalTok{() }\OperatorTok{+}\StringTok{ }
\StringTok{  }\KeywordTok{geom_histogram}\NormalTok{ (}\KeywordTok{aes}\NormalTok{(}\DataTypeTok{y=}\NormalTok{..density..),              }\DataTypeTok{binwidth=}\NormalTok{.}\DecValTok{05}\NormalTok{,}
    \DataTypeTok{colour=}\StringTok{"black"}\NormalTok{, }\DataTypeTok{fill=}\StringTok{"white"}\NormalTok{) }\OperatorTok{+}
\StringTok{    }\KeywordTok{geom_density}\NormalTok{(}\DataTypeTok{alpha=}\NormalTok{.}\DecValTok{2}\NormalTok{, }\DataTypeTok{fill=}\StringTok{"#FF6666"}\NormalTok{)}
\end{Highlighting}
\end{Shaded}

\includegraphics{TudodoR_files/figure-latex/unnamed-chunk-194-1.pdf}

\textbf{Exemplo}
Baixar dados via web.

\begin{Shaded}
\begin{Highlighting}[]
\NormalTok{dados <-}\StringTok{ }\KeywordTok{read.table}\NormalTok{(}\StringTok{"https://www.dropbox.com/s/9woiye3ce9twp78/BanzattoQd4.5.2.txt?dl=1"}\NormalTok{)}
\end{Highlighting}
\end{Shaded}

Criar gráficos.

\begin{Shaded}
\begin{Highlighting}[]
\NormalTok{bar <-}\StringTok{ }\KeywordTok{ggplot}\NormalTok{(}\DataTypeTok{data =}\NormalTok{ dados, }\KeywordTok{aes}\NormalTok{(}\DataTypeTok{y =}\NormalTok{ peso, }\DataTypeTok{x =}\NormalTok{ promalin, }\DataTypeTok{fill =} \KeywordTok{factor}\NormalTok{(promalin)))}
\end{Highlighting}
\end{Shaded}

Nestes exemplos, a altura da barra representará o valor em uma coluna do quadro de dados. Isso é feito usando \texttt{stat="identity"} em vez do padrão \texttt{stat="bin"}.

\begin{Shaded}
\begin{Highlighting}[]
\NormalTok{bar }\OperatorTok{+}\StringTok{  }\KeywordTok{geom_bar}\NormalTok{(}\DataTypeTok{stat=}\StringTok{"identity"}\NormalTok{)}
\end{Highlighting}
\end{Shaded}

\includegraphics{TudodoR_files/figure-latex/unnamed-chunk-197-1.pdf}

Gráfico de barras agrupados

\begin{Shaded}
\begin{Highlighting}[]
\NormalTok{bar }\OperatorTok{+}\StringTok{   }\KeywordTok{geom_bar}\NormalTok{(}\DataTypeTok{stat=}\StringTok{"identity"}\NormalTok{, }\DataTypeTok{position=}\KeywordTok{position_dodge}\NormalTok{())}
\end{Highlighting}
\end{Shaded}

\includegraphics{TudodoR_files/figure-latex/unnamed-chunk-198-1.pdf}

Empilhado

\begin{Shaded}
\begin{Highlighting}[]
\NormalTok{bar }\OperatorTok{+}\StringTok{   }\KeywordTok{geom_bar}\NormalTok{(}\DataTypeTok{stat=}\StringTok{"identity"}\NormalTok{, }\DataTypeTok{colour =}\StringTok{"black"}\NormalTok{)}
\end{Highlighting}
\end{Shaded}

\includegraphics{TudodoR_files/figure-latex/unnamed-chunk-199-1.pdf}

\hypertarget{personalizando-os-gruxe1ficos-1}{%
\section{Personalizando os gráficos}\label{personalizando-os-gruxe1ficos-1}}

\hypertarget{cores-1}{%
\subsection{Cores}\label{cores-1}}

O aspecto \texttt{colour} do boxplot, muda a cor do contorno. Para mudar o preenchimento, basta usar o \texttt{fill}.

Usando \texttt{colour}

\begin{Shaded}
\begin{Highlighting}[]
\KeywordTok{ggplot}\NormalTok{(}\DataTypeTok{data =}\NormalTok{ fisio, }\KeywordTok{aes}\NormalTok{(}\DataTypeTok{x =} \KeywordTok{factor}\NormalTok{(z), }\DataTypeTok{y =}\NormalTok{ ds, }\DataTypeTok{colour =} \KeywordTok{factor}\NormalTok{(z))) }\OperatorTok{+}
\StringTok{  }\KeywordTok{geom_boxplot}\NormalTok{()}
\end{Highlighting}
\end{Shaded}

\includegraphics{TudodoR_files/figure-latex/unnamed-chunk-200-1.pdf}

Usando \texttt{fill}

\begin{Shaded}
\begin{Highlighting}[]
\KeywordTok{ggplot}\NormalTok{(}\DataTypeTok{data =}\NormalTok{ fisio, }\KeywordTok{aes}\NormalTok{(}\DataTypeTok{x =} \KeywordTok{factor}\NormalTok{(z), }\DataTypeTok{y =}\NormalTok{ ds, }\DataTypeTok{fill =} \KeywordTok{factor}\NormalTok{(z))) }\OperatorTok{+}
\StringTok{  }\KeywordTok{geom_boxplot}\NormalTok{()}
\end{Highlighting}
\end{Shaded}

\includegraphics{TudodoR_files/figure-latex/unnamed-chunk-201-1.pdf}

Mude a cor dos objetos sem atribuir a uma variavel. Para isso, observe que os aspectos \texttt{colour} e \texttt{fill} são especificados fora do \texttt{aes()}.

\begin{Shaded}
\begin{Highlighting}[]
\KeywordTok{ggplot}\NormalTok{(}\DataTypeTok{data =}\NormalTok{ fisio, }\KeywordTok{aes}\NormalTok{(}\DataTypeTok{x =} \KeywordTok{factor}\NormalTok{(z), }\DataTypeTok{y =}\NormalTok{ ds)) }\OperatorTok{+}
\StringTok{  }\KeywordTok{geom_boxplot}\NormalTok{(}\DataTypeTok{colour =} \StringTok{"darkblue"}\NormalTok{, }\DataTypeTok{fill=} \StringTok{"blue"}\NormalTok{)}
\end{Highlighting}
\end{Shaded}

\includegraphics{TudodoR_files/figure-latex/unnamed-chunk-202-1.pdf}

\hypertarget{eixos}{%
\subsection{Eixos}\label{eixos}}

Para alterar os rotulos dos eixos acrescentamos as funções \texttt{xlab()} ou \texttt{ylab()}.

\begin{Shaded}
\begin{Highlighting}[]
\NormalTok{box <-}\StringTok{ }\KeywordTok{ggplot}\NormalTok{(}\DataTypeTok{data =}\NormalTok{ fisio, }\KeywordTok{aes}\NormalTok{(}\DataTypeTok{x =} \KeywordTok{factor}\NormalTok{(z), }\DataTypeTok{y =}\NormalTok{ ds, }\DataTypeTok{fill =} \KeywordTok{factor}\NormalTok{(z))) }\OperatorTok{+}
\StringTok{  }\KeywordTok{geom_boxplot}\NormalTok{()}\OperatorTok{+}
\StringTok{  }\KeywordTok{xlab}\NormalTok{(}\StringTok{"Classes de solo"}\NormalTok{) }\OperatorTok{+}
\StringTok{  }\KeywordTok{ylab}\NormalTok{(}\KeywordTok{expression}\NormalTok{(}\KeywordTok{paste}\NormalTok{(Densidade}\OperatorTok{~}\NormalTok{do}\OperatorTok{~}\NormalTok{solo,}\StringTok{" g cm "}\OperatorTok{^}\NormalTok{\{}\OperatorTok{-}\DecValTok{3}\NormalTok{\} )))}
\end{Highlighting}
\end{Shaded}

Alterar os limites dos Gráficos usamos as funções \texttt{xlim()} e \texttt{ylim()}.

\begin{Shaded}
\begin{Highlighting}[]
\NormalTok{  box }\OperatorTok{+}\StringTok{ }\KeywordTok{ylim}\NormalTok{ (}\KeywordTok{c}\NormalTok{(}\FloatTok{1.0}\NormalTok{,}\FloatTok{2.0}\NormalTok{))}
\end{Highlighting}
\end{Shaded}

\includegraphics{TudodoR_files/figure-latex/unnamed-chunk-204-1.pdf}

Especifique marcas de escala diretamente

\begin{Shaded}
\begin{Highlighting}[]
\NormalTok{box }\OperatorTok{+}\StringTok{ }\KeywordTok{coord_cartesian}\NormalTok{(}\DataTypeTok{ylim=}\KeywordTok{c}\NormalTok{(}\DecValTok{1}\NormalTok{, }\DecValTok{2}\NormalTok{)) }\OperatorTok{+}\StringTok{ }
\StringTok{    }\KeywordTok{scale_y_continuous}\NormalTok{(}\DataTypeTok{breaks=}\KeywordTok{seq}\NormalTok{(}\DecValTok{0}\NormalTok{, }\DecValTok{2}\NormalTok{, }\FloatTok{0.20}\NormalTok{))  }
\end{Highlighting}
\end{Shaded}

\includegraphics{TudodoR_files/figure-latex/unnamed-chunk-205-1.pdf}

Troque os eixos x e y

\begin{Shaded}
\begin{Highlighting}[]
\NormalTok{box }\OperatorTok{+}
\StringTok{  }\KeywordTok{coord_flip}\NormalTok{()}
\end{Highlighting}
\end{Shaded}

\includegraphics{TudodoR_files/figure-latex/unnamed-chunk-206-1.pdf}

Definir rótulos de marca de escala

\begin{Shaded}
\begin{Highlighting}[]
\NormalTok{box2 <-}\StringTok{ }\NormalTok{box }\OperatorTok{+}
\StringTok{          }\KeywordTok{scale_x_discrete}\NormalTok{(}\DataTypeTok{breaks=}\KeywordTok{c}\NormalTok{(}\StringTok{"1"}\NormalTok{, }\StringTok{"2"}\NormalTok{, }\StringTok{"3"}\NormalTok{),}
            \DataTypeTok{labels=}\KeywordTok{c}\NormalTok{(}\StringTok{"Nitossolo"}\NormalTok{,}\StringTok{"Latossolo"}\NormalTok{, }\StringTok{"Latossolo"}\NormalTok{))}
\end{Highlighting}
\end{Shaded}

\hypertarget{legenda}{%
\subsection{Legenda}\label{legenda}}

Remover legenda para uma estética específica \texttt{(fill)}

\begin{Shaded}
\begin{Highlighting}[]
\NormalTok{box2 }\OperatorTok{+}\StringTok{ }\KeywordTok{guides}\NormalTok{(}\DataTypeTok{fill=}\OtherTok{FALSE}\NormalTok{)}
\end{Highlighting}
\end{Shaded}

\includegraphics{TudodoR_files/figure-latex/unnamed-chunk-208-1.pdf}

Também pode ser feito ao especificar a \texttt{scale}

\begin{Shaded}
\begin{Highlighting}[]
\NormalTok{box2 }\OperatorTok{+}\StringTok{ }\KeywordTok{scale_fill_discrete}\NormalTok{(}\DataTypeTok{guide=}\OtherTok{FALSE}\NormalTok{)}
\end{Highlighting}
\end{Shaded}

\includegraphics{TudodoR_files/figure-latex/unnamed-chunk-209-1.pdf}

Isso remove todas as legendas

\begin{Shaded}
\begin{Highlighting}[]
\NormalTok{box2 }\OperatorTok{+}\StringTok{ }\KeywordTok{theme}\NormalTok{(}\DataTypeTok{legend.position=}\StringTok{"none"}\NormalTok{)}
\end{Highlighting}
\end{Shaded}

\includegraphics{TudodoR_files/figure-latex/unnamed-chunk-210-1.pdf}

Alterando a ordem dos itens na legenda

\begin{Shaded}
\begin{Highlighting}[]
\NormalTok{box2 }\OperatorTok{+}\StringTok{ }\KeywordTok{scale_fill_discrete}\NormalTok{(}\DataTypeTok{breaks=}\KeywordTok{c}\NormalTok{(}\StringTok{"2"}\NormalTok{,}\StringTok{"3"}\NormalTok{,}\StringTok{"1"}\NormalTok{))}
\end{Highlighting}
\end{Shaded}

\includegraphics{TudodoR_files/figure-latex/unnamed-chunk-211-1.pdf}

Modificando o texto de legenda de tétulos e rótulos

\begin{Shaded}
\begin{Highlighting}[]
\NormalTok{box3 <-}\StringTok{ }\NormalTok{box2 }\OperatorTok{+}
\StringTok{        }\KeywordTok{scale_fill_discrete}\NormalTok{(}\DataTypeTok{name=}\StringTok{"Classes}\CharTok{\textbackslash{}n}\StringTok{de solo"}\NormalTok{,}
                          \DataTypeTok{breaks=}\KeywordTok{c}\NormalTok{(}\StringTok{"1"}\NormalTok{, }\StringTok{"2"}\NormalTok{, }\StringTok{"3"}\NormalTok{),}
                          \DataTypeTok{labels=}\KeywordTok{c}\NormalTok{(}\StringTok{"CTI"}\NormalTok{, }\StringTok{"FEI"}\NormalTok{, }\StringTok{"IAPAR"}\NormalTok{))}
\end{Highlighting}
\end{Shaded}

Modificando a aparência do título e dos rótulos da legenda

\begin{Shaded}
\begin{Highlighting}[]
\CommentTok{# Título}
\NormalTok{box3 }\OperatorTok{+}\StringTok{ }\KeywordTok{theme}\NormalTok{(}\DataTypeTok{legend.title =} \KeywordTok{element_text}\NormalTok{(}\DataTypeTok{colour=}\StringTok{"black"}\NormalTok{, }\DataTypeTok{size=}\DecValTok{13}\NormalTok{, }\DataTypeTok{face=}\StringTok{"bold"}\NormalTok{))}
\end{Highlighting}
\end{Shaded}

\includegraphics{TudodoR_files/figure-latex/unnamed-chunk-213-1.pdf}

\begin{Shaded}
\begin{Highlighting}[]
\CommentTok{# Níveis}
\NormalTok{box3 }\OperatorTok{+}\StringTok{ }\KeywordTok{theme}\NormalTok{(}\DataTypeTok{legend.text =} \KeywordTok{element_text}\NormalTok{(}\DataTypeTok{colour=}\StringTok{"black"}\NormalTok{, }\DataTypeTok{size =} \DecValTok{12}\NormalTok{, }\DataTypeTok{face =} \StringTok{"bold"}\NormalTok{))}
\end{Highlighting}
\end{Shaded}

\includegraphics{TudodoR_files/figure-latex/unnamed-chunk-213-2.pdf}

Modificando a caixa de legenda

\begin{Shaded}
\begin{Highlighting}[]
\NormalTok{box3 }\OperatorTok{+}\StringTok{ }\KeywordTok{theme}\NormalTok{(}\DataTypeTok{legend.background =} \KeywordTok{element_rect}\NormalTok{())}
\end{Highlighting}
\end{Shaded}

\includegraphics{TudodoR_files/figure-latex/unnamed-chunk-214-1.pdf}

\begin{Shaded}
\begin{Highlighting}[]
\NormalTok{box3 }\OperatorTok{+}\StringTok{ }\KeywordTok{theme}\NormalTok{(}\DataTypeTok{legend.background =} \KeywordTok{element_rect}\NormalTok{(}\DataTypeTok{fill=}\StringTok{"gray90"}\NormalTok{))}
\end{Highlighting}
\end{Shaded}

\includegraphics{TudodoR_files/figure-latex/unnamed-chunk-214-2.pdf}

Mudando a posição da legenda

\begin{Shaded}
\begin{Highlighting}[]
\NormalTok{box3 }\OperatorTok{+}\StringTok{ }\KeywordTok{theme}\NormalTok{(}\DataTypeTok{legend.position=}\StringTok{"top"}\NormalTok{)}
\end{Highlighting}
\end{Shaded}

\includegraphics{TudodoR_files/figure-latex/unnamed-chunk-215-1.pdf}

Posicione a legenda no gráfico, em que x, y é 0,0 (canto inferior esquerdo) a 1,1 (canto superior direito)

\begin{Shaded}
\begin{Highlighting}[]
\NormalTok{box3 }\OperatorTok{+}\StringTok{ }\KeywordTok{theme}\NormalTok{(}\DataTypeTok{legend.position=}\KeywordTok{c}\NormalTok{(.}\DecValTok{5}\NormalTok{, }\FloatTok{.5}\NormalTok{))}
\end{Highlighting}
\end{Shaded}

\includegraphics{TudodoR_files/figure-latex/unnamed-chunk-216-1.pdf}

Defina o ``ponto de ancoragem'' da legenda (o canto inferior esquerdo é 0,0; o canto superior direito é 1,1)

\begin{Shaded}
\begin{Highlighting}[]
\NormalTok{box3 }\OperatorTok{+}\StringTok{ }\KeywordTok{theme}\NormalTok{(}\DataTypeTok{legend.justification=}\KeywordTok{c}\NormalTok{(}\DecValTok{0}\NormalTok{,}\DecValTok{0}\NormalTok{), }\DataTypeTok{legend.position=}\KeywordTok{c}\NormalTok{(}\DecValTok{0}\NormalTok{,}\DecValTok{0}\NormalTok{))}
\end{Highlighting}
\end{Shaded}

\includegraphics{TudodoR_files/figure-latex/unnamed-chunk-217-1.pdf}

Coloque o canto inferior direito da caixa de legenda no canto inferior direito do gráfico

\begin{Shaded}
\begin{Highlighting}[]
\NormalTok{box3 }\OperatorTok{+}\StringTok{ }\KeywordTok{theme}\NormalTok{(}\DataTypeTok{legend.justification=}\KeywordTok{c}\NormalTok{(}\DecValTok{1}\NormalTok{,}\DecValTok{0}\NormalTok{), }\DataTypeTok{legend.position=}\KeywordTok{c}\NormalTok{(}\DecValTok{1}\NormalTok{,}\DecValTok{0}\NormalTok{))}
\end{Highlighting}
\end{Shaded}

\includegraphics{TudodoR_files/figure-latex/unnamed-chunk-218-1.pdf}

\hypertarget{tuxedtulo}{%
\subsection{Título}\label{tuxedtulo}}

\begin{Shaded}
\begin{Highlighting}[]
\NormalTok{box3 }\OperatorTok{+}\StringTok{ }\KeywordTok{ggtitle}\NormalTok{(}\StringTok{"Variabilidade da densidade do solo}\CharTok{\textbackslash{}n}\StringTok{ em diferentes solos"}\NormalTok{)}
\end{Highlighting}
\end{Shaded}

\includegraphics{TudodoR_files/figure-latex/unnamed-chunk-219-1.pdf}

\begin{Shaded}
\begin{Highlighting}[]
\NormalTok{box3 }\OperatorTok{+}\StringTok{ }\KeywordTok{labs}\NormalTok{(}\DataTypeTok{title=}\StringTok{"Variabilidade da densidade do solo}\CharTok{\textbackslash{}n}\StringTok{ em diferentes solos"}\NormalTok{)}
\end{Highlighting}
\end{Shaded}

\includegraphics{TudodoR_files/figure-latex/unnamed-chunk-219-2.pdf}

\hypertarget{facets}{%
\subsection{Facets}\label{facets}}

Outra funcionalidade muito importante do \textbf{ggplot2} é o uso de \texttt{facets}.
Você quer dividir seus dados por uma ou mais variáveis e plotar os subconjuntos de dados juntos.

\begin{Shaded}
\begin{Highlighting}[]
\KeywordTok{ggplot}\NormalTok{(}\DataTypeTok{data =}\NormalTok{ fisio, }\KeywordTok{aes}\NormalTok{(}\DataTypeTok{x =}\NormalTok{ ds, }\DataTypeTok{y =}\NormalTok{ cc, }\DataTypeTok{colour =} \KeywordTok{as.factor}\NormalTok{(z))) }\OperatorTok{+}
\StringTok{  }\KeywordTok{geom_point}\NormalTok{() }\OperatorTok{+}
\StringTok{  }\KeywordTok{facet_grid}\NormalTok{(z}\OperatorTok{~}\NormalTok{.)}
\end{Highlighting}
\end{Shaded}

\includegraphics{TudodoR_files/figure-latex/unnamed-chunk-220-1.pdf}

Podemos colocar os graficos lado a lado também.

\begin{Shaded}
\begin{Highlighting}[]
\KeywordTok{ggplot}\NormalTok{(}\DataTypeTok{data =}\NormalTok{ fisio, }\KeywordTok{aes}\NormalTok{(}\DataTypeTok{x =}\NormalTok{ ds, }\DataTypeTok{y =}\NormalTok{ cc, }\DataTypeTok{colour =} \KeywordTok{as.factor}\NormalTok{(z))) }\OperatorTok{+}
\StringTok{  }\KeywordTok{geom_point}\NormalTok{() }\OperatorTok{+}
\StringTok{  }\KeywordTok{facet_grid}\NormalTok{(.}\OperatorTok{~}\NormalTok{z)}
\end{Highlighting}
\end{Shaded}

\includegraphics{TudodoR_files/figure-latex/unnamed-chunk-221-1.pdf}

\hypertarget{exemplos}{%
\section{Exemplos}\label{exemplos}}

\hypertarget{regressuxe3o}{%
\subsection{Regressão}\label{regressuxe3o}}

\textbf{Efeito do Gesso no Peso de grãos de feijão}
Estudo sobre o efeito do gesso no peso de grãos de feijo (\emph{Phaseolus vulgaris} L.) feito por Ragazzi (1979). O experimento foi instalado em delineamento inteiramente casualizado e foram estudados 7 n?veis de gesso, de 0 a 300, igualmente espaados em 50 kg ha-1.

Baixar dados

\begin{Shaded}
\begin{Highlighting}[]
\NormalTok{dados <-}\StringTok{ }\KeywordTok{read.table}\NormalTok{(}\StringTok{"https://www.dropbox.com/s/r6jz7mrktbgnbnx/BanzattoQd7.2.1.txt?dl=1"}\NormalTok{)}
\end{Highlighting}
\end{Shaded}

Verificar a estrutura dos dados

\begin{Shaded}
\begin{Highlighting}[]
\KeywordTok{str}\NormalTok{(dados)}
\end{Highlighting}
\end{Shaded}

\begin{verbatim}
## 'data.frame':    28 obs. of  3 variables:
##  $ gesso: int  0 0 0 0 50 50 50 50 100 100 ...
##  $ rept : int  1 2 3 4 1 2 3 4 1 2 ...
##  $ peso : num  135 140 148 132 162 ...
\end{verbatim}

Analise de regressão

\begin{Shaded}
\begin{Highlighting}[]
\NormalTok{model <-}\StringTok{ }\KeywordTok{lm}\NormalTok{(  gesso }\OperatorTok{~}\StringTok{ }\NormalTok{peso, dados)}

\KeywordTok{summary}\NormalTok{(model)}
\end{Highlighting}
\end{Shaded}

\begin{verbatim}
## 
## Call:
## lm(formula = gesso ~ peso, data = dados)
## 
## Residuals:
##     Min      1Q  Median      3Q     Max 
## -120.41  -70.79  -31.57   74.22  179.24 
## 
## Coefficients:
##             Estimate Std. Error t value Pr(>|t|)  
## (Intercept) -451.935    282.012  -1.603    0.121  
## peso           3.849      1.799   2.139    0.042 *
## ---
## Signif. codes:  0 '***' 0.001 '**' 0.01 '*' 0.05 '.' 0.1 ' ' 1
## 
## Residual standard error: 95.7 on 26 degrees of freedom
## Multiple R-squared:  0.1496, Adjusted R-squared:  0.1169 
## F-statistic: 4.575 on 1 and 26 DF,  p-value: 0.04201
\end{verbatim}

Extrair a equação do modelo

\begin{Shaded}
\begin{Highlighting}[]
\NormalTok{eqn <-}\StringTok{ }\KeywordTok{as.character}\NormalTok{(}\KeywordTok{as.expression}\NormalTok{(}\KeywordTok{substitute}\NormalTok{(}\KeywordTok{italic}\NormalTok{(y) }\OperatorTok{==}\StringTok{ }\NormalTok{a }\OperatorTok{+}\StringTok{ }\NormalTok{b }\OperatorTok{*}\StringTok{ }\KeywordTok{italic}\NormalTok{(x) }\OperatorTok{*}\StringTok{ ","} \OperatorTok{~}\ErrorTok{~}\StringTok{ }\KeywordTok{italic}\NormalTok{(r)}\OperatorTok{^}\DecValTok{2} \OperatorTok{~}\StringTok{ "="} \OperatorTok{~}\StringTok{ }\NormalTok{r2,}\KeywordTok{list}\NormalTok{(}\DataTypeTok{a =} \KeywordTok{format}\NormalTok{(}\KeywordTok{coef}\NormalTok{(model)[}\DecValTok{1}\NormalTok{], }\DataTypeTok{digits=}\DecValTok{3}\NormalTok{),}\DataTypeTok{b =} \KeywordTok{format}\NormalTok{(}\KeywordTok{coef}\NormalTok{(model)[}\DecValTok{2}\NormalTok{], }\DataTypeTok{digits=}\DecValTok{3}\NormalTok{), }\DataTypeTok{r2 =} \KeywordTok{format}\NormalTok{(}\KeywordTok{summary}\NormalTok{(model)}\OperatorTok{$}\NormalTok{r.squared, }\DataTypeTok{digits=}\DecValTok{3}\NormalTok{)))))}
\end{Highlighting}
\end{Shaded}

Criando o gráfico

\begin{Shaded}
\begin{Highlighting}[]
\KeywordTok{ggplot}\NormalTok{(dados,}\KeywordTok{aes}\NormalTok{(}\DataTypeTok{x=}\NormalTok{gesso,}\DataTypeTok{y=}\NormalTok{peso,}\DataTypeTok{color=}\NormalTok{peso))  }\OperatorTok{+}\StringTok{ }
\StringTok{  }\KeywordTok{geom_point}\NormalTok{(}\DataTypeTok{size=}\FloatTok{2.9}\NormalTok{,}\DataTypeTok{shape=}\DecValTok{19}\NormalTok{, }\DataTypeTok{colour=}\StringTok{"grey10"}\NormalTok{) }\OperatorTok{+}\StringTok{ }
\StringTok{    }\KeywordTok{theme_bw}\NormalTok{(}\DataTypeTok{base_size =} \DecValTok{10}\NormalTok{) }\OperatorTok{+}\StringTok{ }
\StringTok{        }\KeywordTok{ylab}\NormalTok{(}\KeywordTok{expression}\NormalTok{(}\KeywordTok{paste}\NormalTok{(  }\StringTok{"Peso (g)"}\NormalTok{ )))  }\OperatorTok{+}\StringTok{ }
\StringTok{        }\KeywordTok{xlab}\NormalTok{(}\KeywordTok{expression}\NormalTok{(}\KeywordTok{paste}\NormalTok{(Gesso,}\StringTok{" kg ha"}\OperatorTok{^}\NormalTok{\{}\OperatorTok{-}\DecValTok{1}\NormalTok{\} )))  }\OperatorTok{+}\StringTok{ }
\StringTok{        }\KeywordTok{annotate}\NormalTok{(}\StringTok{"text"}\NormalTok{, }\DataTypeTok{label=}\NormalTok{eqn, }\DataTypeTok{parse=}\OtherTok{TRUE}\NormalTok{, }\DataTypeTok{x=}\OtherTok{Inf}\NormalTok{, }\DataTypeTok{y=}\OperatorTok{-}\OtherTok{Inf}\NormalTok{,}
             \DataTypeTok{hjust=}\FloatTok{1.}\NormalTok{, }\DataTypeTok{vjust=}\OperatorTok{-}\NormalTok{.}\DecValTok{5}\NormalTok{, }\DataTypeTok{size =} \DecValTok{5}\NormalTok{)  }\OperatorTok{+}\StringTok{    }
\StringTok{        }\KeywordTok{stat_smooth}\NormalTok{(}\DataTypeTok{method =}\NormalTok{ lm, }\DataTypeTok{se =}\NormalTok{ T, }\DataTypeTok{colour=}\StringTok{"red"}\NormalTok{, }\DataTypeTok{size=}\NormalTok{.}\DecValTok{85}\NormalTok{)}
\end{Highlighting}
\end{Shaded}

\begin{verbatim}
## `geom_smooth()` using formula 'y ~ x'
\end{verbatim}

\includegraphics{TudodoR_files/figure-latex/unnamed-chunk-226-1.pdf}

\hypertarget{delineamento-em-blocos-casualizados--dbc}{%
\subsection{Delineamento em blocos casualizados- DBC}\label{delineamento-em-blocos-casualizados--dbc}}

\textbf{Efeito do Promalin sobre Furtos de Macieira}

Resultados de um experimento instalado na Fazenda Chapadão, no município de Angatuba - SP. O delineamento experimental foi o de blocos casualizados, sendo as parcelas constituídas de 4 plantas espaçadas de 6 x 7 metros, com 12 anos de idade na época da instalação do experimento.

Baixar dados

\begin{Shaded}
\begin{Highlighting}[]
\NormalTok{dados <-}\StringTok{ }\KeywordTok{read.table}\NormalTok{(}\StringTok{"https://www.dropbox.com/s/9woiye3ce9twp78/BanzattoQd4.5.2.txt?dl=1"}\NormalTok{)}
\end{Highlighting}
\end{Shaded}

Verificar Estrutura dos dados

\begin{Shaded}
\begin{Highlighting}[]
\KeywordTok{str}\NormalTok{(dados)}
\end{Highlighting}
\end{Shaded}

\begin{verbatim}
## 'data.frame':    20 obs. of  3 variables:
##  $ promalin: Factor w/ 5 levels "12.5","12.5+12.5",..: 1 3 4 2 5 1 3 4 2 5 ...
##  $ bloco   : Factor w/ 4 levels "I","II","III",..: 1 1 1 1 1 2 2 2 2 2 ...
##  $ peso    : num  142 140 141 151 154 ...
\end{verbatim}

Transformação categorica

\begin{Shaded}
\begin{Highlighting}[]
\NormalTok{dados}\OperatorTok{$}\NormalTok{promalin =}\StringTok{ }\KeywordTok{as.factor}\NormalTok{(dados}\OperatorTok{$}\NormalTok{promalin)}
\NormalTok{dados}\OperatorTok{$}\NormalTok{bloco=}\StringTok{ }\KeywordTok{as.factor}\NormalTok{(dados}\OperatorTok{$}\NormalTok{bloco)}
\end{Highlighting}
\end{Shaded}

Estatistísca descritiva

\begin{Shaded}
\begin{Highlighting}[]
\KeywordTok{summary}\NormalTok{(dados)}
\end{Highlighting}
\end{Shaded}

\begin{verbatim}
##        promalin bloco        peso      
##  12.5      :4   I  :5   Min.   :130.6  
##  12.5+12.5 :4   II :5   1st Qu.:136.8  
##  25.0      :4   III:5   Median :141.6  
##  50.0      :4   IV :5   Mean   :143.0  
##  Testemunha:4           3rd Qu.:146.4  
##                         Max.   :165.0
\end{verbatim}

Ativar o pacote ggplot

\begin{Shaded}
\begin{Highlighting}[]
\KeywordTok{library}\NormalTok{(ggplot2)}
\end{Highlighting}
\end{Shaded}

Fazer o gráfico

\begin{Shaded}
\begin{Highlighting}[]
\KeywordTok{ggplot}\NormalTok{(dados,}\KeywordTok{aes}\NormalTok{(}\DataTypeTok{x=}\NormalTok{promalin ,}\DataTypeTok{y=}\NormalTok{peso, }\DataTypeTok{fill=}\NormalTok{promalin)) }\OperatorTok{+}\StringTok{ }
\StringTok{      }\KeywordTok{geom_boxplot}\NormalTok{(}\DataTypeTok{size=}\FloatTok{0.55}\NormalTok{,}\DataTypeTok{shape=}\DecValTok{19}\NormalTok{, }\DataTypeTok{colour=}\StringTok{"black"}\NormalTok{) }\OperatorTok{+}\StringTok{ }
\StringTok{      }\KeywordTok{theme}\NormalTok{(}\DataTypeTok{legend.position=}\StringTok{"top"}\NormalTok{) }
\end{Highlighting}
\end{Shaded}

\includegraphics{TudodoR_files/figure-latex/unnamed-chunk-232-1.pdf}

Analisando os blocos

\begin{Shaded}
\begin{Highlighting}[]
\KeywordTok{ggplot}\NormalTok{(dados,}\KeywordTok{aes}\NormalTok{(}\DataTypeTok{x=}\NormalTok{promalin ,}\DataTypeTok{y=}\NormalTok{peso, }\DataTypeTok{fill=}\NormalTok{promalin)) }\OperatorTok{+}\StringTok{ }
\StringTok{       }\KeywordTok{geom_point}\NormalTok{() }\OperatorTok{+}\StringTok{ }
\StringTok{       }\KeywordTok{theme}\NormalTok{(}\DataTypeTok{legend.position=}\StringTok{"top"}\NormalTok{) }\OperatorTok{+}\StringTok{ }
\StringTok{       }\KeywordTok{facet_wrap}\NormalTok{(}\OperatorTok{~}\NormalTok{bloco,}\DataTypeTok{ncol=}\DecValTok{4}\NormalTok{)}
\end{Highlighting}
\end{Shaded}

\includegraphics{TudodoR_files/figure-latex/unnamed-chunk-233-1.pdf}

Inserindo medias

\begin{Shaded}
\begin{Highlighting}[]
\KeywordTok{ggplot}\NormalTok{(dados,}\KeywordTok{aes}\NormalTok{(}\DataTypeTok{x=}\NormalTok{promalin ,}\DataTypeTok{y=}\NormalTok{peso, }\DataTypeTok{fill=}\NormalTok{promalin)) }\OperatorTok{+}\StringTok{ }
\StringTok{  }\KeywordTok{geom_boxplot}\NormalTok{(}\DataTypeTok{size=}\FloatTok{0.55}\NormalTok{,}\DataTypeTok{shape=}\DecValTok{19}\NormalTok{, }\DataTypeTok{colour=}\StringTok{"black"}\NormalTok{) }\OperatorTok{+}\StringTok{ }
\StringTok{  }\KeywordTok{theme}\NormalTok{(}\DataTypeTok{legend.position=}\StringTok{"top"}\NormalTok{) }\OperatorTok{+}\StringTok{ }
\StringTok{  }\KeywordTok{facet_wrap}\NormalTok{(}\OperatorTok{~}\NormalTok{bloco,}\DataTypeTok{ncol=}\DecValTok{4}\NormalTok{) }
\end{Highlighting}
\end{Shaded}

\includegraphics{TudodoR_files/figure-latex/unnamed-chunk-234-1.pdf}

Inserindo legenda nos eixos

\begin{Shaded}
\begin{Highlighting}[]
\KeywordTok{ggplot}\NormalTok{(dados,}\KeywordTok{aes}\NormalTok{(}\DataTypeTok{x=}\NormalTok{promalin,}\DataTypeTok{y=}\NormalTok{peso, }\DataTypeTok{fill=}\NormalTok{promalin)) }\OperatorTok{+}\StringTok{ }
\StringTok{       }\KeywordTok{geom_boxplot}\NormalTok{(}\DataTypeTok{size=}\FloatTok{0.55}\NormalTok{,}\DataTypeTok{shape=}\DecValTok{19}\NormalTok{, }\DataTypeTok{colour=}\StringTok{"black"}\NormalTok{) }\OperatorTok{+}\StringTok{ }
\StringTok{       }\KeywordTok{theme}\NormalTok{(}\DataTypeTok{legend.position=}\StringTok{"top"}\NormalTok{) }\OperatorTok{+}\StringTok{ }
\StringTok{       }\KeywordTok{xlab}\NormalTok{(}\StringTok{"Tratamentos"}\NormalTok{) }\OperatorTok{+}\StringTok{  }
\StringTok{       }\KeywordTok{ylab}\NormalTok{(}\StringTok{"Peso médio dos frutos (g)"}\NormalTok{) }
\end{Highlighting}
\end{Shaded}

\includegraphics{TudodoR_files/figure-latex/unnamed-chunk-235-1.pdf}

Inserindo legenda nos eixos

\begin{Shaded}
\begin{Highlighting}[]
\KeywordTok{ggplot}\NormalTok{(dados,}\KeywordTok{aes}\NormalTok{(}\DataTypeTok{x=}\NormalTok{promalin ,}\DataTypeTok{y=}\NormalTok{peso, }\DataTypeTok{fill=}\NormalTok{promalin)) }\OperatorTok{+}\StringTok{ }
\StringTok{      }\KeywordTok{geom_boxplot}\NormalTok{(}\DataTypeTok{size=}\FloatTok{0.55}\NormalTok{,}\DataTypeTok{shape=}\DecValTok{19}\NormalTok{, }\DataTypeTok{colour=}\StringTok{"black"}\NormalTok{) }\OperatorTok{+}\StringTok{ }
\StringTok{      }\KeywordTok{theme}\NormalTok{(}\DataTypeTok{legend.position=}\StringTok{"top"}\NormalTok{) }\OperatorTok{+}\StringTok{ }
\StringTok{      }\KeywordTok{stat_summary}\NormalTok{(}\DataTypeTok{fun.y=}\NormalTok{mean, }\DataTypeTok{geom=}\StringTok{"point"}\NormalTok{,}\DataTypeTok{shape=}\DecValTok{1}\NormalTok{,}\DataTypeTok{size=}\DecValTok{2}\NormalTok{) }\OperatorTok{+}\StringTok{ }
\StringTok{      }\KeywordTok{xlab}\NormalTok{(}\StringTok{"Tratamentos"}\NormalTok{) }\OperatorTok{+}\StringTok{  }
\StringTok{      }\KeywordTok{ylab}\NormalTok{(}\StringTok{"Peso médio dos frutos (g)"}\NormalTok{)  }\OperatorTok{+}
\StringTok{      }\KeywordTok{theme}\NormalTok{(}\DataTypeTok{panel.grid.minor =} \KeywordTok{element_line}\NormalTok{(}\DataTypeTok{colour =} \StringTok{"red"}\NormalTok{, }\DataTypeTok{linetype =} \StringTok{"dotted"}\NormalTok{)) }
\end{Highlighting}
\end{Shaded}

\begin{verbatim}
## Warning: `fun.y` is deprecated. Use `fun` instead.
\end{verbatim}

\includegraphics{TudodoR_files/figure-latex/unnamed-chunk-236-1.pdf}

Inserindo \texttt{tema\_bw} preto e branco

\begin{Shaded}
\begin{Highlighting}[]
\KeywordTok{ggplot}\NormalTok{(dados,}\KeywordTok{aes}\NormalTok{(}\DataTypeTok{x=}\NormalTok{promalin ,}\DataTypeTok{y=}\NormalTok{peso, }\DataTypeTok{fill=}\NormalTok{promalin)) }\OperatorTok{+}\StringTok{ }
\StringTok{      }\KeywordTok{geom_boxplot}\NormalTok{(}\DataTypeTok{size=}\FloatTok{0.55}\NormalTok{,}\DataTypeTok{shape=}\DecValTok{19}\NormalTok{, }\DataTypeTok{colour=}\StringTok{"black"}\NormalTok{) }\OperatorTok{+}\StringTok{ }
\StringTok{      }\KeywordTok{theme}\NormalTok{(}\DataTypeTok{legend.position=}\StringTok{"top"}\NormalTok{) }\OperatorTok{+}\StringTok{ }
\StringTok{      }\KeywordTok{stat_summary}\NormalTok{(}\DataTypeTok{fun.y=}\NormalTok{mean, }\DataTypeTok{geom=}\StringTok{"point"}\NormalTok{,}\DataTypeTok{shape=}\DecValTok{1}\NormalTok{,}\DataTypeTok{size=}\DecValTok{2}\NormalTok{) }\OperatorTok{+}\StringTok{ }
\StringTok{      }\KeywordTok{xlab}\NormalTok{(}\StringTok{"Tratamentos"}\NormalTok{) }\OperatorTok{+}\StringTok{  }
\StringTok{      }\KeywordTok{ylab}\NormalTok{(}\StringTok{"Peso médio dos frutos (g)"}\NormalTok{)  }\OperatorTok{+}
\StringTok{      }\KeywordTok{theme_bw}\NormalTok{() }
\end{Highlighting}
\end{Shaded}

\begin{verbatim}
## Warning: `fun.y` is deprecated. Use `fun` instead.
\end{verbatim}

\includegraphics{TudodoR_files/figure-latex/unnamed-chunk-237-1.pdf}

Inserindo legenda no topo

\begin{Shaded}
\begin{Highlighting}[]
\KeywordTok{ggplot}\NormalTok{(dados,}\KeywordTok{aes}\NormalTok{(}\DataTypeTok{x=}\NormalTok{promalin ,}\DataTypeTok{y=}\NormalTok{peso, }\DataTypeTok{fill=}\NormalTok{promalin)) }\OperatorTok{+}\StringTok{ }
\StringTok{      }\KeywordTok{geom_boxplot}\NormalTok{(}\DataTypeTok{size=}\FloatTok{0.55}\NormalTok{,}\DataTypeTok{shape=}\DecValTok{19}\NormalTok{, }\DataTypeTok{colour=}\StringTok{"black"}\NormalTok{) }\OperatorTok{+}\StringTok{ }
\StringTok{      }\KeywordTok{theme}\NormalTok{(}\DataTypeTok{legend.position=}\StringTok{"top"}\NormalTok{) }\OperatorTok{+}\StringTok{ }
\StringTok{      }\KeywordTok{stat_summary}\NormalTok{(}\DataTypeTok{fun.y=}\NormalTok{mean, }\DataTypeTok{geom=}\StringTok{"point"}\NormalTok{,}\DataTypeTok{shape=}\DecValTok{1}\NormalTok{,}\DataTypeTok{size=}\DecValTok{2}\NormalTok{) }\OperatorTok{+}\StringTok{ }
\StringTok{      }\KeywordTok{xlab}\NormalTok{(}\StringTok{"Tratamentos"}\NormalTok{) }\OperatorTok{+}\StringTok{  }
\StringTok{      }\KeywordTok{ylab}\NormalTok{(}\StringTok{"Peso médio dos frutos (g)"}\NormalTok{)  }\OperatorTok{+}
\StringTok{      }\KeywordTok{theme_bw}\NormalTok{() }\OperatorTok{+}
\StringTok{      }\KeywordTok{theme}\NormalTok{(}\DataTypeTok{legend.position=}\StringTok{"top"}\NormalTok{) }
\end{Highlighting}
\end{Shaded}

\begin{verbatim}
## Warning: `fun.y` is deprecated. Use `fun` instead.
\end{verbatim}

\includegraphics{TudodoR_files/figure-latex/unnamed-chunk-238-1.pdf}

Mudando escala do eixo y

\begin{Shaded}
\begin{Highlighting}[]
\KeywordTok{ggplot}\NormalTok{(dados,}\KeywordTok{aes}\NormalTok{(}\DataTypeTok{x=}\NormalTok{promalin ,}\DataTypeTok{y=}\NormalTok{peso, }\DataTypeTok{fill=}\NormalTok{promalin)) }\OperatorTok{+}\StringTok{ }
\StringTok{      }\KeywordTok{geom_boxplot}\NormalTok{(}\DataTypeTok{size=}\FloatTok{0.55}\NormalTok{,}\DataTypeTok{shape=}\DecValTok{19}\NormalTok{, }\DataTypeTok{colour=}\StringTok{"black"}\NormalTok{) }\OperatorTok{+}\StringTok{ }
\StringTok{      }\KeywordTok{theme}\NormalTok{(}\DataTypeTok{legend.position=}\StringTok{"top"}\NormalTok{) }\OperatorTok{+}\StringTok{ }
\StringTok{      }\KeywordTok{stat_summary}\NormalTok{(}\DataTypeTok{fun.y=}\NormalTok{mean, }\DataTypeTok{geom=}\StringTok{"point"}\NormalTok{,}\DataTypeTok{shape=}\DecValTok{1}\NormalTok{,}\DataTypeTok{size=}\DecValTok{2}\NormalTok{) }\OperatorTok{+}\StringTok{ }
\StringTok{      }\KeywordTok{xlab}\NormalTok{(}\StringTok{"Tratamentos"}\NormalTok{) }\OperatorTok{+}\StringTok{  }
\StringTok{      }\KeywordTok{ylab}\NormalTok{(}\StringTok{"Peso médio dos frutos (g)"}\NormalTok{)  }\OperatorTok{+}
\StringTok{      }\KeywordTok{theme_bw}\NormalTok{() }\OperatorTok{+}
\StringTok{      }\KeywordTok{theme}\NormalTok{(}\DataTypeTok{legend.position=}\StringTok{"top"}\NormalTok{) }\OperatorTok{+}
\StringTok{      }\KeywordTok{scale_y_continuous}\NormalTok{(}\DataTypeTok{breaks=}\KeywordTok{seq}\NormalTok{(}\DecValTok{0}\NormalTok{, }\DecValTok{180}\NormalTok{, }\DecValTok{5}\NormalTok{)) }\OperatorTok{+}
\StringTok{      }\KeywordTok{theme}\NormalTok{( }\DataTypeTok{axis.text.x  =} \KeywordTok{element_text}\NormalTok{(}\DataTypeTok{angle=}\DecValTok{90}\NormalTok{, }\DataTypeTok{vjust=}\DecValTok{0}\NormalTok{, }\DataTypeTok{size=}\DecValTok{10}\NormalTok{))}
\end{Highlighting}
\end{Shaded}

\begin{verbatim}
## Warning: `fun.y` is deprecated. Use `fun` instead.
\end{verbatim}

\includegraphics{TudodoR_files/figure-latex/unnamed-chunk-239-1.pdf}

\hypertarget{dados-climuxe1ticos}{%
\subsection{Dados Climáticos}\label{dados-climuxe1ticos}}

Dados climáticos de Rondonópolis - MT

Baixar dados no banco de dados o arquivo \href{https://www.dropbox.com/s/1ajoi1c8pla3yk6/roo.csv?dl=1}{roo.xlsx}

\begin{Shaded}
\begin{Highlighting}[]
\NormalTok{roo <-}\StringTok{ }\KeywordTok{read.csv2}\NormalTok{(}\StringTok{"https://www.dropbox.com/s/1ajoi1c8pla3yk6/roo.csv?dl=1"}\NormalTok{)}
\KeywordTok{View}\NormalTok{(roo)}

\KeywordTok{str}\NormalTok{(roo)}
\end{Highlighting}
\end{Shaded}

\begin{verbatim}
## 'data.frame':    4337 obs. of  11 variables:
##  $ dd    : int  1 1 2 3 4 5 6 7 8 9 ...
##  $ mm    : int  1 2 2 2 2 2 2 2 2 2 ...
##  $ ano   : int  1998 1998 1998 1998 1998 1998 1998 1998 1998 1998 ...
##  $ Prec  : num  NA 8.2 51 0.6 0 0 0 2.4 NA 0.8 ...
##  $ Tmax  : num  30 35.6 31.8 35.4 35.6 36.4 36.8 36.8 36.6 35.2 ...
##  $ Tmin  : num  21.7 21.8 21.8 21.5 22.1 22.5 23.5 23.5 24.3 22.9 ...
##  $ n     : num  NA NA NA NA NA NA NA NA NA NA ...
##  $ Tbs   : num  NA NA 25.1 25 26.6 ...
##  $ Tbu   : num  NA NA 23.9 23 23.9 ...
##  $ UR    : num  NA NA 89.8 86.5 80 ...
##  $ Vvento: num  NA NA 0.125 0.125 0.275 0.325 0.2 0.175 0.15 0.25 ...
\end{verbatim}

Boxplot para tempearatura minima

\begin{Shaded}
\begin{Highlighting}[]
\KeywordTok{ggplot}\NormalTok{(}\DataTypeTok{data =}\NormalTok{ roo, }\KeywordTok{aes}\NormalTok{(}\DataTypeTok{x =} \KeywordTok{factor}\NormalTok{(mm),}\DataTypeTok{y =}\NormalTok{ (Tmin)))}\OperatorTok{+}
\StringTok{  }\KeywordTok{geom_boxplot}\NormalTok{() }\OperatorTok{+}
\StringTok{  }\KeywordTok{scale_x_discrete}\NormalTok{(}\DataTypeTok{breaks=}\KeywordTok{c}\NormalTok{(}\StringTok{"1"}\NormalTok{, }\StringTok{"2"}\NormalTok{, }\StringTok{"3"}\NormalTok{, }\StringTok{"4"}\NormalTok{, }\StringTok{"5"}\NormalTok{, }\StringTok{"6"}\NormalTok{, }\StringTok{"7"}\NormalTok{, }\StringTok{"8"}\NormalTok{, }\StringTok{"9"}\NormalTok{, }\StringTok{"10"}\NormalTok{, }\StringTok{"11"}\NormalTok{,}\StringTok{"12"}\NormalTok{),}
            \DataTypeTok{labels=}\KeywordTok{c}\NormalTok{(}\StringTok{"Jan"}\NormalTok{,}\StringTok{"Fev"}\NormalTok{, }\StringTok{"Mar"}\NormalTok{, }\StringTok{"Abr"}\NormalTok{, }\StringTok{"Mai"}\NormalTok{, }\StringTok{"Jun"}\NormalTok{, }\StringTok{"Jul"}\NormalTok{, }\StringTok{"Ago"}\NormalTok{, }\StringTok{"Set"}\NormalTok{, }\StringTok{"Out"}\NormalTok{, }\StringTok{"Nov"}\NormalTok{, }\StringTok{"Dez"}\NormalTok{))}
\end{Highlighting}
\end{Shaded}

\begin{verbatim}
## Warning: Removed 222 rows containing non-finite values (stat_boxplot).
\end{verbatim}

\includegraphics{TudodoR_files/figure-latex/unnamed-chunk-241-1.pdf}

Grafico de distruição de temperatura minima total

\begin{Shaded}
\begin{Highlighting}[]
\KeywordTok{ggplot}\NormalTok{(}\DataTypeTok{data =}\NormalTok{ roo, }\KeywordTok{aes}\NormalTok{(}\DataTypeTok{x =}\NormalTok{ (Tmin)))}\OperatorTok{+}
\StringTok{  }\KeywordTok{geom_density}\NormalTok{()}
\end{Highlighting}
\end{Shaded}

\begin{verbatim}
## Warning: Removed 222 rows containing non-finite values (stat_density).
\end{verbatim}

\includegraphics{TudodoR_files/figure-latex/unnamed-chunk-242-1.pdf}

Grafico de distribuição de temperatura minima para cada mês

\begin{Shaded}
\begin{Highlighting}[]
\KeywordTok{ggplot}\NormalTok{(}\DataTypeTok{data =}\NormalTok{ roo, }\KeywordTok{aes}\NormalTok{(}\DataTypeTok{x =}\NormalTok{ (Tmin), }\DataTypeTok{fill=}\KeywordTok{factor}\NormalTok{(mm)))}\OperatorTok{+}
\StringTok{  }\KeywordTok{geom_density}\NormalTok{() }
\end{Highlighting}
\end{Shaded}

\begin{verbatim}
## Warning: Removed 222 rows containing non-finite values (stat_density).
\end{verbatim}

\includegraphics{TudodoR_files/figure-latex/unnamed-chunk-243-1.pdf}

\hypertarget{referuxeancia-3}{%
\section{Referência}\label{referuxeancia-3}}

GROLEMUND, G. WICKHAM, H. R for Data Science Site: \url{http://r4ds.had.co.nz/}

SITE: \url{https://www.statmethods.net/index.html}

CHANG, W. R Graphics Cookbook: Practical Recipes for Visualizing Data, Publisher: O'Reilly Media, 2002,416 p.~Site: \url{http://shop.oreilly.com/pesouct/0636920023135.do}

Este Capitulo foi baseado no livro \href{https://www.editoraufv.com.br/produto/conhecendo-o-r-uma-visao-mais-que-estatistica/1109294}{\textbf{Conhecendo o R: Um visão mais que estatística}}, e na página do \href{http://www.leg.ufpr.br/~paulojus/}{\textbf{Prof.~Paulo Justiniando Ribeiro}}

\hypertarget{testes-estatuxedsticos}{%
\chapter{Testes Estatísticos}\label{testes-estatuxedsticos}}

O R inclui em sua gama de utilidades, uma poderosa ferramenta da estatástica contemporânea: os testes estatísticos. Dentre esses, podemos destacar os testes de media, amplamente usados em várias áreas do conhecimento.

\hypertarget{teste-t-de-student}{%
\section{Teste t de Student}\label{teste-t-de-student}}

O teste t é bastante usado em várias situações do cotidiano quando se deseja fazer comparações entre \emph{uma ou mais médias}, sejam elas dependentes ou não.
Abaixo estão exemplos de vários modos de realizarmos o teste t.

Dados referentes a temperatura média do ar em duas condições: dentro de uma casa de vegetação e no campo.

\begin{Shaded}
\begin{Highlighting}[]
\NormalTok{pira_tem <-}\StringTok{ }\KeywordTok{read.csv2}\NormalTok{ (}\StringTok{"https://www.dropbox.com/s/zvp5iftcpb6bdpe/pira_tem.csv?dl=1"}\NormalTok{,}
  \DataTypeTok{dec=}\StringTok{"."}\NormalTok{)}
\KeywordTok{str}\NormalTok{(pira_tem)}
\end{Highlighting}
\end{Shaded}

\begin{verbatim}
## 'data.frame':    768 obs. of  5 variables:
##  $ hora   : Factor w/ 96 levels "0:00","0:15",..: 1 2 3 4 5 6 7 8 49 50 ...
##  $ periodo: Factor w/ 4 levels "equi_out","equi_prim",..: 4 4 4 4 4 4 4 4 4 4 ...
##  $ local  : Factor w/ 2 levels "campo","estufa": 2 2 2 2 2 2 2 2 2 2 ...
##  $ temp   : num  23.1 22.9 22.7 22.6 22.5 ...
##  $ X      : logi  NA NA NA NA NA NA ...
\end{verbatim}

Apresentação dos dados em forma de gráfico

\begin{Shaded}
\begin{Highlighting}[]
\KeywordTok{library}\NormalTok{(ggplot2)}
\KeywordTok{ggplot}\NormalTok{(}\DataTypeTok{data=}\NormalTok{ pira_tem, }\KeywordTok{aes}\NormalTok{ (}\DataTypeTok{x =}\NormalTok{ hora, }\DataTypeTok{y =}\NormalTok{ temp, }\DataTypeTok{colour =}\NormalTok{periodo)) }\OperatorTok{+}
\StringTok{  }\KeywordTok{geom_point}\NormalTok{(}\DataTypeTok{size=}\DecValTok{2}\NormalTok{,}\DataTypeTok{shape=}\DecValTok{19}\NormalTok{) }\OperatorTok{+}
\StringTok{  }\KeywordTok{geom_line}\NormalTok{() }\OperatorTok{+}
\StringTok{  }\KeywordTok{facet_grid}\NormalTok{(.}\OperatorTok{~}\NormalTok{local) }\OperatorTok{+}
\StringTok{  }\KeywordTok{xlab}\NormalTok{(}\StringTok{"Horas"}\NormalTok{) }\OperatorTok{+}
\StringTok{  }\KeywordTok{ylab}\NormalTok{(}\StringTok{"Temperatura ºC"}\NormalTok{) }\OperatorTok{+}\StringTok{ }
\StringTok{             }\KeywordTok{ggtitle}\NormalTok{(}\StringTok{"Variação da temperatura mediana}\CharTok{\textbackslash{}n}\StringTok{ nas quatro efemêrides") +}
\StringTok{             theme(plot.title=element_text(face="}\NormalTok{bold}\StringTok{", size=12, hjust = 0.5))  +}
\StringTok{  theme_bw()}
\end{Highlighting}
\end{Shaded}

\begin{verbatim}
## geom_path: Each group consists of only one observation. Do you need to adjust
## the group aesthetic?
## geom_path: Each group consists of only one observation. Do you need to adjust
## the group aesthetic?
\end{verbatim}

\includegraphics{TudodoR_files/figure-latex/unnamed-chunk-245-1.pdf}

\texttt{t.test()}
Realiza o teste t-Student para uma ou duas amostras.

sintaxe:
\texttt{t.test(amostra1,\ amostra2,\ opções)}

\textbf{Parâmetros}

\emph{amostra1:} Vetor contendo a amostra da qual se quer testar a média populacional, ou comparar a média populacional com a média populacional da amostra 2.

\emph{amostra2:} Vetor contendo a amostra 2 para comparação da média populacional com a média populacional da amostra 1.

\textbf{Opções}

\emph{alternative:} string indicando a hipótese alternativa desejada.
Valores possíveis: \emph{``two-sided'', ``less'' ou ``greater''}.

\emph{mu:} valor indicando o verdadeiro valor da média populacional para o caso de uma amostra, ou a diferença entre as mêdias para o caso de duas amostras.

\emph{paired:}
- TRUE - realiza o teste t pareado.
- FALSE - realiza o teste t não pareado.

\emph{var.equal}:
- TRUE - indica que a variância populacional é igual nas duas amostras.
- FALSE - indica que a variância populacional de cada amostra é diferente.

\emph{conf.level}: coeficiente de confiança do intervalo.

\hypertarget{para-uma-muxe9dia}{%
\subsection{Para uma média}\label{para-uma-muxe9dia}}

Vamos testar se a temperatura horaria do solsticio de verão no campo tem média igual ou maior que \textbf{21 ºC} na cidade de Piracicaba-SP.

\emph{H0: mu \textgreater= 21}

\emph{IC 95 para mu}

1.0 Passo filtrar os dados pelo fator ``periodo'' com o nivel sol\_verao (solsticio de verão).

\begin{Shaded}
\begin{Highlighting}[]
 \CommentTok{#Dividir os dados - subset()}
\NormalTok{    sol_verao_amb <-}\StringTok{ }\KeywordTok{subset}\NormalTok{(pira_tem, periodo }\OperatorTok{==}\StringTok{ "sol_verao"}\NormalTok{)}
\end{Highlighting}
\end{Shaded}

2.0 Passo filtrar os dados pelo fator ``local'' com o nivel campo.

\begin{Shaded}
\begin{Highlighting}[]
\NormalTok{ sol_verao_camp <-}\StringTok{ }\KeywordTok{subset}\NormalTok{(sol_verao_amb, local }\OperatorTok{==}\StringTok{ "campo"}\NormalTok{)}
\end{Highlighting}
\end{Shaded}

3.0 Verificar dados graficamente

\begin{Shaded}
\begin{Highlighting}[]
\KeywordTok{attach}\NormalTok{(pira_tem)}
\end{Highlighting}
\end{Shaded}

\begin{verbatim}
## The following objects are masked from pira_tem (pos = 13):
## 
##     hora, local, periodo, temp, X
\end{verbatim}

\begin{Shaded}
\begin{Highlighting}[]
\KeywordTok{boxplot}\NormalTok{(temp)}
\end{Highlighting}
\end{Shaded}

\includegraphics{TudodoR_files/figure-latex/unnamed-chunk-248-1.pdf}

4.0 Usar o teste T

\begin{Shaded}
\begin{Highlighting}[]
\KeywordTok{t.test}\NormalTok{(sol_verao_camp}\OperatorTok{$}\NormalTok{temp,                     }\CommentTok{#amostra a ser testada}
\DataTypeTok{mu=}\DecValTok{21}\NormalTok{,                                          }\CommentTok{#hipótese de nulidade}
\DataTypeTok{alternative=}\StringTok{"greater"}\NormalTok{,                         }\CommentTok{#teste unilateral pela direita}
\DataTypeTok{conf.level =} \FloatTok{0.95}\NormalTok{ )                         }\CommentTok{#Intervalo de confiancia de 95%  }
\end{Highlighting}
\end{Shaded}

\begin{verbatim}
## 
##  One Sample t-test
## 
## data:  sol_verao_camp$temp
## t = 21.648, df = 95, p-value < 2.2e-16
## alternative hypothesis: true mean is greater than 21
## 95 percent confidence interval:
##  24.96332      Inf
## sample estimates:
## mean of x 
##  25.29271
\end{verbatim}

Agora basta fazer a interpretação correta da saída do R.
Para saber qual hipótese foi aceita, basta verificar o valor do \emph{p-value} e estipular um nível de significância. Se neste exemplo o nível de significância fosse de 5\% a hipótese alternativa seria aceita uma vez que o \emph{p-value} foi menor ou igual a 0,05. Caso o \emph{p-value} tivesse sido maior que 5\% então aceitaríamos a hipótese de nulidade.
Como a hipótese alternativa foi a aceita isso implica que a temperatura do ar no solsticio de verão possui média estatisticamente diferente do valor 21ºC a um nível de significância de 5\%.

\textbf{Exercicio 1}

Vamos testar se X tem média estatiscamente igual a 35 ou maior
H0: mu =\textgreater35

\begin{Shaded}
\begin{Highlighting}[]
\NormalTok{x <-}\KeywordTok{c}\NormalTok{ (}\FloatTok{30.5}\NormalTok{,}\FloatTok{35.3}\NormalTok{,}\FloatTok{33.2}\NormalTok{,}\FloatTok{40.8}\NormalTok{,}\FloatTok{42.3}\NormalTok{,}\FloatTok{41.5}\NormalTok{,}\FloatTok{36.3}\NormalTok{,}\FloatTok{43.2}\NormalTok{,}\FloatTok{34.6}\NormalTok{,}\FloatTok{38.5}\NormalTok{)}

\KeywordTok{boxplot}\NormalTok{(x)}
\end{Highlighting}
\end{Shaded}

\includegraphics{TudodoR_files/figure-latex/unnamed-chunk-250-1.pdf}

Teste t.

\begin{Shaded}
\begin{Highlighting}[]
\KeywordTok{t.test}\NormalTok{(x,}
       \DataTypeTok{mu=}\DecValTok{35}\NormalTok{,}
       \DataTypeTok{alternative =} \StringTok{"greater"}\NormalTok{)}
\end{Highlighting}
\end{Shaded}

\begin{verbatim}
## 
##  One Sample t-test
## 
## data:  x
## t = 1.9323, df = 9, p-value = 0.04268
## alternative hypothesis: true mean is greater than 35
## 95 percent confidence interval:
##  35.13453      Inf
## sample estimates:
## mean of x 
##     37.62
\end{verbatim}

Com foi significativo admitimos que a amostra \emph{x} é oriunda de um população com média maior que o valor de 35, com nivel de 5\% de significância.

\textbf{Exercicio 2}

Um pesquisador afirmou que a temperatura média de solsticio de verão medido na casa de vegetação em Piracicaba-SP tem média \textbf{22,2 ºC}.
Desconfiando desse resultado um outro pesquisador com dados provinientes da mesma estação climatológicas em períodos diferentes encontrou os seguintes resultados:

\emph{H0: mu = 22,2}

\begin{Shaded}
\begin{Highlighting}[]
\NormalTok{  sol_verao_amb <-}\StringTok{ }\KeywordTok{subset}\NormalTok{(pira_tem, periodo }\OperatorTok{==}\StringTok{ "sol_verao"}\NormalTok{)}
\end{Highlighting}
\end{Shaded}

\begin{Shaded}
\begin{Highlighting}[]
\NormalTok{  sol_verao_est <-}\StringTok{ }\KeywordTok{subset}\NormalTok{(sol_verao_amb, local }\OperatorTok{==}\StringTok{ "estufa"}\NormalTok{)}
  \KeywordTok{boxplot}\NormalTok{(sol_verao_est}\OperatorTok{$}\NormalTok{temp)}
\end{Highlighting}
\end{Shaded}

\includegraphics{TudodoR_files/figure-latex/unnamed-chunk-253-1.pdf}

Essa afirmação é verdadeira?

\begin{Shaded}
\begin{Highlighting}[]
\KeywordTok{t.test}\NormalTok{(sol_verao_est}\OperatorTok{$}\NormalTok{temp,            }\CommentTok{#amostra a ser testada}
\DataTypeTok{mu=}\FloatTok{22.2}\NormalTok{,                              }\CommentTok{#hipótese de nulidade}
\DataTypeTok{alternative=}\StringTok{"two.sided"}\NormalTok{,              }\CommentTok{#teste bilateral não considera se é maior ou menor}
\DataTypeTok{conf.level =} \FloatTok{0.99}\NormalTok{)                    }\CommentTok{#significância de 1%        }
\end{Highlighting}
\end{Shaded}

\begin{verbatim}
## 
##  One Sample t-test
## 
## data:  sol_verao_est$temp
## t = 10.98, df = 95, p-value < 2.2e-16
## alternative hypothesis: true mean is not equal to 22.2
## 99 percent confidence interval:
##  25.06976 26.87629
## sample estimates:
## mean of x 
##  25.97302
\end{verbatim}

\hypertarget{para-duas-muxe9dias-independentes}{%
\subsection{Para duas médias independentes}\label{para-duas-muxe9dias-independentes}}

Para a realização do teste t pressupoe-se que as amostras possuem variâncias iguais
alem de seguirem distribuição normal.

Vamos a um exemplo:

Suponha dois conjuntos de dados de temperatura de media do ar de dois ambientes(casa de vegetação e campo). Verifique se as temperaturas dos dois ambientes são estatisticamente diferentes usando 5\% de significância.
\emph{H0: mu da temp da casa de vegetação = mu da temp do campo}

\begin{Shaded}
\begin{Highlighting}[]
\KeywordTok{boxplot}\NormalTok{(sol_verao_camp}\OperatorTok{$}\NormalTok{temp, sol_verao_est}\OperatorTok{$}\NormalTok{temp)}
\end{Highlighting}
\end{Shaded}

\includegraphics{TudodoR_files/figure-latex/unnamed-chunk-255-1.pdf}

\begin{Shaded}
\begin{Highlighting}[]
\KeywordTok{t.test}\NormalTok{(sol_verao_camp}\OperatorTok{$}\NormalTok{temp, sol_verao_est}\OperatorTok{$}\NormalTok{temp, }\CommentTok{#amostras a serem testadas}
      \DataTypeTok{alternative =} \StringTok{"greater"}\NormalTok{,                  }\CommentTok{#unilateral a direita }
      \DataTypeTok{var.equal =}\NormalTok{ T )                            }\CommentTok{#variância homogênea}
\end{Highlighting}
\end{Shaded}

\begin{verbatim}
## 
##  Two Sample t-test
## 
## data:  sol_verao_camp$temp and sol_verao_est$temp
## t = -1.7147, df = 190, p-value = 0.956
## alternative hypothesis: true difference in means is greater than 0
## 95 percent confidence interval:
##  -1.336097       Inf
## sample estimates:
## mean of x mean of y 
##  25.29271  25.97302
\end{verbatim}

Uma vez que o \emph{p-value} foi maior que 0,05, podemos concluir que as médias de temperatura dos dois ambientes não são diferentes, estatisticamente, a 5\% de significância.
Veja que o resultado desta analise mostra o valor de t (estatística do teste), os graus de liberdade (df) e o valor de p (significância). Alem disso, o resultado do teste ainda mostra as médias para cada grupo.

\hypertarget{para-duas-muxe9dias-dependentes}{%
\subsection{Para duas médias dependentes}\label{para-duas-muxe9dias-dependentes}}

Neste caso vamos usar o mesmo nível de significância do exemplo das amostras independentes.
As hipóteses se mantêm. Agora basta adicionar o argumento \texttt{paired=T}, informando que as amostras são dependentes.

\begin{Shaded}
\begin{Highlighting}[]
\KeywordTok{t.test}\NormalTok{(sol_verao_camp}\OperatorTok{$}\NormalTok{temp, sol_verao_est}\OperatorTok{$}\NormalTok{temp, }\CommentTok{#amostras a serem testadas}
      \DataTypeTok{conf.level=}\FloatTok{0.99}\NormalTok{,                          }\CommentTok{#nível de confiança}
      \DataTypeTok{paired=}\NormalTok{T,                                 }\CommentTok{#indica dependência entre as amostras}
      \DataTypeTok{var.equal =}\NormalTok{ T )                           }\CommentTok{#variância homogênea      }
\end{Highlighting}
\end{Shaded}

\begin{verbatim}
## 
##  Paired t-test
## 
## data:  sol_verao_camp$temp and sol_verao_est$temp
## t = -3.8433, df = 95, p-value = 0.0002193
## alternative hypothesis: true difference in means is not equal to 0
## 99 percent confidence interval:
##  -1.1455999 -0.2150251
## sample estimates:
## mean of the differences 
##              -0.6803125
\end{verbatim}

Note que a estatística do teste-t pareado não é baseada na média dos tratamentos, e sim na diferença entre os pares de tratamentos.

\hypertarget{teste-de-variuxe2ncia}{%
\section{Teste de variância}\label{teste-de-variuxe2ncia}}

\hypertarget{usando-o-teste-de-f}{%
\subsection{Usando o teste de F}\label{usando-o-teste-de-f}}

\emph{H0: a variancias das amostras são homogeneas }

\begin{Shaded}
\begin{Highlighting}[]
\KeywordTok{var.test}\NormalTok{ (sol_verao_camp}\OperatorTok{$}\NormalTok{temp, sol_verao_est}\OperatorTok{$}\NormalTok{temp)}
\end{Highlighting}
\end{Shaded}

\begin{verbatim}
## 
##  F test to compare two variances
## 
## data:  sol_verao_camp$temp and sol_verao_est$temp
## F = 0.33301, num df = 95, denom df = 95, p-value = 1.801e-07
## alternative hypothesis: true ratio of variances is not equal to 1
## 95 percent confidence interval:
##  0.2221988 0.4990825
## sample estimates:
## ratio of variances 
##          0.3330098
\end{verbatim}

As variâncias não são homogeneas.

Vamos resolver novamente o exercicio anterior, modificando o argumento \texttt{var.equal}

\begin{Shaded}
\begin{Highlighting}[]
 \KeywordTok{t.test}\NormalTok{(sol_verao_camp}\OperatorTok{$}\NormalTok{temp, sol_verao_est}\OperatorTok{$}\NormalTok{temp, }\CommentTok{#amostras a serem testadas}
      \DataTypeTok{conf.level=}\FloatTok{0.99}\NormalTok{,                          }\CommentTok{#nível de confiança}
      \DataTypeTok{paired=}\NormalTok{T,                                 }\CommentTok{#indica dependência entre as amostras}
      \DataTypeTok{var.equal =}\NormalTok{ F )                           }\CommentTok{#variância homogênea }
\end{Highlighting}
\end{Shaded}

\begin{verbatim}
## 
##  Paired t-test
## 
## data:  sol_verao_camp$temp and sol_verao_est$temp
## t = -3.8433, df = 95, p-value = 0.0002193
## alternative hypothesis: true difference in means is not equal to 0
## 99 percent confidence interval:
##  -1.1455999 -0.2150251
## sample estimates:
## mean of the differences 
##              -0.6803125
\end{verbatim}

\hypertarget{teste-para-a-normalidade---shapiro.test}{%
\section{\texorpdfstring{Teste para a normalidade - \texttt{shapiro.test()}}{Teste para a normalidade - shapiro.test()}}\label{teste-para-a-normalidade---shapiro.test}}

Por vezes temos necessidade de identificar com certa confiança se uma amostra ou conjunto de dados segue a distribuição normal. Isso e possível, no R, com o uso do comando \texttt{shapiro.test()}

Verifique normalidade dos dados

\begin{Shaded}
\begin{Highlighting}[]
\KeywordTok{shapiro.test}\NormalTok{(sol_verao_camp}\OperatorTok{$}\NormalTok{temp)}
\end{Highlighting}
\end{Shaded}

\begin{verbatim}
## 
##  Shapiro-Wilk normality test
## 
## data:  sol_verao_camp$temp
## W = 0.90789, p-value = 5.043e-06
\end{verbatim}

\begin{Shaded}
\begin{Highlighting}[]
\KeywordTok{shapiro.test}\NormalTok{(sol_verao_est}\OperatorTok{$}\NormalTok{temp)}
\end{Highlighting}
\end{Shaded}

\begin{verbatim}
## 
##  Shapiro-Wilk normality test
## 
## data:  sol_verao_est$temp
## W = 0.85041, p-value = 2.027e-08
\end{verbatim}

O comando \texttt{qqnorm()}nos fornece diretamente um gráfico da distribuição de percentagens
acumuladas chamado de gráfico de probabilidade normal. Se os pontos deste gráfico seguem um padrão aproximado de uma reta, este fato evidencia que a variável aleatória em questão tem a distribuição aproximadamente normal.

\begin{Shaded}
\begin{Highlighting}[]
\KeywordTok{qqnorm}\NormalTok{(sol_verao_camp}\OperatorTok{$}\NormalTok{temp) }\CommentTok{#obtendo o normal probability plot só para comparação}
\end{Highlighting}
\end{Shaded}

\includegraphics{TudodoR_files/figure-latex/unnamed-chunk-260-1.pdf}

\begin{Shaded}
\begin{Highlighting}[]
\KeywordTok{qqnorm}\NormalTok{(sol_verao_est}\OperatorTok{$}\NormalTok{temp)}
\end{Highlighting}
\end{Shaded}

\includegraphics{TudodoR_files/figure-latex/unnamed-chunk-260-2.pdf}

\hypertarget{teste-u-de-mann-whitney}{%
\section{Teste U de Mann-Whitney}\label{teste-u-de-mann-whitney}}

\emph{H0: mu da temp da casa de vegetação = mu da temp do campo}

\begin{Shaded}
\begin{Highlighting}[]
\KeywordTok{wilcox.test}\NormalTok{(sol_verao_camp}\OperatorTok{$}\NormalTok{temp,sol_verao_est}\OperatorTok{$}\NormalTok{temp,}
  \DataTypeTok{alternative =} \StringTok{"two.side"}\NormalTok{)}
\end{Highlighting}
\end{Shaded}

\begin{verbatim}
## 
##  Wilcoxon rank sum test with continuity correction
## 
## data:  sol_verao_camp$temp and sol_verao_est$temp
## W = 4579, p-value = 0.941
## alternative hypothesis: true location shift is not equal to 0
\end{verbatim}

\hypertarget{covariuxe2ncia-e-correlauxe7uxe3o}{%
\section{Covariância e Correlação}\label{covariuxe2ncia-e-correlauxe7uxe3o}}

A covariância e a correlação entre dois conjuntos de dados quaisquer podem ser obtidos pelos comandos \texttt{cov(x,y)} e \texttt{cor(x,y)}, respectivamente.
São medidads utilizadas no estudo do comportamento conjunto de duas variáveis quantitativas distintas. Elas informam a variação conjunta (covarincia) ou grau de associaçãp (correlação) entre duas variaveis aleatorias X e Y.

A correlação de \textbf{Pearson} é uma medida paramétrica de associação linear entre duas variaveis.

A correlação de ordem de \textbf{Sperman} é uma medidad não paramétrica de associação entre duas variáveis

A correlação de ordem de \textbf{Kendall} é outra medida não paramétrica da associação, baseada na concordância ou discordância dos pares x-y

\begin{Shaded}
\begin{Highlighting}[]
\KeywordTok{help}\NormalTok{ (}\StringTok{"cor.test"}\NormalTok{)}
\end{Highlighting}
\end{Shaded}

Plote os valores

\begin{Shaded}
\begin{Highlighting}[]
\KeywordTok{plot}\NormalTok{(sol_verao_camp}\OperatorTok{$}\NormalTok{temp,sol_verao_est}\OperatorTok{$}\NormalTok{temp, }\DataTypeTok{las=}\DecValTok{2}\NormalTok{)}
\end{Highlighting}
\end{Shaded}

\includegraphics{TudodoR_files/figure-latex/unnamed-chunk-263-1.pdf}

Teste de correlação de Pearson

\begin{Shaded}
\begin{Highlighting}[]
\KeywordTok{cor}\NormalTok{(sol_verao_camp}\OperatorTok{$}\NormalTok{temp,sol_verao_est}\OperatorTok{$}\NormalTok{temp, }
    \DataTypeTok{method =} \StringTok{"pearson"}
\NormalTok{    )}
\end{Highlighting}
\end{Shaded}

\begin{verbatim}
## [1] 0.9250728
\end{verbatim}

Teste de correlação de Pearson (the default)

\begin{Shaded}
\begin{Highlighting}[]
\KeywordTok{cor}\NormalTok{(sol_verao_camp}\OperatorTok{$}\NormalTok{temp,sol_verao_est}\OperatorTok{$}\NormalTok{temp)}
\end{Highlighting}
\end{Shaded}

\begin{verbatim}
## [1] 0.9250728
\end{verbatim}

Teste de correlação de Pearson trocando o X e Y

\begin{Shaded}
\begin{Highlighting}[]
\KeywordTok{cor}\NormalTok{(sol_verao_est}\OperatorTok{$}\NormalTok{temp, sol_verao_camp}\OperatorTok{$}\NormalTok{temp)}
\end{Highlighting}
\end{Shaded}

\begin{verbatim}
## [1] 0.9250728
\end{verbatim}

Teste de correlação de Spearman

\begin{Shaded}
\begin{Highlighting}[]
\KeywordTok{cor}\NormalTok{(sol_verao_camp}\OperatorTok{$}\NormalTok{temp,sol_verao_est}\OperatorTok{$}\NormalTok{temp, }
    \DataTypeTok{method =} \StringTok{"spearman"}\NormalTok{)}
\end{Highlighting}
\end{Shaded}

\begin{verbatim}
## [1] 0.9412321
\end{verbatim}

Teste de correlação de Kendall

\begin{Shaded}
\begin{Highlighting}[]
\KeywordTok{cor}\NormalTok{(sol_verao_camp}\OperatorTok{$}\NormalTok{temp,sol_verao_est}\OperatorTok{$}\NormalTok{temp, }
    \DataTypeTok{method =} \StringTok{"kendall"}\NormalTok{)}
\end{Highlighting}
\end{Shaded}

\begin{verbatim}
## [1] 0.8113615
\end{verbatim}

Teste de correlação de Pearson

\begin{Shaded}
\begin{Highlighting}[]
\KeywordTok{cor.test}\NormalTok{ (sol_verao_camp}\OperatorTok{$}\NormalTok{temp,sol_verao_est}\OperatorTok{$}\NormalTok{temp, }
    \DataTypeTok{method =} \StringTok{"pearson"}
\NormalTok{    )}
\end{Highlighting}
\end{Shaded}

\begin{verbatim}
## 
##  Pearson's product-moment correlation
## 
## data:  sol_verao_camp$temp and sol_verao_est$temp
## t = 23.615, df = 94, p-value < 2.2e-16
## alternative hypothesis: true correlation is not equal to 0
## 95 percent confidence interval:
##  0.8895702 0.9494668
## sample estimates:
##       cor 
## 0.9250728
\end{verbatim}

\begin{Shaded}
\begin{Highlighting}[]
\KeywordTok{cor.test}\NormalTok{ (sol_verao_camp}\OperatorTok{$}\NormalTok{temp,sol_verao_est}\OperatorTok{$}\NormalTok{temp, }
    \DataTypeTok{method =} \StringTok{"spearman"}
\NormalTok{    )}
\end{Highlighting}
\end{Shaded}

\begin{verbatim}
## Warning in cor.test.default(sol_verao_camp$temp, sol_verao_est$temp, method =
## "spearman"): Cannot compute exact p-value with ties
\end{verbatim}

\begin{verbatim}
## 
##  Spearman's rank correlation rho
## 
## data:  sol_verao_camp$temp and sol_verao_est$temp
## S = 8664.7, p-value < 2.2e-16
## alternative hypothesis: true rho is not equal to 0
## sample estimates:
##       rho 
## 0.9412321
\end{verbatim}

\begin{Shaded}
\begin{Highlighting}[]
\KeywordTok{cor.test}\NormalTok{ (sol_verao_camp}\OperatorTok{$}\NormalTok{temp,sol_verao_est}\OperatorTok{$}\NormalTok{temp, }
    \DataTypeTok{method =} \StringTok{"spearman"}\NormalTok{, }\DataTypeTok{exact =}\NormalTok{ F}
\NormalTok{    )}
\end{Highlighting}
\end{Shaded}

\begin{verbatim}
## 
##  Spearman's rank correlation rho
## 
## data:  sol_verao_camp$temp and sol_verao_est$temp
## S = 8664.7, p-value < 2.2e-16
## alternative hypothesis: true rho is not equal to 0
## sample estimates:
##       rho 
## 0.9412321
\end{verbatim}

\begin{Shaded}
\begin{Highlighting}[]
\KeywordTok{cov}\NormalTok{ (sol_verao_camp}\OperatorTok{$}\NormalTok{temp,sol_verao_est}\OperatorTok{$}\NormalTok{temp)}
\end{Highlighting}
\end{Shaded}

\begin{verbatim}
## [1] 6.051517
\end{verbatim}

\hypertarget{outros-testes}{%
\section{Outros testes}\label{outros-testes}}

Utilizaremos o banco de dados \href{https://www.dropbox.com/s/zg7fyg1iewtji49/dadosfisio.csv?dl=0}{dadosfisio}

\begin{Shaded}
\begin{Highlighting}[]
\NormalTok{fisio <-}\StringTok{ }\KeywordTok{read.csv2}\NormalTok{(}\StringTok{"https://www.dropbox.com/s/zg7fyg1iewtji49/dadosfisio.csv?dl=1"}\NormalTok{)}
\KeywordTok{attach}\NormalTok{(fisio)}
\end{Highlighting}
\end{Shaded}

\begin{verbatim}
## The following objects are masked _by_ .GlobalEnv:
## 
##     a, b, x, y, z
\end{verbatim}

\begin{verbatim}
## The following objects are masked from fisio (pos = 13):
## 
##     a, b, cc, cota, ds, ma, ptotal, tibe, tibo, x, X120, X3, X60, X90,
##     y, z
\end{verbatim}

\begin{Shaded}
\begin{Highlighting}[]
\KeywordTok{pairs}\NormalTok{(fisio[,}\DecValTok{4}\OperatorTok{:}\DecValTok{10}\NormalTok{])}
\end{Highlighting}
\end{Shaded}

\includegraphics{TudodoR_files/figure-latex/unnamed-chunk-274-1.pdf}

Teste de Spearman

\begin{Shaded}
\begin{Highlighting}[]
\KeywordTok{cor}\NormalTok{(fisio[,}\DecValTok{3}\OperatorTok{:}\DecValTok{8}\NormalTok{],}\DataTypeTok{method =} \StringTok{"spearman"}\NormalTok{)}
\end{Highlighting}
\end{Shaded}

\begin{verbatim}
##                   y        cota           ds           cc          ma
## y       1.000000000  0.03913488 -0.007304133 -0.005043318 -0.02834692
## cota    0.039134875  1.00000000 -0.625860523  0.506913897  0.51222110
## ds     -0.007304133 -0.62586052  1.000000000 -0.719094380 -0.80918958
## cc     -0.005043318  0.50691390 -0.719094380  1.000000000  0.40193585
## ma     -0.028346924  0.51222110 -0.809189577  0.401935847  1.00000000
## ptotal -0.001391263  0.54405131 -0.962597591  0.812104786  0.75424837
##              ptotal
## y      -0.001391263
## cota    0.544051305
## ds     -0.962597591
## cc      0.812104786
## ma      0.754248369
## ptotal  1.000000000
\end{verbatim}

\hypertarget{hydrogof}{%
\subsection{hydroGOF}\label{hydrogof}}

Carregando a biblioteca hydroGOF, que contém dados e funções usadas nesta análise.

\begin{Shaded}
\begin{Highlighting}[]
\KeywordTok{library}\NormalTok{(hydroGOF)}
\end{Highlighting}
\end{Shaded}

Cálculo das medidas numéricas de qualidade do ajuste para o ``melhor'' caso (inatingível)

\begin{Shaded}
\begin{Highlighting}[]
\KeywordTok{gof}\NormalTok{(}\DataTypeTok{sim =}\NormalTok{ fisio}\OperatorTok{$}\NormalTok{ds, }\DataTypeTok{obs=}\NormalTok{ fisio}\OperatorTok{$}\NormalTok{cc)}
\end{Highlighting}
\end{Shaded}

\begin{verbatim}
##            [,1]
## ME         1.36
## MAE        1.36
## MSE        1.90
## RMSE       1.38
## NRMSE % 1866.00
## PBIAS %  453.80
## RSR       18.66
## rSD        2.24
## NSE     -350.47
## mNSE     -20.82
## rNSE    -460.60
## d          0.07
## md         0.04
## rd        -0.22
## cp      -911.57
## r         -0.87
## R2         0.75
## bR2        0.15
## KGE       -4.06
## VE        -3.54
\end{verbatim}

\hypertarget{delineamento-em-bloco-casualizado}{%
\chapter{Delineamento em bloco casualizado}\label{delineamento-em-bloco-casualizado}}

O delineamento em blocos casualizados (DBC) tem três princípios basicos de experimentação:

\begin{itemize}
\item
  repetição
\item
  casualização
\item
  controle local
\end{itemize}

É o deliamento mais utilizado de todos delineamento. Ele é utilizado quando há heterogeneidade nas condições experimentais. Nesse caso divide-se o material experimental, ou amostra, em bloco homogêneos de forma a contemplar as diferenças entre grupos. A ANOVA associada a este modelo de experimento é também conhecida como \emph{Two Way ANOVA}.

\hypertarget{anuxe1lise-de-experimento-dbc}{%
\section{Análise de experimento DBC}\label{anuxe1lise-de-experimento-dbc}}

Resultados de um experimento instalado na Fazenda Chapadão, no município de Angatuba - SP. O delineamento experimental foi o de blocos casualizados, sendo as parcelas constituídas de 4 plantas espa?adas de 6 x 7 metros, com 12 anos de idade na época da instalação do experimento.

Importando dados

\begin{Shaded}
\begin{Highlighting}[]
\NormalTok{dados <-}\StringTok{ }\KeywordTok{read.table}\NormalTok{(}\StringTok{"https://www.dropbox.com/s/9woiye3ce9twp78/BanzattoQd4.5.2.txt?dl=1"}\NormalTok{) }
\end{Highlighting}
\end{Shaded}

conferir se temos fatores para fazer a análise de variância

\begin{Shaded}
\begin{Highlighting}[]
\KeywordTok{str}\NormalTok{(dados)}
\end{Highlighting}
\end{Shaded}

\begin{verbatim}
## 'data.frame':    20 obs. of  3 variables:
##  $ promalin: Factor w/ 5 levels "12.5","12.5+12.5",..: 1 3 4 2 5 1 3 4 2 5 ...
##  $ bloco   : Factor w/ 4 levels "I","II","III",..: 1 1 1 1 1 2 2 2 2 2 ...
##  $ peso    : num  142 140 141 151 154 ...
\end{verbatim}

Lembramos que o \emph{peso} deve ter conteudo numerico e o \emph{promalin} e \emph{bloco} deve ser fator.

\begin{Shaded}
\begin{Highlighting}[]
\NormalTok{dados}\OperatorTok{$}\NormalTok{promalin<-}\KeywordTok{as.factor}\NormalTok{(dados}\OperatorTok{$}\NormalTok{promalin)}
\NormalTok{dados}\OperatorTok{$}\NormalTok{bloco<-}\KeywordTok{as.numeric}\NormalTok{(dados}\OperatorTok{$}\NormalTok{bloco)}
\end{Highlighting}
\end{Shaded}

Verificação gráfica**

\begin{Shaded}
\begin{Highlighting}[]
\KeywordTok{require}\NormalTok{(lattice)}
\KeywordTok{xyplot}\NormalTok{(peso }\OperatorTok{~}\StringTok{ }\NormalTok{promalin, }
        \DataTypeTok{groups =}\NormalTok{ bloco, }
        \DataTypeTok{data=}\NormalTok{ dados)}
\end{Highlighting}
\end{Shaded}

\includegraphics{TudodoR_files/figure-latex/unnamed-chunk-281-1.pdf}

O efeito do bloco é aditivo?

Ligar as observações com o mesmo bloco com a função \texttt{type\ ="o"}

\begin{Shaded}
\begin{Highlighting}[]
\KeywordTok{xyplot}\NormalTok{(peso }\OperatorTok{~}\StringTok{ }\KeywordTok{reorder}\NormalTok{(promalin, peso), }
        \DataTypeTok{groups =}\NormalTok{ bloco, }
        \DataTypeTok{data=}\NormalTok{ dados,}
        \DataTypeTok{type =} \StringTok{"o"}\NormalTok{)}
\end{Highlighting}
\end{Shaded}

\includegraphics{TudodoR_files/figure-latex/unnamed-chunk-282-1.pdf}

Reordenar os tratamentos

\begin{Shaded}
\begin{Highlighting}[]
\KeywordTok{require}\NormalTok{(plyr)}
\NormalTok{dados}\OperatorTok{$}\NormalTok{promalin <-}\StringTok{ }\KeywordTok{with}\NormalTok{(dados, }\KeywordTok{reorder}\NormalTok{(promalin, peso))}
\NormalTok{dados <-}\StringTok{ }\KeywordTok{arrange}\NormalTok{(dados, promalin, bloco)}
\end{Highlighting}
\end{Shaded}

Graficos reordenados da menor média a maior média por tratamento

\begin{Shaded}
\begin{Highlighting}[]
\KeywordTok{xyplot}\NormalTok{(peso }\OperatorTok{~}\StringTok{ }\KeywordTok{reorder}\NormalTok{(promalin, peso), }
        \DataTypeTok{groups =}\NormalTok{ bloco, }
        \DataTypeTok{data=}\NormalTok{ dados,}
        \DataTypeTok{type =} \StringTok{"o"}\NormalTok{)}
\end{Highlighting}
\end{Shaded}

\includegraphics{TudodoR_files/figure-latex/unnamed-chunk-284-1.pdf}

\hypertarget{anuxe1lise-de-variuxe2ncia}{%
\subsection{Análise de variância}\label{anuxe1lise-de-variuxe2ncia}}

Fazendo a análise de variância

\begin{Shaded}
\begin{Highlighting}[]
\NormalTok{m0 <-}\StringTok{ }\KeywordTok{lm}\NormalTok{ (dados}\OperatorTok{$}\NormalTok{peso }\OperatorTok{~}\StringTok{ }\NormalTok{dados}\OperatorTok{$}\NormalTok{bloco }\OperatorTok{+}\StringTok{ }\NormalTok{dados}\OperatorTok{$}\NormalTok{promalin, }\DataTypeTok{data =}\NormalTok{ dados)}
\end{Highlighting}
\end{Shaded}

\begin{Shaded}
\begin{Highlighting}[]
\KeywordTok{anova}\NormalTok{(m0)}
\end{Highlighting}
\end{Shaded}

\begin{verbatim}
## Analysis of Variance Table
## 
## Response: dados$peso
##                Df Sum Sq Mean Sq F value   Pr(>F)   
## dados$bloco     1  71.57  71.572  2.4574 0.139291   
## dados$promalin  4 788.95 197.238  6.7721 0.002994 **
## Residuals      14 407.75  29.125                    
## ---
## Signif. codes:  0 '***' 0.001 '**' 0.01 '*' 0.05 '.' 0.1 ' ' 1
\end{verbatim}

Extraindo o coeficiente de variação

\begin{Shaded}
\begin{Highlighting}[]
\KeywordTok{require}\NormalTok{(agricolae)}
\KeywordTok{cv.model}\NormalTok{(m0)}
\end{Highlighting}
\end{Shaded}

\begin{verbatim}
## [1] 3.774477
\end{verbatim}

Análise gráfica dos resíduos

\begin{Shaded}
\begin{Highlighting}[]
\KeywordTok{par}\NormalTok{(}\DataTypeTok{mfrow=} \KeywordTok{c}\NormalTok{(}\DecValTok{2}\NormalTok{,}\DecValTok{2}\NormalTok{))}
\KeywordTok{plot}\NormalTok{(m0)}
\end{Highlighting}
\end{Shaded}

\includegraphics{TudodoR_files/figure-latex/unnamed-chunk-288-1.pdf}

Analisando a Figura acima sugere que o principal problema deste conjunto de dados pode ser a não normalidade.

\hypertarget{teste-das-pressuposiuxe7uxf5es-da-anuxe1lise-de-variuxe2ncia}{%
\subsubsection{Teste das pressuposições da análise de variância}\label{teste-das-pressuposiuxe7uxf5es-da-anuxe1lise-de-variuxe2ncia}}

\hypertarget{teste-de-bartllet-para-homocedasticidade}{%
\paragraph{Teste de Bartllet para homocedasticidade}\label{teste-de-bartllet-para-homocedasticidade}}

\begin{Shaded}
\begin{Highlighting}[]
\KeywordTok{bartlett.test}\NormalTok{(m0}\OperatorTok{$}\NormalTok{res, dados}\OperatorTok{$}\NormalTok{promalin)}
\end{Highlighting}
\end{Shaded}

\begin{verbatim}
## 
##  Bartlett test of homogeneity of variances
## 
## data:  m0$res and dados$promalin
## Bartlett's K-squared = 1.7485, df = 4, p-value = 0.7819
\end{verbatim}

Como observamos uma não significancia estatística neste resultado \emph{(p-value = 0.7819)}, devemos aceitar a hipótese nula de que as variâncias sejam as mesma em todos os níveis do fator.

\hypertarget{teste-de-shapiro-wilk-para-normalidade}{%
\paragraph{Teste de Shapiro-Wilk para Normalidade}\label{teste-de-shapiro-wilk-para-normalidade}}

\begin{Shaded}
\begin{Highlighting}[]
\KeywordTok{shapiro.test}\NormalTok{(m0}\OperatorTok{$}\NormalTok{res)}
\end{Highlighting}
\end{Shaded}

\begin{verbatim}
## 
##  Shapiro-Wilk normality test
## 
## data:  m0$res
## W = 0.855, p-value = 0.006472
\end{verbatim}

Como observamos uma significancia estatística neste resultado \emph{(p-value = 0.006472)}, devemos rejeitar a hipótese nula de que os residuoes tedem a distruibuição normal.

\hypertarget{transformauxe7uxe3o-de-dados}{%
\subsection{Transformação de dados}\label{transformauxe7uxe3o-de-dados}}

Tranformação de dados é uma das possíveis formas de contarnar o problema de dados que não obedecem os pressupostos da análise de variância. Vamos ver como isto poder ser feito com o programa R.

\hypertarget{transformauxe7uxe3o-de-dados-com-o-box-cox}{%
\subsubsection{Transformação de dados com o BOX-COX}\label{transformauxe7uxe3o-de-dados-com-o-box-cox}}

Para tentar contornar o problema vamos usar a transformação Box-Cox, que consiste em transformar os dados de acordo com uma expressão.

A função \texttt{boxcox()} do pacote MASS calcula a verossimilhança perfilhada do parâmetro lambda. Devemos escolher o valor que maximiza esta função. Nos comandos a seguir começamos carregando o pacote MASS e depois obtemos o gráfico da verossimilhança perfilhada. Como estamos interessados no máximo fazermos um novo gráfico com um zoom na região de interesse.

\begin{Shaded}
\begin{Highlighting}[]
\KeywordTok{require}\NormalTok{(MASS) }
  \KeywordTok{boxcox}\NormalTok{(m0)}
\end{Highlighting}
\end{Shaded}

\includegraphics{TudodoR_files/figure-latex/unnamed-chunk-291-1.pdf}

\begin{Shaded}
\begin{Highlighting}[]
  \KeywordTok{boxcox}\NormalTok{(m0, }\DataTypeTok{lam =} \KeywordTok{seq}\NormalTok{(}\OperatorTok{-}\DecValTok{8}\NormalTok{, }\DecValTok{8}\NormalTok{, }\DecValTok{1}\OperatorTok{/}\DecValTok{10}\NormalTok{))}
\end{Highlighting}
\end{Shaded}

\includegraphics{TudodoR_files/figure-latex/unnamed-chunk-291-2.pdf}

Localizando o ponto máximo.

\hypertarget{anuxe1lise-de-variuxe2ncia---ajuste-com-a-variuxe1vel-transformada.}{%
\subsubsection{Análise de variância - Ajuste com a variável transformada.}\label{anuxe1lise-de-variuxe2ncia---ajuste-com-a-variuxe1vel-transformada.}}

\begin{Shaded}
\begin{Highlighting}[]
\NormalTok{m1 <-}\StringTok{ }\KeywordTok{aov}\NormalTok{ (}\KeywordTok{log}\NormalTok{(dados}\OperatorTok{$}\NormalTok{peso) }\OperatorTok{~}\StringTok{ }\NormalTok{dados}\OperatorTok{$}\NormalTok{promalin, }\DataTypeTok{data =}\NormalTok{ dados)}
\end{Highlighting}
\end{Shaded}

Anáise gráfica dos resíduos

\begin{Shaded}
\begin{Highlighting}[]
\KeywordTok{par}\NormalTok{(}\DataTypeTok{mfrow =} \KeywordTok{c}\NormalTok{(}\DecValTok{2}\NormalTok{,}\DecValTok{2}\NormalTok{))}
\KeywordTok{plot}\NormalTok{(m1)}
\end{Highlighting}
\end{Shaded}

\includegraphics{TudodoR_files/figure-latex/unnamed-chunk-294-1.pdf}

Os pressupostos foram atendindos ?

Teste de Shapiro-Wilk para Normalidade

\begin{Shaded}
\begin{Highlighting}[]
\KeywordTok{shapiro.test}\NormalTok{(m1}\OperatorTok{$}\NormalTok{res)}
\end{Highlighting}
\end{Shaded}

\begin{verbatim}
## 
##  Shapiro-Wilk normality test
## 
## data:  m1$res
## W = 0.93909, p-value = 0.2305
\end{verbatim}

Teste de Bartllet para homocedasticidade

\begin{Shaded}
\begin{Highlighting}[]
\KeywordTok{bartlett.test}\NormalTok{(m1}\OperatorTok{$}\NormalTok{res, dados}\OperatorTok{$}\NormalTok{promalin)}
\end{Highlighting}
\end{Shaded}

\begin{verbatim}
## 
##  Bartlett test of homogeneity of variances
## 
## data:  m1$res and dados$promalin
## Bartlett's K-squared = 2.1761, df = 4, p-value = 0.7034
\end{verbatim}

\begin{Shaded}
\begin{Highlighting}[]
\KeywordTok{anova}\NormalTok{(m1)}
\end{Highlighting}
\end{Shaded}

\begin{verbatim}
## Analysis of Variance Table
## 
## Response: log(dados$peso)
##                Df   Sum Sq   Mean Sq F value   Pr(>F)   
## dados$promalin  4 0.036571 0.0091428  6.0129 0.004296 **
## Residuals      15 0.022808 0.0015205                    
## ---
## Signif. codes:  0 '***' 0.001 '**' 0.01 '*' 0.05 '.' 0.1 ' ' 1
\end{verbatim}

\hypertarget{pacote-para-analise-de-experimentos}{%
\section{Pacote para analise de experimentos}\label{pacote-para-analise-de-experimentos}}

\begin{Shaded}
\begin{Highlighting}[]
\KeywordTok{library}\NormalTok{(ExpDes.pt)}
\end{Highlighting}
\end{Shaded}

Conhecer o pacote ExpDes.pt

\begin{Shaded}
\begin{Highlighting}[]
\KeywordTok{ls}\NormalTok{(}\StringTok{"package:ExpDes.pt"}\NormalTok{)}
\end{Highlighting}
\end{Shaded}

\begin{verbatim}
##  [1] "anscombetukey"  "bartlett"       "ccboot"         "ccf"           
##  [5] "dbc"            "dic"            "dql"            "duncan"        
##  [9] "faixas"         "fat2.ad.dbc"    "fat2.ad.dic"    "fat2.dbc"      
## [13] "fat2.dic"       "fat3.ad.dbc"    "fat3.ad.dic"    "fat3.dbc"      
## [17] "fat3.dic"       "ginv"           "graficos"       "han"           
## [21] "lastC"          "layard"         "levene"         "lsd"           
## [25] "lsdb"           "oneilldbc"      "oneillmathews"  "order.group"   
## [29] "order.stat.SNK" "plotres"        "psub2.dbc"      "psub2.dic"     
## [33] "reg.nl"         "reg.poly"       "samiuddin"      "scottknott"    
## [37] "snk"            "tapply.stat"    "tukey"
\end{verbatim}

Utilizando o exemplo anterior.

\begin{Shaded}
\begin{Highlighting}[]
\NormalTok{x <-}\StringTok{ }\KeywordTok{dbc}\NormalTok{(}\DataTypeTok{trat =}\NormalTok{ dados}\OperatorTok{$}\NormalTok{promalin,}
          \DataTypeTok{bloco =}\NormalTok{ dados}\OperatorTok{$}\NormalTok{bloco,}
          \DataTypeTok{resp =} \KeywordTok{log}\NormalTok{(dados}\OperatorTok{$}\NormalTok{peso),}
          \DataTypeTok{quali =}\NormalTok{ T,}
          \DataTypeTok{mcomp =} \StringTok{"tukey"}\NormalTok{)}
\end{Highlighting}
\end{Shaded}

\begin{verbatim}
## ------------------------------------------------------------------------
## Quadro da analise de variancia
## ------------------------------------------------------------------------
##            GL       SQ        QM     Fc   Pr>Fc
## Tratamento  4 0.036571 0.0091428 5.7552 0.00800
## Bloco       3 0.003745 0.0012483 0.7858 0.52459
## Residuo    12 0.019063 0.0015886               
## Total      19 0.059379                         
## ------------------------------------------------------------------------
## CV = 0.8 %
## 
## ------------------------------------------------------------------------
## Teste de normalidade dos residuos 
## valor-p:  0.005994506 
## ATENCAO: a 5% de significancia, os residuos nao podem ser considerados normais!
## ------------------------------------------------------------------------
## 
## ------------------------------------------------------------------------
## Teste de homogeneidade de variancia 
## valor-p:  0.8927087 
## De acordo com o teste de oneillmathews a 5% de significancia, as variancias podem ser consideradas homogeneas.
## ------------------------------------------------------------------------
## 
## Teste de Tukey
## ------------------------------------------------------------------------
## Grupos Tratamentos Medias
## a     Testemunha      5.043544 
## ab    12.5    4.961465 
##  b    12.5+12.5   4.940843 
##  b    50.0    4.932278 
##  b    25.0    4.927864 
## ------------------------------------------------------------------------
\end{verbatim}

Carregar pacotes

\begin{Shaded}
\begin{Highlighting}[]
\KeywordTok{library}\NormalTok{(ggplot2)}
\KeywordTok{library}\NormalTok{(dplyr)}
\end{Highlighting}
\end{Shaded}

Calculo do erro

\begin{Shaded}
\begin{Highlighting}[]
\NormalTok{erro =}\StringTok{ }\KeywordTok{summarise}\NormalTok{(}\KeywordTok{group_by}\NormalTok{(dados, promalin), }
       \DataTypeTok{avg =} \KeywordTok{mean}\NormalTok{(peso), }\DataTypeTok{sd =} \KeywordTok{sd}\NormalTok{(peso))}
\end{Highlighting}
\end{Shaded}

Gerando gráfico

\begin{Shaded}
\begin{Highlighting}[]
\KeywordTok{ggplot}\NormalTok{(erro, }\KeywordTok{aes}\NormalTok{(promalin, avg, }\DataTypeTok{fill=}\NormalTok{promalin))}\OperatorTok{+}
\StringTok{  }\KeywordTok{geom_bar}\NormalTok{(}\DataTypeTok{stat=}\StringTok{"identity"}\NormalTok{)}\OperatorTok{+}
\StringTok{  }\KeywordTok{geom_errorbar}\NormalTok{(}\KeywordTok{aes}\NormalTok{(}\DataTypeTok{ymin=}\NormalTok{avg}\OperatorTok{-}\NormalTok{sd, }\DataTypeTok{ymax =}\NormalTok{avg}\OperatorTok{+}\NormalTok{sd), }\DataTypeTok{with=}\FloatTok{0.1}\NormalTok{, }\DataTypeTok{col=}\StringTok{"black"}\NormalTok{) }\OperatorTok{+}
\StringTok{    }\KeywordTok{xlab}\NormalTok{(}\StringTok{"Tratamentos"}\NormalTok{) }\OperatorTok{+}\StringTok{ }
\StringTok{    }\KeywordTok{ylab}\NormalTok{(}\StringTok{"Peso médio dos frutos (g)"}\NormalTok{) }\OperatorTok{+}\StringTok{ }
\StringTok{  }\KeywordTok{theme_bw}\NormalTok{() }\OperatorTok{+}\StringTok{ }
\StringTok{  }\KeywordTok{theme}\NormalTok{(}\DataTypeTok{legend.position=}\StringTok{"top"}\NormalTok{) }\OperatorTok{+}
\StringTok{      }\KeywordTok{annotate}\NormalTok{(}\StringTok{"text"}\NormalTok{, }\DataTypeTok{label=}\StringTok{"ab"}\NormalTok{, }\DataTypeTok{x=}\DecValTok{1}\NormalTok{, }\DataTypeTok{y=}\DecValTok{100}\NormalTok{, }\DataTypeTok{size =} \DecValTok{5}\NormalTok{)  }\OperatorTok{+}
\StringTok{      }\KeywordTok{annotate}\NormalTok{(}\StringTok{"text"}\NormalTok{, }\DataTypeTok{label=}\StringTok{"b"}\NormalTok{, }\DataTypeTok{x=}\DecValTok{2}\NormalTok{, }\DataTypeTok{y=}\DecValTok{100}\NormalTok{, }\DataTypeTok{size =} \DecValTok{5}\NormalTok{) }\OperatorTok{+}
\StringTok{      }\KeywordTok{annotate}\NormalTok{(}\StringTok{"text"}\NormalTok{, }\DataTypeTok{label=}\StringTok{"b"}\NormalTok{, }\DataTypeTok{x=}\DecValTok{3}\NormalTok{, }\DataTypeTok{y=}\DecValTok{100}\NormalTok{, }\DataTypeTok{size =} \DecValTok{5}\NormalTok{)  }\OperatorTok{+}
\StringTok{      }\KeywordTok{annotate}\NormalTok{(}\StringTok{"text"}\NormalTok{, }\DataTypeTok{label=}\StringTok{"b"}\NormalTok{, }\DataTypeTok{x=}\DecValTok{4}\NormalTok{, }\DataTypeTok{y=}\DecValTok{100}\NormalTok{, }\DataTypeTok{size =} \DecValTok{5}\NormalTok{)  }\OperatorTok{+}
\StringTok{      }\KeywordTok{annotate}\NormalTok{(}\StringTok{"text"}\NormalTok{, }\DataTypeTok{label=}\StringTok{"a"}\NormalTok{, }\DataTypeTok{x=}\DecValTok{5}\NormalTok{, }\DataTypeTok{y=}\DecValTok{100}\NormalTok{, }\DataTypeTok{size =} \DecValTok{5}\NormalTok{)  }\OperatorTok{+}
\StringTok{  }\KeywordTok{theme}\NormalTok{(}\DataTypeTok{legend.position=}\StringTok{"none"}\NormalTok{) }\OperatorTok{+}
\StringTok{  }\KeywordTok{labs}\NormalTok{(}\DataTypeTok{caption =} \StringTok{"Médias seguidas de mesma letra indicam diferença nula à 5%"}\NormalTok{)}
\end{Highlighting}
\end{Shaded}

\begin{verbatim}
## Warning: Ignoring unknown parameters: with
\end{verbatim}

\includegraphics{TudodoR_files/figure-latex/unnamed-chunk-303-1.pdf}

\hypertarget{teste-nuxe3o-parametrico}{%
\section{Teste não parametrico}\label{teste-nuxe3o-parametrico}}

As funções para comparações multiplas não-paramétricas incluídas no pacote agricolae são: \textbf{kruskal}, \textbf{waerden.test}, \textbf{friedman}, \textbf{durbin.test} e \textbf{Conover (1999)}.
Os testes não-paramétricos post hoc (kruskal, friedman, durbin e waerden) estão usando o critério a diferença menos significativa de Fisher (LSD).

Carregar pacote

\begin{Shaded}
\begin{Highlighting}[]
\KeywordTok{library}\NormalTok{(agricolae)}
\end{Highlighting}
\end{Shaded}

A função \texttt{kruskal} é usada para N amostras (N\textgreater{} 2), populações ou dados provenientes de um experimento aleatório (populações = tratamentos).

\begin{Shaded}
\begin{Highlighting}[]
\NormalTok{woutKruskal<-}\KeywordTok{with}\NormalTok{(dados,}\KeywordTok{kruskal}\NormalTok{(promalin, }\DataTypeTok{y =}\NormalTok{ peso}
\NormalTok{  ,}\DataTypeTok{p.adj=}\StringTok{"bon"}\NormalTok{,}\DataTypeTok{group=}\NormalTok{T, }\DataTypeTok{console=}\NormalTok{T))}
\end{Highlighting}
\end{Shaded}

\begin{verbatim}
## 
## Study: peso ~ promalin
## Kruskal-Wallis test's
## Ties or no Ties
## 
## Critical Value: 10.41429
## Degrees of freedom: 4
## Pvalue Chisq  : 0.03399839 
## 
## promalin,  means of the ranks
## 
##             peso r
## 12.5       12.00 4
## 12.5+12.5   7.75 4
## 25.0        7.50 4
## 50.0        7.00 4
## Testemunha 18.25 4
## 
## Post Hoc Analysis
## 
## P value adjustment method: bonferroni
## t-Student: 3.286039
## Alpha    : 0.05
## Minimum Significant Difference: 10.40002 
## 
## Treatments with the same letter are not significantly different.
## 
##             peso groups
## Testemunha 18.25      a
## 12.5       12.00     ab
## 12.5+12.5   7.75      b
## 25.0        7.50      b
## 50.0        7.00      b
\end{verbatim}

\begin{Shaded}
\begin{Highlighting}[]
\KeywordTok{print}\NormalTok{(woutKruskal}\OperatorTok{$}\NormalTok{group)}
\end{Highlighting}
\end{Shaded}

\begin{verbatim}
##             peso groups
## Testemunha 18.25      a
## 12.5       12.00     ab
## 12.5+12.5   7.75      b
## 25.0        7.50      b
## 50.0        7.00      b
\end{verbatim}

Gráficos

\begin{Shaded}
\begin{Highlighting}[]
\KeywordTok{par}\NormalTok{(}\DataTypeTok{mfrow=}\KeywordTok{c}\NormalTok{(}\DecValTok{2}\NormalTok{,}\DecValTok{2}\NormalTok{),}\DataTypeTok{mar=}\KeywordTok{c}\NormalTok{(}\DecValTok{3}\NormalTok{,}\DecValTok{3}\NormalTok{,}\DecValTok{1}\NormalTok{,}\DecValTok{1}\NormalTok{),}\DataTypeTok{cex=}\FloatTok{0.8}\NormalTok{)}
\KeywordTok{bar.group}\NormalTok{(woutKruskal}\OperatorTok{$}\NormalTok{group,}\DataTypeTok{ylim=}\KeywordTok{c}\NormalTok{(}\DecValTok{0}\NormalTok{,}\DecValTok{100}\NormalTok{), }\DataTypeTok{xlab =}\StringTok{"promalin"}\NormalTok{)}
\KeywordTok{bar.group}\NormalTok{(woutKruskal}\OperatorTok{$}\NormalTok{group,}\DataTypeTok{xlim=}\KeywordTok{c}\NormalTok{(}\DecValTok{0}\NormalTok{,}\DecValTok{100}\NormalTok{),}\DataTypeTok{horiz =} \OtherTok{TRUE}\NormalTok{)}
\KeywordTok{plot}\NormalTok{(woutKruskal)}
\KeywordTok{plot}\NormalTok{(woutKruskal,}\DataTypeTok{variation=}\StringTok{"IQR"}\NormalTok{,}\DataTypeTok{horiz =} \OtherTok{TRUE}\NormalTok{)}
\end{Highlighting}
\end{Shaded}

\includegraphics{TudodoR_files/figure-latex/unnamed-chunk-306-1.pdf}

A função \texttt{friedman} é usada para análise de tratamentos do estudo randomizado
de bloco completo, onde a resposta não pode ser tratada através da análise de variância.

\begin{Shaded}
\begin{Highlighting}[]
\NormalTok{woutfriedman <-}\StringTok{ }\NormalTok{out<-}\KeywordTok{with}\NormalTok{(dados,}\KeywordTok{friedman}\NormalTok{(bloco,promalin, peso,}\DataTypeTok{alpha=}\FloatTok{0.05}\NormalTok{, }\DataTypeTok{group=}\NormalTok{T,}
  \DataTypeTok{console=}\OtherTok{TRUE}\NormalTok{))}
\end{Highlighting}
\end{Shaded}

\begin{verbatim}
## 
## Study: peso ~ bloco + promalin 
## 
## promalin,  Sum of the ranks
## 
##            peso r
## 12.5          8 4
## 12.5+12.5     8 4
## 25.0         10 4
## 50.0         14 4
## Testemunha   20 4
## 
## Friedman's Test
## ===============
## Adjusted for ties
## Critical Value: 10.4
## P.Value Chisq: 0.0342027
## F Value: 5.571429
## P.Value F: 0.009007502 
## 
## Post Hoc Analysis
## 
## Alpha: 0.05 ; DF Error: 12
## t-Student: 2.178813
## LSD: 6.656383 
## 
## Treatments with the same letter are not significantly different.
## 
##            Sum of ranks groups
## Testemunha           20      a
## 50.0                 14     ab
## 25.0                 10      b
## 12.5                  8      b
## 12.5+12.5             8      b
\end{verbatim}

Grafico

\begin{Shaded}
\begin{Highlighting}[]
\KeywordTok{par}\NormalTok{(}\DataTypeTok{mfrow=}\KeywordTok{c}\NormalTok{(}\DecValTok{2}\NormalTok{,}\DecValTok{2}\NormalTok{),}\DataTypeTok{mar=}\KeywordTok{c}\NormalTok{(}\DecValTok{3}\NormalTok{,}\DecValTok{3}\NormalTok{,}\DecValTok{1}\NormalTok{,}\DecValTok{1}\NormalTok{),}\DataTypeTok{cex=}\FloatTok{0.8}\NormalTok{)}
\KeywordTok{bar.group}\NormalTok{(woutfriedman}\OperatorTok{$}\NormalTok{group,}\DataTypeTok{ylim=}\KeywordTok{c}\NormalTok{(}\DecValTok{0}\NormalTok{,}\DecValTok{100}\NormalTok{), }\DataTypeTok{xlab =}\StringTok{"promalin"}\NormalTok{)}
\KeywordTok{bar.group}\NormalTok{(woutfriedman}\OperatorTok{$}\NormalTok{group,}\DataTypeTok{xlim=}\KeywordTok{c}\NormalTok{(}\DecValTok{0}\NormalTok{,}\DecValTok{100}\NormalTok{),}\DataTypeTok{horiz =} \OtherTok{TRUE}\NormalTok{)}
\KeywordTok{plot}\NormalTok{(woutfriedman)}
\KeywordTok{plot}\NormalTok{(woutfriedman,}\DataTypeTok{variation=}\StringTok{"IQR"}\NormalTok{,}\DataTypeTok{horiz =} \OtherTok{TRUE}\NormalTok{)}
\end{Highlighting}
\end{Shaded}

\includegraphics{TudodoR_files/figure-latex/unnamed-chunk-308-1.pdf}

\hypertarget{exercicio-1}{%
\section{Exercicio 1}\label{exercicio-1}}

Obtenha: Analise exploratoria, Analise de variancia, teste de comparação multipla, e recomendações.

Comparação de métodos de Semeadura do Mamoeiro

Estudo realizado em Jaboticabal - SP por Ruiz (1977) que comparou métodos de semeadura no mamoeiro. O experimento foi instalado em delineamento de blocos casualizados, com 4 repetições, avaliando 3 métodos de semeadura. Foram avaliadas duas unidades experimentais por método em cada bloco.

Importando dados

\begin{Shaded}
\begin{Highlighting}[]
\NormalTok{dados <-}\StringTok{ }\KeywordTok{read.table}\NormalTok{(}\StringTok{"https://www.dropbox.com/s/40m95attfw2fdh2/BanzattoQd4.7.1.txt?dl=1"}\NormalTok{) }
\end{Highlighting}
\end{Shaded}

Conferir se temos fatores para fazer a análise de variância

\begin{Shaded}
\begin{Highlighting}[]
\KeywordTok{str}\NormalTok{(dados)}
\end{Highlighting}
\end{Shaded}

\begin{verbatim}
## 'data.frame':    24 obs. of  3 variables:
##  $ bloco : Factor w/ 4 levels "I","II","III",..: 1 1 2 2 3 3 4 4 1 1 ...
##  $ semead: Factor w/ 3 levels "Direta no campo",..: 1 1 1 1 1 1 1 1 2 2 ...
##  $ altura: num  136.1 105.3 98.8 86.8 108.8 ...
\end{verbatim}

Gráficos

\begin{Shaded}
\begin{Highlighting}[]
\KeywordTok{addmargins}\NormalTok{(}\KeywordTok{with}\NormalTok{(dados,}
  \KeywordTok{tapply}\NormalTok{(}\DataTypeTok{X =}\NormalTok{ altura,}
    \DataTypeTok{INDEX =} \KeywordTok{list}\NormalTok{(semead, bloco),}
    \DataTypeTok{FUN =}\NormalTok{ sum)))}
\end{Highlighting}
\end{Shaded}

\begin{verbatim}
##                     I    II   III    IV    Sum
## Direta no campo 241.4 185.6 218.5 162.9  808.4
## Recip. ao sol   157.7 120.7 129.0  80.1  487.5
## Recip. ripado   123.5 127.1 132.9 109.8  493.3
## Sum             522.6 433.4 480.4 352.8 1789.2
\end{verbatim}

\begin{Shaded}
\begin{Highlighting}[]
\KeywordTok{xyplot}\NormalTok{(altura }\OperatorTok{~}\StringTok{ }\NormalTok{semead, }\DataTypeTok{data =}\NormalTok{ dados,}
  \DataTypeTok{groups =}\NormalTok{ bloco, }\DataTypeTok{type =} \KeywordTok{c}\NormalTok{(}\StringTok{"p"}\NormalTok{, }\StringTok{"a"}\NormalTok{),}
  \DataTypeTok{xlab =} \StringTok{"Método de semeadura de mamoeiro"}\NormalTok{,}
  \DataTypeTok{ylab =} \StringTok{"Altura média de planta de mamoeiro aos 147 DAS (cm)"}\NormalTok{,}
  \DataTypeTok{auto.key =} \KeywordTok{list}\NormalTok{(}\DataTypeTok{title =} \StringTok{"Bloco"}\NormalTok{, }\DataTypeTok{cex.title =} \DecValTok{1}\NormalTok{, }\DataTypeTok{columns =} \DecValTok{2}\NormalTok{))}
\end{Highlighting}
\end{Shaded}

\includegraphics{TudodoR_files/figure-latex/unnamed-chunk-312-1.pdf}

Análise de Variância

\begin{Shaded}
\begin{Highlighting}[]
\NormalTok{m0 <-}\StringTok{ }\KeywordTok{aov}\NormalTok{(altura}\OperatorTok{~}\NormalTok{bloco}\OperatorTok{+}\NormalTok{semead, }\DataTypeTok{data=}\NormalTok{dados)}
\KeywordTok{class}\NormalTok{(m0)}
\end{Highlighting}
\end{Shaded}

\begin{verbatim}
## [1] "aov" "lm"
\end{verbatim}

\begin{Shaded}
\begin{Highlighting}[]
\KeywordTok{anova}\NormalTok{(m0)}
\end{Highlighting}
\end{Shaded}

\begin{verbatim}
## Analysis of Variance Table
## 
## Response: altura
##           Df Sum Sq Mean Sq F value    Pr(>F)    
## bloco      3 2648.2   882.7  7.2162  0.002219 ** 
## semead     2 8429.1  4214.6 34.4535 7.014e-07 ***
## Residuals 18 2201.9   122.3                      
## ---
## Signif. codes:  0 '***' 0.001 '**' 0.01 '*' 0.05 '.' 0.1 ' ' 1
\end{verbatim}

\begin{Shaded}
\begin{Highlighting}[]
\KeywordTok{summary}\NormalTok{(m0)}
\end{Highlighting}
\end{Shaded}

\begin{verbatim}
##             Df Sum Sq Mean Sq F value   Pr(>F)    
## bloco        3   2648     883   7.216  0.00222 ** 
## semead       2   8429    4215  34.453 7.01e-07 ***
## Residuals   18   2202     122                     
## ---
## Signif. codes:  0 '***' 0.001 '**' 0.01 '*' 0.05 '.' 0.1 ' ' 1
\end{verbatim}

Checagem gráfica

\begin{Shaded}
\begin{Highlighting}[]
\KeywordTok{par}\NormalTok{(}\DataTypeTok{mfrow=}\KeywordTok{c}\NormalTok{(}\DecValTok{2}\NormalTok{,}\DecValTok{2}\NormalTok{))}
\KeywordTok{plot}\NormalTok{(m0)}
\end{Highlighting}
\end{Shaded}

\includegraphics{TudodoR_files/figure-latex/unnamed-chunk-314-1.pdf}

\begin{Shaded}
\begin{Highlighting}[]
\KeywordTok{layout}\NormalTok{(}\DecValTok{1}\NormalTok{)}
\end{Highlighting}
\end{Shaded}

Teste das pressuposições de normalidade de homocedasticidade

\begin{Shaded}
\begin{Highlighting}[]
\KeywordTok{shapiro.test}\NormalTok{(}\KeywordTok{residuals}\NormalTok{(m0))}
\end{Highlighting}
\end{Shaded}

\begin{verbatim}
## 
##  Shapiro-Wilk normality test
## 
## data:  residuals(m0)
## W = 0.95197, p-value = 0.2988
\end{verbatim}

\begin{Shaded}
\begin{Highlighting}[]
\KeywordTok{bartlett.test}\NormalTok{(}\KeywordTok{residuals}\NormalTok{(m0)}\OperatorTok{~}\NormalTok{dados}\OperatorTok{$}\NormalTok{semead)}
\end{Highlighting}
\end{Shaded}

\begin{verbatim}
## 
##  Bartlett test of homogeneity of variances
## 
## data:  residuals(m0) by dados$semead
## Bartlett's K-squared = 4.5219, df = 2, p-value = 0.1043
\end{verbatim}

Teste de médias

\textbf{Teste de Tukey}

\begin{Shaded}
\begin{Highlighting}[]
\KeywordTok{require}\NormalTok{(agricolae)}
\end{Highlighting}
\end{Shaded}

\begin{Shaded}
\begin{Highlighting}[]
\NormalTok{tu <-}\StringTok{ }\KeywordTok{with}\NormalTok{(dados, }\KeywordTok{HSD.test}\NormalTok{(altura, semead,}
\DataTypeTok{DFerror=}\KeywordTok{df.residual}\NormalTok{(m0),}
\DataTypeTok{MSerror=}\KeywordTok{deviance}\NormalTok{(m0)}\OperatorTok{/}\KeywordTok{df.residual}\NormalTok{(m0)))}
\end{Highlighting}
\end{Shaded}

\begin{Shaded}
\begin{Highlighting}[]
\KeywordTok{plot}\NormalTok{(tu)}
\end{Highlighting}
\end{Shaded}

\includegraphics{TudodoR_files/figure-latex/unnamed-chunk-318-1.pdf}

\begin{Shaded}
\begin{Highlighting}[]
\KeywordTok{print}\NormalTok{(tu)}
\end{Highlighting}
\end{Shaded}

\begin{verbatim}
## $statistics
##    MSerror Df  Mean       CV      MSD
##   122.3258 18 74.55 14.83581 14.11359
## 
## $parameters
##    test name.t ntr StudentizedRange alpha
##   Tukey semead   3         3.609304  0.05
## 
## $means
##                   altura       std r  Min   Max    Q25    Q50     Q75
## Direta no campo 101.0500 19.263956 8 70.5 136.1 91.000 102.05 109.025
## Recip. ao sol    60.9375 15.187489 8 36.3  79.8 53.175  63.25  69.650
## Recip. ripado    61.6625  9.544922 8 43.7  77.1 58.575  62.95  65.425
## 
## $comparison
## NULL
## 
## $groups
##                   altura groups
## Direta no campo 101.0500      a
## Recip. ripado    61.6625      b
## Recip. ao sol    60.9375      b
## 
## attr(,"class")
## [1] "group"
\end{verbatim}

\begin{Shaded}
\begin{Highlighting}[]
\KeywordTok{require}\NormalTok{(dplyr)}
\KeywordTok{require}\NormalTok{(ggplot2)}
\NormalTok{  erro =}\StringTok{ }\KeywordTok{summarise}\NormalTok{(}\KeywordTok{group_by}\NormalTok{(dados, semead), }
    \DataTypeTok{avg =} \KeywordTok{mean}\NormalTok{(altura), }\DataTypeTok{sd =} \KeywordTok{sd}\NormalTok{(altura))}

  
  \KeywordTok{ggplot}\NormalTok{(erro, }\KeywordTok{aes}\NormalTok{(semead, avg, }\DataTypeTok{fill=}\NormalTok{semead))}\OperatorTok{+}
\StringTok{    }\KeywordTok{geom_bar}\NormalTok{(}\DataTypeTok{stat=}\StringTok{"identity"}\NormalTok{)}\OperatorTok{+}
\StringTok{    }\KeywordTok{geom_errorbar}\NormalTok{(}\KeywordTok{aes}\NormalTok{(}\DataTypeTok{ymin=}\NormalTok{avg}\OperatorTok{-}\NormalTok{sd, }\DataTypeTok{ymax =}\NormalTok{avg}\OperatorTok{+}\NormalTok{sd), }\DataTypeTok{with=}\FloatTok{0.1}\NormalTok{, }\DataTypeTok{col=}\StringTok{"black"}\NormalTok{) }\OperatorTok{+}
\StringTok{    }\KeywordTok{xlab}\NormalTok{(}\StringTok{"Tratamentos"}\NormalTok{) }\OperatorTok{+}\StringTok{ }
\StringTok{    }\KeywordTok{ylab}\NormalTok{(}\StringTok{"Altura média de planta de mamoeiro aos 147 DAS (cm)"}\NormalTok{) }\OperatorTok{+}\StringTok{ }
\StringTok{    }\KeywordTok{theme_bw}\NormalTok{() }\OperatorTok{+}\StringTok{ }
\StringTok{    }\KeywordTok{theme}\NormalTok{(}\DataTypeTok{legend.position=}\StringTok{"top"}\NormalTok{) }\OperatorTok{+}
\StringTok{    }\KeywordTok{annotate}\NormalTok{(}\StringTok{"text"}\NormalTok{, }\DataTypeTok{label=}\NormalTok{tu}\OperatorTok{$}\NormalTok{groups}\OperatorTok{$}\NormalTok{groups[}\DecValTok{1}\NormalTok{], }\DataTypeTok{x=}\DecValTok{1}\NormalTok{, }\DataTypeTok{y=}\DecValTok{20}\NormalTok{, }\DataTypeTok{size =} \DecValTok{5}\NormalTok{)  }\OperatorTok{+}
\StringTok{    }\KeywordTok{annotate}\NormalTok{(}\StringTok{"text"}\NormalTok{, }\DataTypeTok{label=}\NormalTok{tu}\OperatorTok{$}\NormalTok{groups}\OperatorTok{$}\NormalTok{groups[}\DecValTok{2}\NormalTok{], }\DataTypeTok{x=}\DecValTok{2}\NormalTok{, }\DataTypeTok{y=}\DecValTok{20}\NormalTok{, }\DataTypeTok{size =} \DecValTok{5}\NormalTok{)  }\OperatorTok{+}
\StringTok{    }\KeywordTok{annotate}\NormalTok{(}\StringTok{"text"}\NormalTok{, }\DataTypeTok{label=}\NormalTok{tu}\OperatorTok{$}\NormalTok{groups}\OperatorTok{$}\NormalTok{groups[}\DecValTok{3}\NormalTok{], }\DataTypeTok{x=}\DecValTok{3}\NormalTok{, }\DataTypeTok{y=}\DecValTok{20}\NormalTok{, }\DataTypeTok{size =} \DecValTok{5}\NormalTok{)  }\OperatorTok{+}
\StringTok{    }
\StringTok{    }\KeywordTok{theme}\NormalTok{(}\DataTypeTok{legend.position=}\StringTok{"none"}\NormalTok{) }\OperatorTok{+}
\StringTok{    }\KeywordTok{labs}\NormalTok{(}\DataTypeTok{caption =} \StringTok{"Médias seguidas de mesma letra indicam diferença nula à 5%"}\NormalTok{)}
\end{Highlighting}
\end{Shaded}

\begin{verbatim}
## Warning: Ignoring unknown parameters: with
\end{verbatim}

\includegraphics{TudodoR_files/figure-latex/unnamed-chunk-320-1.pdf}

\textbf{Teste de Scott-Knott}

\begin{Shaded}
\begin{Highlighting}[]
  \KeywordTok{library}\NormalTok{(ScottKnott)}
\NormalTok{  sk <-}\StringTok{ }\KeywordTok{SK}\NormalTok{(}\DataTypeTok{x=}\NormalTok{dados, }\DataTypeTok{y=}\NormalTok{dados}\OperatorTok{$}\NormalTok{altura, }\DataTypeTok{model=}\StringTok{"altura~bloco+semead"}\NormalTok{, }\DataTypeTok{which=}\StringTok{"semead"}\NormalTok{)}
  \KeywordTok{summary}\NormalTok{(sk)}
\end{Highlighting}
\end{Shaded}

\begin{verbatim}
##           Levels    Means SK(5%)
##  Direta no campo 101.0500      a
##    Recip. ripado  61.6625      b
##    Recip. ao sol  60.9375      b
\end{verbatim}

\begin{Shaded}
\begin{Highlighting}[]
\KeywordTok{print}\NormalTok{(sk)}
\end{Highlighting}
\end{Shaded}

\begin{verbatim}
## $av
## Call:
##    aov(formula = altura ~ bloco + semead, data = dat)
## 
## Terms:
##                    bloco   semead Residuals
## Sum of Squares  2648.193 8429.103  2201.864
## Deg. of Freedom        3        2        18
## 
## Residual standard error: 11.0601
## Estimated effects may be unbalanced
## 
## $groups
## [1] 1 2 2
## 
## $nms
## [1] "Direta no campo" "Recip. ao sol"   "Recip. ripado"  
## 
## $ord
## [1] 1 3 2
## 
## $m.inf
##                     mean  min   max
## Direta no campo 101.0500 70.5 136.1
## Recip. ripado    61.6625 43.7  77.1
## Recip. ao sol    60.9375 36.3  79.8
## 
## $sig.level
## [1] 0.05
## 
## attr(,"class")
## [1] "SK"   "list"
\end{verbatim}

\hypertarget{referuxeancia-4}{%
\section{Referência}\label{referuxeancia-4}}

MELO, M. P.; PETERNELI, L. A. Conhecendo o R: Um visão mais que estatística. Viçosa, MG: UFV, 2013. 222p. Cap. 1.

BANZATTO, D. A; KRONKA, S. N. Experimentação agrícola. Jaboticabal, SP: FUNEP, 2006, 237p.

ZEVIANI, W. M. Estatística Básica e Experimentação no R. 45p.

Site: \url{http://www.leg.ufpr.br/~paulojus/}

  \bibliography{book.bib,packages.bib}

\end{document}
